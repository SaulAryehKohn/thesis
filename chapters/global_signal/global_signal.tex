\chapter{Interferometric sensitivity to the monopole moment}
\label{chapter:TAV}

The 21 cm signal from the Epoch of Reionization (EoR) contains a wealth of
information about the Universe at these early times, and can provide key insight
inaccessible to other observational techniques \citep{Loeb.12}. 

In particular, the sky-averaged 21 cm signal, the so-called global signal, directly
contains key information about the thermal history of the intergalactic medium
(IGM) as a function of redshift \citep{Pritchard.10}. Historically,
observational efforts to detect the global signal have been ``single dish''
experiments, where a single receiving element is characterized to high
precision, and then operated in an effort to detect the EoR signal as a function
of frequency (and thus redshift). Experiments such as the Experiment to Detect
the Global EoR Step (EDGES, \citealt{Bowman.10}), Sonda Cosmol\'{o}gica
de las Islas para la Detecc\'{i}on de Hidr\'{o}geno Neutro (SCI-HI,
\citealt{Voytek.14}), and the Cosmic Twilight Polarimeter (CPT,
\citealt{Nhan.16}), have been proposed or constructed seeking to measure
the global signal from the Dark Ages or the EoR, through a deep understanding of
the properties of the instruments. In all cases, the instruments consist of a
single-element, and much of the observational effort contributes toward a
thorough understanding of systematic uncertainties of the instrument. The signal
from the Dark Ages and EoR is thought to be 4--5 orders of magnitude fainter
than nearby bright foregrounds, such as galactic synchrotron radiation
\citep{McQuinn.07}. As such, an exquisite understanding of the correlated
noise in these instruments is of the utmost importance. Largely, these
experiments have not yet detected a feature in frequency-space that can clearly
be interpreted as a detection of either the Dark Ages or EoR global signal. To
date, EDGES has provided a lower-limit on the duration of reionization of
$\Delta z \geq 0.06$ \citep{Bowman.10}.
