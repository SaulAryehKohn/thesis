\chapter{Conclusions}
\label{chapter:conc}

% In this work, I have shown.... 
% - that the EoR is cool and worth looking for
% - inherent chromatism of the instrument is a challenge, but gives us our greatest tool of wedge+EoR window
% - MUST MAINTAIN SMOOTHNESS
% - astrophysical polarization (FRot'd) violates this, must be checked for 
%
% - the ins and outs of working with wide-field polarized interferometric measurements
% -- compression, high-dynamic range calibration, images, the ionosphere 
% -- wide and shallow (PAPER-32) vs narrow and deep (PAPER-128) integrations on Fourier representation of polarized data
% -- the first published power spectra from HERA. Everything going to plan.
% 
% interesting new ways to use the data:
% global signal
% cross correlations
% ML

% where do we go from here?
%
% how smooth are sync foregrounds really - my paper idea
% jackknife on ionosphere
% polarization needs to be characterized down to EoR dynamic ranges: WE NEED LITTLE P
% imaging and subtraction of individual Mueller components is low hanging fruit
% solar minimum for HERA makes the ionosphere unimportant, but could be for long term projects e.g. SKA Low
% polcal has great potential, requires characterization
% global signal needs dedicated experiment -- cite something from Kara -- need to account for higher ell modes if we use our path
% trispectra may be one of the most viable options for cross-correlation studies in the short term
% cross correlations with JWST - map ultra deep fields in to k space just as we have with kSZ. Do we overlap?
% ML has huge potential -- clustering on data_prep_and_proc statistics, RFI flagging, source subtraction...
% let's detect the EoR!!! GO TEAM HERA.
