\chapter{Peering through the EoR Window}
\label{chapter:eor_window_theory}

Chapter~\ref{chapter:eor_intro} argued the promise of direct observations of {\sc hi} during the EoR. A direct detection has not yet been made, largely due to the overwhelming power of foregrounds compared to the target signal. As shown in Chapter~\ref{chapter:astro_rad}, foreground radiation is a factor of roughly $10^4 -- 10^5$ times brighter than predicted 21\,cm anisotropies .  In the past decade, however, enormous improvements have been made in understanding how to decontaminate interferometric visibilities, and excavate the target signal. The leverage astronomers have to use is the exceptional smoothness of low-frequency synchrotron radiation, the dominant foreground emission mechanism. In this Chapter, I review the `Foreground Wedge \& EoR Window' paradigm used to delineate foreground power from noise and {\sc hi} emission. In Section

\section{The Wedge and the Window}  % ugh I don't like this section title
\label{sec:wedge_and_window}



%(Datta et al. 2010; Parsons et al. 2012b; Vedantham et al. 2012; Morales et al. 2012; Hazelton et al. 2013; Pober et al. 2013; Thyagarajan et al. 2013; Liu et al. 2014a,b).

% foreground wedge and EoR window -- Nithya predictions -- importance of beam



\subsection{Foreground avoidance}
% Foreground avoidence: delay spectrum; basic PAPER and HERA results (including Kohn et al. 2016, 2018)
\subsubsection{Power spectra}
% Define power spectra in the delay paradigm
\subsubsection{Noise spectra}
% define noise spectrum

\subsection{Foreground subtraction}
% Foreground subtraction: theory and LOFAR results

\subsection{Hybrid methods}
% Hybrid approach: MWA

\section{The Problem of Polarization}
% polarization in wedge space -- my nice freq vs fourier space diagram? -- basically everything from Moore 2013


In this Part, I have recorded my efforts to reduce interferometric observations from PAPER and HERA, taking as much care as possible to avoid introducing spectral structure to the visibilities. My work has centered around the problem of polarization in the EoR window paradigm. In the following Chapters, I discuss quality assurance metrics and data compression (Chapter~\ref{chapter:data_prep_and_proc}), polarized calibration (Chapter~\ref{chapter:polcal}), the effect of the ionosphere on polarized power spectra (Chapter~\ref{chapter:ionosphere}), and three successively deeper integrations on polarized power, concentrating on successively thinner slices of $k_{\perp}$ (Chapters~\ref{chapter:eor_window_paper32img} -- \ref{chapter:eor_window_psa128}).