\chapter{The Epoch of Reionization}
\label{chapter:eor_intro}

% what is the EoR -- history (global signal history)

\section{The promise of direct measurements of the EoR}

% why is it interesting -- sensitivity to astrophysics & cosmology

\subsection{Statistical detections -- power spectra}
% current state of the art -- power spectra -- (limits from PAPER, LOFAR, MWA).


\subsection{Global Signal}
% spectrum
% global signal - EDGES

The 21 cm signal from the Epoch of Reionization (EoR) contains a wealth of
information about the Universe at these early times, and can provide key insight
inaccessible to other observational techniques \citep{Loeb.12}. 

In particular, the sky-averaged 21 cm signal, the so-called global signal, directly
contains key information about the thermal history of the intergalactic medium
(IGM) as a function of redshift \citep{Pritchard.10}. Historically,
observational efforts to detect the global signal have been ``single dish''
experiments, where a single receiving element is characterized to high
precision, and then operated in an effort to detect the EoR signal as a function
of frequency (and thus redshift). Experiments such as the Experiment to Detect
the Global EoR Step (EDGES, \citealt{Bowman.10}), Sonda Cosmol\'{o}gica
de las Islas para la Detecc\'{i}on de Hidr\'{o}geno Neutro (SCI-HI,
\citealt{Voytek.14}), and the Cosmic Twilight Polarimeter (CPT,
\citealt{Nhan.16}), have been proposed or constructed seeking to measure
the global signal from the Dark Ages or the EoR, through a deep understanding of
the properties of the instruments. In all cases, the instruments consist of a
single-element, and much of the observational effort contributes toward a
thorough understanding of systematic uncertainties of the instrument. The signal
from the Dark Ages and EoR is thought to be 4--5 orders of magnitude fainter
than nearby bright foregrounds, such as galactic synchrotron radiation
\citep{McQuinn.07}. As such, an exquisite understanding of the correlated
noise in these instruments is of the utmost importance. Largely, these
experiments have not yet detected a feature in frequency-space that can clearly
be interpreted as a detection of either the Dark Ages or EoR global signal. To
date, EDGES has provided a lower-limit on the duration of reionization of
$\Delta z \geq 0.06$ \citep{Bowman.10}.

\subsection{Imaging}
\subsection{Challenges, briefly}
% challenges, briefly, throwing-off to other chapters:
%% v. faint compared to foregrounds
%% v. faint compared to system noise
%% => QA over long seasons
%% unknown spectrum (unlike CMB)
%% polarization

\section{Indirect probes}
% current (CMB tau, high-z galaxies, HI limits, kSZ) and future (HI, CO, C+ intensity mapping, extreme deep fields from JWST) probes



% in the following chapters I will speak about...


\section{This thesis}

Everything in this work -- algorithmic development, mathematical theory, observations -- was carried-out in order to facilitate the detection of the EoR. While these efforts took many forms, they shared that singular motivation of moving the field forward towards a detection of {\sc hi} at cosmological distances. 

This thesis is divided into three parts. Part {\sc i} is devoted to introducing concepts used throughout this work and building a mathematical formalism around those concepts. 
Chapter~\ref{chapter:astro_rad} reviews astrophysical mechanisms for producing polarized and unpolarized radiation at low radio frequencies. 
Chapter~\ref{chapter:interferometry} builds a formalism around measuring low frequency radio waves with interferometers (and the challenges associated with accurately measuring polarized radiation), and Chapter~\ref{chapter:instruments} introduces the instruments used throughout this work.

In Part {\sc ii} I present the bulk of my efforts: building an understanding of the imprint of the polarized sky, and the instrument itself, in the Fourier space used to set limits on the EoR power spectrum. 
Chapter~\ref{chapter:eor_window_theory} reviews the current theory and major results of mapping low frequency interferometric measurements into Fourier space. 
Chapter~\ref{chapter:data_prep_and_proc} details several required quality assurance and compression steps that must be taken to clean and interact with the data. Building from clean data, Chapter~\ref{chapter:polcal} presents new algorithms developed to calibrate the measurements.
Chapter~\ref{chapter:ionosphere} discusses the impact of Earth's ionosphere on our measurements.
In Chapters~\ref{chapter:eor_window_paper32img}, \ref{chapter:eor_window_HERA} and \ref{chapter:eor_window_psa128} I present successively-deeper integrations on polarized foregrounds in successively-narrower regions of Fourier space.

Part {\sc iii} explores other uses of EoR measurements, beyond detection of the power spectrum. In Chapter~\ref{chapter:TAV}, I discuss the potential of using long time-averages of interferometric measurements to measure some component of the monopole moment of the sky. In Chapter~\ref{chapter:ksz_21cm}, I present a new formalism for cross-correlating 21\,cm emission and CMB anisotropies in Fourier space. Chapter~\ref{chapter:hera_ml} describes my initial investigations into utilizing deep learning techniques for recovering cosmological parameters from simulated EoR measurements. I conclude in Chapter~\ref{chapter:conc}.
