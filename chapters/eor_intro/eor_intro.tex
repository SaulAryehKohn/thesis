\chapter{The Epoch of Reionization}
\label{chapter:eor_intro}

% what is the EoR
% why is it interesting
% current state of the art -- power spectra -- (limits from PAPER, LOFAR, MWA).
% current (CMB, high-z galaxies, HI limits) and future (HI, CO, C+ intensity mapping, extreme deep fields from JWST) probes

% challenges, briefly, throwing-off to other chapters:
%% v. faint compared to foregrounds
%% v. faint compared to system noise
%% QA over long seasons
%% unknown spectrum (unlike CMB)

% in the following chapters I will speak about...


\section{This thesis}

Everything in this work -- algorithmic development, mathematical theory, observations -- was carried-out in order to facilitate the detection of the EoR. While these efforts took many forms, they shared that singular motivation of moving the field forward towards a detection of {\sc hi} at cosmological distances. 

This thesis is divided into three parts. Part {\sc i} is devoted to introducing concepts used throughout this work and building a mathematical formalism around those concepts. 
Chapter~\ref{chapter:astro_rad} reviews astrophysical mechanisms for producing polarized and unpolarized radiation at low radio frequencies. 
Chapter~\ref{chapter:interferometry} builds a formalism around measuring low frequency radio waves with interferometers (and the challenges associated with accurately measuring polarized radiation), and Chapter~\ref{chapter:instruments} introduces the instruments used throughout this work.

In Part {\sc ii} I present the bulk of my efforts: building an understanding of the imprint of the polarized sky, and the instrument itself, in the Fourier space used to set limits on the EoR power spectrum. 
Chapter~\ref{chapter:eor_window_theory} reviews the current theory and major results of mapping low frequency interferometric measurements into Fourier space. 
Chapter~\ref{chapter:data_prep_and_proc} details several required quality assurance and compression steps that must be taken to clean and interact with the data. Building from clean data, Chapter~\ref{chapter:polcal} presents new algorithms developed to calibrate the measurements.
Chapter~\ref{chapter:ionosphere} discusses the impact of Earth's ionosphere on our measurements.
In Chapters~\ref{chapter:eor_window_paper32img}, \ref{chapter:eor_window_HERA} and \ref{chapter:eor_window_psa128} I present successively-deeper integrations on polarized foregrounds in Fourier space.

Part {\sc iii} explores other uses of EoR measurements, beyond detection of the power spectrum. In Chapter~\ref{chapter:TAV}, I discuss the potential of using long time-averages of interferometric measurements to measure some component of the monopole moment of the sky. In Chapter~\ref{chapter:ksz_21cm}, I present a new formalism for cross-correlating 21\,cm emission and CMB anisotropies in Fourier space. Chapter~\ref{chapter:hera_ml} describes my initial investigations into utilizing deep learning techniques for recovering cosmological parameters from simulated EoR measurements. I conclude in Chapter~\ref{chapter:conc}.


