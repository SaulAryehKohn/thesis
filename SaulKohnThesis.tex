\documentclass[12pt,twoside,openany]{book}
\usepackage{graphicx}
\usepackage{epsfig}
\usepackage{color}
\usepackage{amsmath,amssymb}
\usepackage{float}
\usepackage[hyphens]{url}
%\usepackage{bigints,relsize}
\usepackage{mathrsfs,mathtools,xfrac}
%\usepackage{mathtools,xfrac}
%\usepackage{subfig} % needed for proper size caption in Table A.1
\usepackage[T1]{fontenc}
\usepackage{mathptmx} % needed to make text and equations in Times New Roman 
%\usepackage{times}
\usepackage{aastex_hack,natbib}
\usepackage{deluxetable}
%\usepackage{tabularx}
\usepackage{rotating,rotate}
\usepackage{pdflscape,longtable}
\usepackage[toc,page]{appendix}
%\usepackage{fixmainmatter} % I created this pkg to fix the blank page issue in \mainmatter
\usepackage{rotating} % Provides {sideways}{sidewaysfigure}{sidewaystable} environments
%\documentclass{aastex}
\usepackage{grffile}
%\usepackage{epstopdf}
\usepackage{lscape}
\usepackage{natbib}
\usepackage{xr}
\usepackage{url}
%\usepackage{hyperref}
\usepackage{caption}
\usepackage[caption=false]{subfig}
%\usepackage{float,lscape}
%\usepackage{pdflscape}
%\usepackage{equation}
%\usepackage{fancyhdr}
%\captionsetup[deluxetable]{labelformat=empty}
\usepackage{multirow}
\usepackage{lscape}
\usepackage{bm}
\usepackage{cancel}
\usepackage{epstopdf}
\usepackage{accents}
\usepackage{multicol}

% XXX CHECK THIS IS OK!
\bibstyle{apj_w_etal}


\DeclareMathAlphabet{\mathcal}{OMS}{cmsy}{m}{n}
%\setlength{\topmargin}{0.9in}
\setlength{\hoffset}{0.05in}
\setlength{\textheight}{8.5in}
\setlength{\headheight}{0in}
\setlength{\headsep}{-.22in}
\setlength{\oddsidemargin}{0.55in}
\setlength{\evensidemargin}{0.55in}
\setlength{\textwidth}{5.9in}
\newcommand{\doublespaced}{\renewcommand{\baselinestretch}{2}\normalfont}
\newcommand{\singlespaced}{\renewcommand{\baselinestretch}{1}\normalfont}
\newcommand{\halfspaced}{\renewcommand{\baselinestretch}{1.5}\normalfont}
\renewcommand{\arraystretch}{0.7}
\newcommand*\rfrac[2]{{}^{#1}\!/_{#2}}
\newcommand{\avg}[1]{\ensuremath{\langle #1 \rangle}}
\newcommand{\Ang}{\; \mathring{\text{A}}}
\newcommand{\Lya}{Ly$\alpha$ }
%\newcommand{\Ang}{\mbox{ \AA}} 
\newcommand\blfootnote[1]{%
  \begingroup
  \renewcommand\thefootnote{}\footnote{#1}%
  \addtocounter{footnote}{-1}%
  \endgroup
}
\makeatletter
\newcommand{\unchapter}[1]{%
  \begingroup
  \let\@makechapterhead\@gobble % make \@makechapterhead do nothing
  \chapter{#1}
  \endgroup
}
\makeatother

%force sub subsections to be numbered and show show up in the table of contents

\setcounter{secnumdepth}{3}
\setcounter{tocdepth}{3}
% define some shortcuts
\newcommand{\Fig}[1]{Fig.~\ref{#1}}
\newcommand{\Sec}[1]{Section~\ref{#1}}
\newcommand{\Eqn}[1]{Eq.~\ref{#1}} \newcommand{\lya}{Ly$\,\alpha$ }
\newcommand{\trec}{\ensuremath{t_{\rm rec}}}
\newcommand{\tq}{\ensuremath{t_{\rm q}}}
\newcommand{\nbar}[1]{\ensuremath{\bar{n}_{\rm #1}}}
\newcommand{\pow}[2]{\ensuremath{#1 \times 10^{#2}}}
\newcommand{\hmpc}{\ensuremath{\,h^{-1}\,{\rm Mpc}\,}}
\newcommand{\ihmpc}{\ensuremath{\,h\,{\rm Mpc^{-1}}}}
\newcommand{\K}{\mbox{ K}}
\newcommand{\bma}{\begin{math}}
\newcommand{\ema}{\end{math}}
\newcommand{\beq}{\begin{equation}}
\newcommand{\eeq}{\end{equation}}
\newcommand{\beqa}{\begin{eqnarray}}
\newcommand{\eeqa}{\end{eqnarray}}
\newcommand{\bc}{\begin{center}}
\newcommand{\ec}{\end{center}} 
\newcommand{\bit}{\begin{itemize}}
\newcommand{\eit}{\end{itemize}}
\font\BFd=cmmib10
\font\BFt=cmmib10
\font\BFs=cmmib10 scaled 700
\font\BFss=cmmib10 scaled 500
\def\bbox#1{%
\relax\ifmmode
\mathchoice
{{\hbox{\BFd #1}}}
{{\hbox{\BFt #1}}}
{{\hbox{\BFs #1}}}
{{\hbox{\BFss #1}}}
\else \mbox{#1} \fi }
\def\k{{\bbox{k}}}
\def\q{{\bbox{q}}}
\def\r{{\bbox{r}}}
\def\x{{\bbox{x}}}
\def\thetab{\pmb{\theta}}
\def\dk{\frac{d^3k}{2 \pi^3}}
\def\dq{\frac{d^3q}{2 \pi^3}}
\def\dkc{\frac{d^3k_3}{(2 \pi)^3}}
\newcommand{\MHz}{\mbox{MHz}}
%%%%%%%%%%%%%%%%%%%%%%%%%%%%%%%%%%%%%%%%%%%%%%%%%%%%%%%%%%%%

\newcommand{\tita}{{\bf OUTER SPACE AND FOURIER SPACE}:\\UNDERSTANDING FOREGROUNDS FOR NEUTRAL HYDROGEN\\EPOCH OF REIONIZATION MEASUREMENTS}
\newcommand{\titlow}{Outer space and Fourier space:\\ understanding foregrounds for neutral hydrogen Epoch of Reionization measurements}

\begin{document}
\addcontentsline{toc}{chapter}{Title}
\frontmatter
\doublespaced
\thispagestyle{empty}
\parskip=0.3in
\begin{center}
{\tita }\\

Saul Aryeh Kohn\\

A DISSERTATION\\

in\\ 

Physics and Astronomy\\

Presented to the Faculties of the University of Pennsylvania \\
in Partial Fulfillment of the Requirements for the Degree of Doctor of Philosophy\\

2018
\end{center}
\parskip=0in

\begin{multicols}{2}
\noindent Supervisor of Dissertation \\

\begin{flushright}
\noindent Graduate Group Chairperson\\
\end{flushright}

\end{multicols}

\noindent\makebox[0in][l]{\rule[2ex]{2.8in}{.3mm}} \hspace{3.05in} \makebox[0in][l]{\rule[2ex]{2.8in}{.3mm}} 
\vspace{-.5in}
\begin{multicols}{2}
\singlespaced
\noindent James E. Aguirre\\ \small Associate Professor of Physics and Astronomy\\

\normalsize

%this is the grad chair, not the chair of the committee, right?
\begin{flushright}
Joshua Klein\\ \small Professor of Physics and Astronomy
\end{flushright}
\end{multicols}

\halfspaced
\noindent Dissertation Committee:

\noindent Cullen Blake, Assistant Professor of Physics and Astronomy

\noindent Adam Lidz, Associate Professor of Physics and Astronomy

\noindent Elliot Lipeles, Associate Professor of Physics and Astronomy

\noindent Masao Sako, Associate Professor of Physics and Astronomy

\newpage

\pagestyle{plain}
\unchapter{Dedication}
%\doublespaced
\vspace*{2in}
\begin{center}
{\large\emph{to my grandparents}}
\end{center}

\newpage

%\doublespacing

\thispagestyle{empty} % No page number as per Manual, p. 11

\vspace*{\fill}

\begin{flushleft}
{\tita }

\copyright \space COPYRIGHT
 
2018

Saul Aryeh Kohn\\[24 pt] % If traditional copyright then delete everything below here, but keep \end{flushleft}

This work is licensed under the \\
Creative Commons Attribution \\
NonCommercial-ShareAlike 3.0 \\
License

To view a copy of this license, visit

\url{http://creativecommons.org/licenses/by-nc-sa/3.0/}
\end{flushleft}


%%%%%%%%%%%%%%%%%%%%%%%%%%%%%%%%%%%%%%%%%%%%%%%%%%%%%%%%%%%%

\chapter{Acknowledgments}
%\halfspaced
{\sloppy
% intro...
\noindent
While I am the one to have performed the research and compiled it, this thesis is the product of many, of which I must thank a specific few. \\

% PAPER & HERA teams
\noindent
It has been a privilege to work alongside the talented and passionate individuals that comprise the PAPER and HERA teams. To my co-conspirators of PAPER analysis: Danny Jacobs, Carina Cheng, Matt Kolopanis and Josh Kerrigan, it has been a struggle and a pleasure, and I will sorely miss working with you. A special thanks to the Penn portion to the effort: David Moore, Paul La Plante, Zac Martinot, Immanuel Washington and Tashalee Billings. I look forward to continued conversations and updates.\\

\noindent
The Department of Physics \& Astronomy of the University of Pennsylvania is a fantastic place to work. The cohorts of graduate students it brings together are consistently awesome, engaging and friendly. I have made some of my best friends here, including Ashley Baker, Christian Herwig, Elodie Resseguie, Christina Krawiec, Rachel Wolf and Alyssa Barlis. Outside of Physics \& Astronomy, but within the Penn umbrella, I must thank my dear friends Julie Westover, Irteza Binte-Farid and Steve Kocher for their loving support. Thank you as well to the School of Arts and Sciences for supporting my final year at Penn with a Dissertation Completion Fellowship.\\\\\\

\noindent
James, you are an incredible supervisor. I have learned so much from you, and treasure the opportunities you have given me to grow and become the scientist that I am. I know how rare it is to be able to call a PhD supervisor a friend, and I count my luck there daily.\\

\noindent
Matt, Debby and Naomi, thank you for your great conversations, encouraging words and hospitality. I can't want to be part of the family.\\

\noindent
To my siblings, Hannah, Yosef and Ezra: you are are always full of loving support and cutting sarcasm, neither of which I ever want to be without.\\

\noindent
Mom and Dad, this thesis belongs to you as much as it does me. Your hard work, encouragement and love made me who I am. Thank you.\\

\noindent
Gabby -- words here simply won't do justice to how grateful I am for you. You're the best, and I love you.
}
%%%%%%%%%%%%%%%%%%%%%%%%%%%%%%%%%%%%%%%%%%%%%%%%%%%%%%%%%%%%


\newpage
%\vspace*{.75 in}
\vspace*{.15 in}
\begin{center}
\addcontentsline{toc}{chapter}{Abstract}
{\bf ABSTRACT}\\
\tita \\
\parskip=0.2in


Saul A. Kohn\\
James E. Aguirre
\end{center}
\noindent
%350 Word Limit
{\sloppy
The Epoch of Reionization (EoR) was a cosmic phase transition from a neutral to an ionized state.
The first generation of luminous objects were able to heat and ionize their surrounding media, predominantly clouds of neutral hydrogen.
Detection of brightness temperature fluctuations from the redshifted hyperfine 21\,cm line of neutral hydrogen would provide a direct, inherently three-dimensional probe of astrophysics and cosmology during this transformational stage of cosmic history.

Foreground power dominates the measurements of interferometers that seek a statistical detection of the EoR. The inherent spectral smoothness of synchrotron radiation -- the dominant foreground emission mechanism -- the chromaticity of the instrument, and the spectral structure of the target signal allows these experiments to delineate a boundary between spectrally smooth and structured emission in Fourier space.
These separate components are referred to as the `foreground wedge' and the `EoR window'.
However, Faraday rotation of polarized synchrotron radiation induces frequency-dependent structure in its spectrum, which will leak power from the foreground wedge into the EoR window. This makes polarization a potential contaminant for EoR measurements. 

This work presents investigations into the impact of polarization on EoR measurements, in the Fourier space relevant to current EoR experiments. 
We show, separately, both the widest and the deepest integrations on polarized power in the EoR window to date. All results are consistent with negligible leakage, to the noise levels attained. Our deep integration also represents the best limits to date on both polarized and unpolarized power in the EoR window. Polarized redundant calibration is also described and implemented on observations for the first time in this work.

We also present new methods to expand the potential of EoR measurements, while remaining within the EoR window paradigm. Interferometric sensitivity to a monopole signal is explored, with the first ever implementations on observational data. We construct higher-order correlation functions to investigate the connection between the kinetic Sunyaev-Zeldovich effect, mapped into Fourier space, and the 21\,cm power spectrum. Finally, we examine the role deep learning could play in the analysis of more futuristic 21\,cm image cubes.
}
%Our results continually reinforce the importance of understanding instrumental polarization; how an interferometer is capable of leaking polarized signal into nominally unpolarized data. 


\vspace*{\fill}

\newpage

\singlespaced
\tableofcontents

\newpage
\phantomsection
\addcontentsline{toc}{chapter}{List of Tables}
\listoftables

\newpage
\addcontentsline{toc}{chapter}{List of Figures}
\listoffigures


\halfspaced
\setlength{\parindent}{0.25in}

%%%%%%%%%%%%%%%%%%%%%%%%%%%%%%%%%%%%%%%%%%%%%%%%%%%%%%%%%%%%


\mainmatter
\part{Radiation, Interferometry \& Polarimetry}

\vspace*{\fill} 
\begin{quote} 
\centering 
... His light is like a niche, within which is a lamp, the lamp within glass, the glass as if it were a white star... Light upon light.\\
\textit{Al Quran, Surah An Nur (24:35)}
\end{quote}
\vspace*{\fill}

\chapter{The Epoch of Reionization}
\label{chapter:eor_intro}

Shortly after the Big Bang, the Universe existed as an opaque, primordial soup of quarks, leptons, gluons and extremely energetic photons.
With density anisotropies formed by hugely inflated quantum fluctuations, ionized hydrogen, deuterium, helium, lithium and beryllium (but mostly hydrogen and helium) filled the Universe as a hot plasma. 
Black-body photons were continuously scattered throughout this plasma. All the while, the Universe adiabatically expanded, and the plasma cooled. 

About 380,000 years after the Big Bang, the number of photons with energies above the 13.6\,eV threshold required to ionize neutral hydrogen (astronomers refer to neutral hydrogen as {\sc hi}, and ionized hydrogen [i.e. protons] as {\sc hii}) became outnumbered by the number of baryons, and {\sc hii} was able to recapture electrons without immediately being ionized. 
During this critical period, known as recombination, the plasma was able to neutralize. The Universe underwent a cosmic transition from optically thick to optically thin as free electrons were captured, allowing light to travel unimpeded, in straight lines for the first time.
Fast-forwarding about 14 billion years (bear with me), some of these photons that existed at the time of `last scattering', redshifted by the expansion of the Universe, are observed today as the Cosmic Microwave Background (CMB).

We exist today in a structured, complicated and diverse Universe of stars and galaxies, dark matter and dark energy, but very little neutral gas. At the same time, observations of the CMB \citep[e.g.][]{Planck.16.1, Planck.16} and standard cosmological models (referred to under the umbrella term of $\Lambda$CDM, standing for Dark Energy \& Cold Dark Matter; e.g. \citet{Komatsu.09}) find extraordinary agreement with the story told above. However, there also exists a large observational gap: how did the Universe transition from it's neutral state at recombination, to its ionized and structured state today? How, and when, did the Universe \textit{reionize}?

The prevailing theory of the formation of cosmological structure begins with the primordial density anisotropies of the hot plasma. The dark matter that pervaded the Universe should have traced those perturbations, and gravitationally accreted into those regions, increasing the overdensities. Eventually, overdensities above some threshold density\footnote{For simple collapse models, this threshold is 18$\pi^2$ above the average density of the Universe; e.g. \citet{Press.74}.} collapsed into halos; structures supported by their own gravitational potential. The similarly pervasive {\sc hi} field should have traced the dark matter overdensities. This gravity-dominated period represents an epoch of relatively simple physics directly driven by $\Lambda$CDM. It is known colloquially as the ``Dark Ages", as at that time no luminous structures existed, and the only photons that existed were from the slowly fading CMB\footnote{For most purposes in this thesis, we neglect exotic physics such as Dark Matter annihilations that could in principle be an additional source of radiation at early times.}.

Within the early dark matter halos, enough {\sc hi} accreted to a great enough density to fuse, igniting the first stellar cores. These first stars (referred to by astronomers as Population-3 or ``Pop{\sc iii}" stars) were the first source of ultraviolet (UV) and X-ray photons capable of ionizing {\sc hi} since recombination. They were also extremely massive and short-lived, and their supernovae likely provided the seeds for the first galaxies (composed of Pop{\sc ii} stars; e.g. \citet{Ricotti.16}). The Pop{\sc iii} era is sometimes referred to as ``Cosmic Dawn", and represents the birth of astrophysics in our Universe. With the origin of galaxies, {\sc hi} surrounding haloes (the intergalactic medium; IGM) began to be reionized, forming ``bubbles" of {\sc hii}. As more luminous structures formed, UV photon production increased, and the reionization rate overcame recombination.

In our local Universe, the IGM is highly ionized. But without recombination, and therefore a neutral IGM, the CMB could not have arisen. The cosmic phase transition from neutral to ionized: a competition between cosmological physics and astrophysics; the formation of large-scale luminous structure; is known as the Epoch of Reionization (EoR). The myriad contemporary challenges associated with its detection are the subject of this thesis.

This Chapter is structured as follows. In Section~\ref{sec:eor_intro_current} I will review current evidence for the nature and timing of the EoR. In Section~\ref{sec:eor_intro_hi}, I introduce the motivation for my thesis work -- directly measuring high-redshift {\sc hi} via radio emission from the 21\,cm hyperfine transition. In Section~\ref{sec:eor_intro_future}, I point to future prospects of observational cosmology at EoR redshifts. Finally, in Section~\ref{sec:eor_intro_this_thesis}, I provide some context for the contents of the rest of the thesis.

\section{Current Measurements}
\label{sec:eor_intro_current}

The existence of Cosmic Dawn and the Dark Ages, at least as described above, have not yet been observationally confirmed. We do however have tantalizing clues of it's nature, and evidence when it ended. So far, these clues have largely come from high redshift galaxies and the CMB.

\subsection{High-redshift galaxies}

Quasars\footnote{``Quasar" is a contraction of ``quasi-stellar radio source". They were first identified in radio surveys as extremely bright point-like sources, which might have been characterized as stars when compared to optical images -- except that stars do not shine brightly at radio wavelengths.} are among the most luminous objects in Universe \citep[e.g.][]{Manti.17}. Powered by accretion of gas onto a supermassive black hole, quasars emit an abundance of UV and X-ray photons. Most notably, they are bright sources of Lyman-$\alpha$ photons (Ly$\alpha$; rest-wavelength of 121.567\,nm). Ly$\alpha$ is a spectral line from the $2p \rightarrow 1s$ transition of {\sc hi} with a high cross-section, so it will become absorbed if propagating through a neutral region of the IGM. 

\cite{Gunn.65} exploited this effect, predicting that quasars embedded in a highly neutral IGM would have all emission at wavelengths smaller than Ly$\alpha$ obscured. This was because all spectral emission at bluer wavelengths would redshift into the Ly$\alpha$ wavelength and be absorbed by the IGM. This effect, known as the Gunn-Peterson Trough, was finally detected by \cite{Becker.01}. Their investigation of the spectra of four quasars with $5.80<z<6.28$ found the optical depth towards the quasars increasing with redshift, indicating an increasing neutral fraction with redshift. \cite{Gunn.65} showed that the optical depth of Ly$\alpha$ through {\sc hi} is $\tau_{\rm GP}\sim10^4x_H$, where $x_{HI}$ is the fraction of {\sc hi} out of {\sc hi} and {\sc hii} towards the quasar. This meant that \cite{Becker.01} could claim evidence of the IGM having $x_{HI}\gtrsim10^{-4}$ in the direction of the quasar at $z\sim6$ -- and that the EoR did not end before that redshift. \cite{Fan.06.2} presented spectra of 19 $z\sim 6$ quasars, finding a sharp increase in $\tau_{\rm GP}$ with increasing redshift, suggesting evolution of $x_{HI}$ at $z>5.7$. Figure~\ref{fig:eor_intro_qso} shows the spectra from their study, featuring Gunn-Peterson troughs. The sharpness of the cut-off -- related to the effective optical depth -- increases with redshift.

 \begin{figure}
 \centering
 \includegraphics[width=0.8\textwidth]{chapters/eor_intro/figures/FanSpectra.png}
 \caption[The Ly$\alpha$ spectra of 19 high-redshift quasars.]{The Ly$\alpha$ spectra of 19 high-redshift quasars, as reported by \cite{Fan.06.2}. The effective optical depth towards each quasar increases with redshift, suggesting an evolution in the neutral fraction of the IGM. This figure was taken from \cite{Fan.06.review}.}
 \label{fig:eor_intro_qso}
 \end{figure}

Spectroscopic observations of high-redshift sources are relatively difficult to perform, and require large amounts of integration time. However, the distinctive Ly$\alpha$ limit in the spectra shown in Figure~\ref{fig:eor_intro_qso} is a distinctive property of high-redshift quasars. As an alternative, astronomers have developed a photometric method that is cheaper and easier to perform, called the ``Ly$\alpha$ dropout" technique. In this scheme, a survey will use an appropriate set of filters to probe a given redshift (imagine rectangles overlaid on the spectra in Figure~\ref{fig:eor_intro_qso}). If an excess is seen towards one galaxy in one filter, and very little signal is seen in the next-bluest filter, that galaxy may be considered a candidate high-$z$ quasar. With some modelling, the luminosity function of  Ly$\alpha$ dropout galaxies can be used to measure the mass distribution of galaxies as a function of redshift \citep[e.g.][]{Bouwens.15, Bouwens.16}. Using these luminosity functions and comparing to star formation rate models and measurements, \cite{Robertson.15} argue that current measurements are consistent with quasars playing a relatively minor role in reionization, with the bulk of ionizing photons arising form ``normal" galaxies rather than black hole accretion discs. However, this claim is moderately dependent upon uncertain model parameters, such as the escape fraction of ionizing photons from their local halo \citep{Robertson.13} and the star formation history at high redshifts. The latter is beginning to be answered as more high-redshift gamma-ray bursts are detected and characterized \citep{Wang.09, Robertson.12, Wang.15}.
%\footnote{In work unrelated to the bulk of this thesis, I presented observations of gamma-ray burst hosts suggesting that gamma-ray bursts were unbiased tracers of star formation rate -- a crucial component for using gamma-ray bursts to inform our understanding of the EoR \citep{Kohn.15}.}.

At the time of writing, photometric and spectroscopic observations of high-$z$ quasars suggest that the IGM had a neutral fraction $>10^{-4}$ at $z\sim 6$, indicating that the EoR ended around 1 Gyr after the Big Bang \citep{Barnett.17}. \cite{Fan.06.2} reported a direction dependence on their measurements, suggesting that reionization was patchy, rather than homogeneous.

\subsection{Observations of the CMB}
% CMB tau, kSZ
CMB photons will Thomson-scatter off-of free electrons in their path, suppressing the observed signal in a given direction. This suppression is described by an optical depth,

\begin{equation}
\tau_{\rm cmb} \propto \int^{z_{\rm recomb}}_0 x_i(z) a(z) \frac{{\rm d}l}{{\rm d}z}{\rm d}z,
\label{eq:eor_intro_tau_cmb}
\end{equation}
where $x_i(z)$ is the average ionized fraction at redshift $z$, $z_{\rm recomb}\approx1100$ is the redshift of recombination (and last scattering), $a(z) = (1+z)^{-1}$ and  ${\rm d}l/{\rm d}z$ is the cosmological line-element to redshift $z$. The optical depth to the CMB will suppress the observed primary anisotropies\footnote{The CMB is most-usefully described in harmonic $C_{\ell}$-space. Primary anisotropies, due to damped acoustic oscillations in the early hot plasma \citep{Silk.68}, are strongest at multipoles $\ell\lesssim3000$ -- equivalent to spatial scales $\theta \gtrsim 0.1^{\circ}$.} by a factor of $e^{-\tau_{\rm cmb}}$.

The dependence on $x_i(z)$ is crucial to our current understanding of the EoR; the largest contribution to the value of $\tau_{\rm cmb}$ will come from the large injection of electrons into the IGM during the EoR. Its measurement is not without complications: the suppression effect is degenerate with the overall amplitude of the CMB power spectrum, and it is an integrated quantity. 

The first complication can be addressed by performing polarized observations of the CMB, which breaks the degeneracy. A CMB photon that scatters off of an electron within a quadrupolar anisotropy will become linearly polarized, boosting polarized power at $\ell\sim2\sqrt{\tau_{\rm cmb}}$ \citep{Zaldarriaga.97.pol}. The WMAP satellite was the first telescope capable of making all-sky measurements of the CMB polarization, and detected an such an excess in polarized power, as shown in Figure~\ref{fig:eor_intro_spergel_tau} \citep{Kogut.03, Spergel.03}.

\begin{figure}
\centering
\includegraphics[width=0.8\textwidth]{chapters/eor_intro/figures/spergel_tau.png}
\caption[The $TE$ cross-power spectrum from WMAP.]{The Temperature-Linear Polarization ($TE$) cross-power spectrum from WMAP. The power was consistent with collisionless models save for an excess at large scales, consistent with large amounts of Thomson scattering of CMB photons during the EoR. Figure taken from \cite{Spergel.03}.}
\label{fig:eor_intro_spergel_tau}
\end{figure}

Such a measurement is able to recover a value of $tau_{\rm cmb}$, but due to its integral nature does not provide an ionization history. Values of $x_i(z)$ are therefore extremely model-dependent; typically extracted by assuming ``instantaneous reionization" in which $x_i(z)$ transitions from 0 to 1 quickly and smoothly. These models are often described using the tanh form:

\begin{equation}
x_i(z) \propto 1+\tanh\left(\frac{z - z_{\rm re}}{\Delta z}\right),
\end{equation}
where $z_{\rm re}$ is the redshift of instantaneous reionization, but is more usefully thought of as the redshift at which $x_i \approx x_{HI} \approx 0.5$. $\Delta z$ is the duration of reionization. Using such models, \cite{Planck.16.reionization} reported limits of  $7.8 < z_{\rm re} < 8.8$ and $\Delta z < 2.8$. They also reported that all of their modelling suggested that $x_i(z\gtrsim10)<0.1$.

The CMB provides us with more than one probe of reionization. \cite{Zeldovich.69} and \cite{Sunyaev.70} showed that CMB photons that scattered off of free electrons would introduce secondary anisotropies to the CMB (i.e. additional features in the $C_{\ell}$ spectrum). The Sunyaev-Zeldovich (SZ) effect is divided in to two categories: the thermal effect (tSZ) and the kinetic effect (kSZ). The tSZ occurs when CMB photons scatter off of electrons with high energies ($\sim 10$\,keV) due to their temperature; such high temperatures can only occur in high-mass galaxy clusters, and hence at redshifts lower than the EoR. 

The kSZ is of relevance to EoR studies. CMB photons scattering off of clouds of electrons in coherent velocity flows can be Doppler-shifted, which will redshift or blueshift (depending on the sign of the line-of-sight velocity) spectral observations of the CMB. Of course, clouds of electrons exist in relatively low-redshift galaxy clusters, and it was their contribution to the kSZ that was first detected \citep{Hand.12}. However, detection of the EoR contribution to the signal would be a powerful probe of dynamics on cosmological scales in the early Universe. The distortion induced by the kSZ on the CMB is given by

\begin{equation}
\frac{\delta T}{T_{\rm CMB}}(\hat{s}) = \frac{\sigma_T}{c} \int_0^{z_{\rm recomb}} n_e(z)e^{-\tau(z)} \hat{s}\cdot\vec{q} \frac{{\rm d}s}{{\rm d}z}{\rm d}z,
\end{equation}
where $\sigma_T$ is the Thomson Cross Section, $n_e(z)$ is the average number density of electrons at redshift $z$, $\tau(z)$ is the optical depth to redshift $z$, and d$s$/d$z$ is the cosmological line element to redshift $z$ along direction $\hat{s}$. We can model inhomogeneities in the ionization and baryon fraction as vector field
\begin{equation}
\vec{q}(\hat{s}) = (1+\delta_x)(1+\delta_b)\vec{v}(\hat{s}),
\end{equation}
where $1+\delta_x = x_i/\left\langle x_i \right\rangle$, $1+\delta_b= \rho_b/\left\langle \rho_b \right\rangle$ and $\vec{v}$ is the free electron bulk flow.

This is an optical depth effect much like the optical depth of the CMB in Equation~\ref{eq:eor_intro_tau_cmb}. However, it has important distinctive features: it is a direction dependent, probing the velocity of electron bulk flows along the line of sight, rather than the average ionization history, and it can create excesses and decrements in power, depending on the sign of the velocity component.

Detection of the kSZ is difficult, as the secondary anisotropies it causes in brightness temperature are a few percent of tSZ. Compared to the primary anisotropies, the tSZ is roughly 10 times weaker in $\mathcal{D}_{\ell}=(\ell+1)\ell C_{\ell}/2\pi^2$, and the kSZ is roughly 100 times weaker.
\cite{George.15} reported a detection of the kSZ at $\ell\approx3000$ (where the primary anisotropies and the Cosmic Infrared Background, a CMB contaminant at small scales, are both close to their minimum values), using a suite of cosmological and spectral models to remove CMB foregrounds, the primary anisotropies and the tSZ \citep[e.g.][]{Shaw.10}. Again, measurements from the CMB are integrated quantities, so their constraints on reionization are model dependent. Using a symmetric model of reionization history \citep{Zahn.12}, they set the constraint that $\Delta z < 5.4$, with their likelihood peaking around $\Delta z = 1.3$. Using similar techniques with shallower data from the same telescope, \cite{Zahn.12} set consistent limits that $\Delta z \geq 2$, and also that reionization ended at $z>5.8$.

Wide-field measurements of the kSZ would be rewarding, but the foregrounds of the Galaxy, CIB, CMB and tSZ represent contemporary challenges for direct-imaging experiments. However, one could hope to use correlations of the velocity field the kSZ sources, and the density field associated with it, to decorrelate the foregrounds and estimate the underlying fields \citep[e.g.][]{Cooray.04, Alvarez.16}. The mathematical formalism for such a technique is laid-out in Chapter~\ref{chapter:ksz_21cm}.

\section{Direct measurements of {\sc hi}}
\label{sec:eor_intro_hi}

To borrow a line from \cite{DannyThesis}, the Dark Ages were not dark. Our Universe is dynamic and energetic, and particles constantly seek a lower energy state.

Throughout the Dark Ages and the EoR, {\sc hi} was shining, weakly, at radio wavelengths. The {\sc hi} atom is capable of a hyperfine transition between the ${}_{1}s_{1}$ and ${}_{1}s_{0}$. The electron spin can spontaneously change from parallel to the proton spin to antiparallel, emitting a photon of frequency $\nu_{21\,cm}\approx1420.4$\,MHz; a rest-wavelength of roughly 21\,cm. The emission coefficient of this transition is $\sim 2.9\times10^{-15}\,s^{-1}$, corresponding to a half-life of approximately 11 million years per atom. This is both the strength and the weakness of observing {\sc hi}. According to Fermi's Golden Rule, the probability of emission is equal to that of absorption; a small transition rate leads to a relatively dim signal, but 21\,cm photons are extremely unlikely to be observed between their emission and observation. This allows the 21\,cm signal to be tomographically mapped -- a sharp excess at the relevant frequency indicates a redshift -- and constructed into a 3D cube in space and time.

Observations of {\sc hi} during the Dark Ages and the EoR would directly constrain the ionization history of the Universe, allowing us to probe the evolution of the density field at high redshifts, the properties of the first stars, galaxies and black holes, and map the structure of the IGM as it evolves as a function of redshift. It would also represent the furthest baryons ever detected, and the largest volume of the Universe ever surveyed. This Section reviews the nature of the 21\,cm line during the EoR, and two parallel endeavours to detect it, using the monopolar signal (i.e. power averaged over the sky), and the anisotropic signal (power as a function of spatial scale).

\subsection{The 21cm transition}

As noted above, the probability of a single {\sc hi} atom undergoing the 21\,cm transition is exceedingly small. However, during the Dark Ages and the EoR, {\sc hi} was ubiquitous, and such transitions would be occurring constantly.
The relative occupancy of the hyperfine state in {\sc hi} is given by

\begin{equation}
\frac{N_1}{N_0} = \frac{g_1}{g_0} e^{-h\nu_{\rm 21cm}/k_B T_S},
\end{equation}
where $g_1$ and $g_0$ refer to the triplet and singlet states of the atom and are equal to 3 and 1, respectively. This Equation defines the spin temperature $T_S$, an analog of molecular excitation temperature for {\sc hi}. By solving the radiative transfer equation for this system in the Rayleigh-Jeans limit \citep[e.g.][]{MooreThesis}, we can write the brightness temperature of the emission as:

\begin{equation}
T(z) = T_S \tau_{\nu}(z) + T_{\rm bg}(1-\tau_nu(z)),
\end{equation}
where the optical depth of {\sc hi}, $\tau_{\nu}$, is assumed to be small, and the temperature of the background radiation $T_{\rm bg}$ is, for EoR observations, the temperature of the CMB at redshift $z$. \cite{Furlanetto.06} show that the optical depth can be expressed in terms of cosmologically-interesting quantities, assuming the IGM is diffuse:

\begin{equation}
\tau_{\nu} \approx 9.2\times10^{-3} (1+\delta_b)(1+z)^{3/2}\frac{x_{HI}}{T_S}\frac{H(z)}{1+z}\frac{{\rm d}v_{\parallel}}{{\rm d}r_{\parallel}},
\end{equation}
where d$v_{\parallel}$/d$r_{\parallel}$ is the component of the peculiar velocity of the {\sc hi} cloud per unit depth along the line-of-sight, with respect to the Hubble expansion given by $H(z)/1+z$. The prefactor of $9.2\times10^{-3}$ justifies the assumption of small $\tau_{\nu}$ made above. Comparison to the optical depth for the Ly$\alpha$ line in Section~\ref{sec:eor_intro_current} emphasizes that {\sc hi} observations refer an opposite regime of EoR probes.

Combining the above equations, we can define the brightness contrast between redshifted {\sc hi} emission and the CMB,

\begin{equation}
\delta T \approx 9{\rm mK} \times x_{HI} (1+\delta_b)(1+z)^{1/2} \left( 1 - \frac{T_{\rm CMB}(z)}{T_S} \right)
\left( \frac{H(z)/(1+z)}{{\rm d}v_{\parallel}/{\rm d}r_{\parallel}} \right).
\end{equation}




% spin temperaure
% WF coupling
% unpolarized
% spectral structure?

% foregrounds at the redshifted low frequency ~ 10^5 K, signal at 10^-3 K

% why is it interesting (near term) - Liu Tau constraints, Kittiswitt maps (topology of EoR = cool), Kern emulators -- sensitivity to astrophysics & cosmology

%\subsection{Foreground Challenges}
% foreground sychtrotron and thermal noise

\subsection{Global spectrum}
% spectrum cartoon

%The 21 cm signal from the Epoch of Reionization (EoR) contains a wealth of information about the Universe at these early times, and can provide key insight inaccessible to other observational techniques \citep{Loeb.12}. 

In particular, the sky-averaged 21 cm signal, the so-called global signal, directly
contains key information about the thermal history of the intergalactic medium
(IGM) as a function of redshift \citep{Pritchard.10}. Historically,
observational efforts to detect the global signal have been ``single dish''
experiments, where a single receiving element is characterized to high
precision, and then operated in an effort to detect the EoR signal as a function
of frequency (and thus redshift). 

\subsubsection{Limits on the EoR} % change this title...

Experiments such as the Experiment to Detect
the Global EoR Step (EDGES, \citealt{Bowman.10}), Sonda Cosmol\'{o}gica
de las Islas para la Detecc\'{i}on de Hidr\'{o}geno Neutro (SCI-HI,
\citealt{Voytek.14}), and the Cosmic Twilight Polarimeter (CPT,
\citealt{Nhan.16}), have been proposed or constructed seeking to measure
the global signal from the Dark Ages or the EoR, through a deep understanding of
the properties of the instruments. In all cases, the instruments consist of a
single-element, and much of the observational effort contributes toward a
thorough understanding of systematic uncertainties of the instrument. 

%The signal from the Dark Ages and EoR is thought to be 4--5 orders of magnitude fainter than nearby bright foregrounds, such as galactic synchrotron radiation \citep{McQuinn.07}. As such, an exquisite understanding of the correlated noise in these instruments is of the utmost importance. Largely, these experiments have not yet detected a feature in frequency-space that can clearly be interpreted as a detection of either the Dark Ages or EoR global signal. To date, EDGES has provided a lower-limit on the duration of reionization of $\Delta z \geq 0.06$ \citep{Bowman.10}.

\subsection{Anisotropic signal}

\subsubsection{Current Limits}
% galaxies => z_end -- ended z < 6 and z > 5.8 (kSZ limits)
% CMB tau => z_middle -- middle 7.8 < z < 8.8
% CMB kSZ + tau => dz < 2.8
% global limits => dz 
% 

\section{Future probes}
\label{sec:eor_intro_future}
% 
% HI, CO, C+ intensity mapping, extreme deep fields from JWST [HIGH REDSHIFT SPECTROSCOPY? e.g. Salvaterra+ 2011]
% 


%
% what is the EoR -- history (global signal history)
% why is it interesting (near term) - Liu Tau constraints, Kittiswitt maps, Kern emulators -- sensitivity to astrophysics & cosmology
% "why do we think reionization ended at z=6?"
% --> hold cosmo params fixed, then we can constrain astro ones. If you want to constrain cosmology, we have some, weak, sensitivity (we have a complicated proxy for the density field)

% current (CMB tau, high-z galaxies (CANDELS; lyman alpha dropouts; quasars), Gunn Peterson, AHEM AHEM kSZ), EoR power spectrum, and future (HI, CO, C+ intensity mapping, extreme deep fields from JWST [HIGH REDSHIFT SPECTROSCOPY? e.g. Salvaterra+ 2011]) probes
% --> check out Mesinger paper trail and Robertson+'15
% current state of the art -- power spectra -- (limits from PAPER, LOFAR, MWA).


% why are we still looking => foregrounds -> relative signal scales -> more next chapter
% in the following chapters I will speak about...
%


% spectrum
% global signal - EDGES




% in the following chapters I will speak about...


\section{This thesis}
\label{sec:eor_intro_this_thesis}
Everything in this work -- algorithmic development, mathematical theory, observations -- was carried-out in order to facilitate the detection of the EoR. While these efforts took many forms, they shared that singular motivation of moving the field forward towards a detection of {\sc hi} at cosmological distances. 

This thesis is divided into three parts. Part {\sc i} is devoted to introducing concepts used throughout this work and building a mathematical formalism around those concepts. 
Chapter~\ref{chapter:astro_rad} reviews astrophysical mechanisms for producing polarized and unpolarized radiation at low radio frequencies. 
Chapter~\ref{chapter:interferometry} builds a formalism around measuring low frequency radio waves with interferometers (and the challenges associated with accurately measuring polarized radiation), and Chapter~\ref{chapter:instruments} introduces the instruments used throughout this work.

In Part {\sc ii} I present the bulk of my efforts: building an understanding of the imprint of the polarized sky, and the instrument itself, in the Fourier space used to set limits on the EoR power spectrum. 
Chapter~\ref{chapter:eor_window_theory} reviews the current theory and major results of mapping low frequency interferometric measurements into Fourier space. 
Chapter~\ref{chapter:data_prep_and_proc} details several required quality assurance and compression steps that must be taken to clean and interact with the data. Building from clean data, Chapter~\ref{chapter:polcal} presents new algorithms developed to calibrate the measurements.
Chapter~\ref{chapter:ionosphere} discusses the impact of Earth's ionosphere on our measurements.
In Chapters~\ref{chapter:eor_window_paper32img}, \ref{chapter:eor_window_HERA} and \ref{chapter:eor_window_psa128} I present successively-deeper integrations on polarized foregrounds in successively-narrower regions of Fourier space.

Part {\sc iii} explores other uses of EoR measurements, beyond detection of the power spectrum. In Chapter~\ref{chapter:TAV}, I discuss the potential of using long time-averages of interferometric measurements to measure some component of the monopole moment of the sky. In Chapter~\ref{chapter:ksz_21cm}, I present a new formalism for cross-correlating 21\,cm emission and CMB anisotropies in Fourier space. Chapter~\ref{chapter:hera_ml} describes my initial investigations into utilizing deep learning techniques for recovering cosmological parameters from simulated EoR measurements. I conclude in Chapter~\ref{chapter:conc}.
 % The Epoch of Reionization
%
%
%
\chapter{Astrophysical Radiation}
\label{chapter:astro_rad} % Astrophysical Radiation
%
%
%
\chapter{Interferometry, Calibration \& Polarimetry}
\label{chapter:interferometry}

In this Chapter I wished to build a formalism around wide-field, polarized interferometric measurements that could be used throughout this work. Many traditional assumptions used in radio interferometry are broken in the case of the wide-field, fully-polarized, drift-scanning measurements native to interferometric EoR observations. In Section~\ref{sec:interferometry_vis}, I derive the equation describing the fundamental observable for an interferometer, called a ``visibility". Section~\ref{sec:interferometry_cal}, I describe calibration techniques relevant to this work and in Section~\ref{sec:interferometry_pol} I review some of the implications of the previous two sections for polarized measurements.

For a comprehensive review of interferometry from a more traditional perspective, see \cite{TMS}.

\section{The Visibility Equation}
\label{sec:interferometry_vis}

A radio interferometer (a term used interchangeably with ``interferometric array" for radio observations) is an ensemble of receiving elements, where each element's measurement is correlated with every other element's. The simplest case is a two-element interferometer, which we will focus on below. We assume (for now) that the elements are coplanar and identical.

\subsection{The Classical Visibility Equation}

Consider two receiving elements $i$ and $j$, separated by baseline vector $\vec{b}$. Suppose a plane wave of wavelength $\lambda$ is incident upon these elements, with direction of propagation $-\hat{s}$. The geometry of this interferometer is illustrated in Figure~\ref{fig:interferometry_2element}.

\begin{figure}
\centering
\includegraphics[width=0.9\textwidth]{chapters/interferometry/figures/visibility_explanation.pdf}
\caption{The geometry of a two-element interferometer, with a plane wave incident from direction $\hat{s}$.}
\label{fig:interferometry_2element}
\end{figure}

We can define the electromagnetic wave to have a frequency dependent phase, such that the electric field measured by element $i$ at time $t$ is

\begin{equation}
E_i = E_0 e^{-2\pi i \nu t}.
\end{equation}

The time difference between the arrival at $i$ and $j$ is called the ``geometrical delay", $\tau_g$:

\begin{equation}
\tau_g = \frac{\vec{b}\cdot\hat{s}}{c}
\end{equation}

and the electric field measured by element $j$ is

\begin{equation}
E_j = E_0 e^{-2\pi i \nu (t+\tau_g)}
\end{equation}

An interferometer is an instrument which correlates these electric fields together, integrating their product over some coherent time-scale. This correlation grants:

\begin{equation}
\langle E_i E_j^* \rangle 
= \lim_{T\rightarrow\inf}\frac{1}{2T}\int^T_{-T} E_i(t) E_j(t) {\rm d}t
= | E_0 |^2 e^{-2\pi i \nu \tau_g}
\end{equation}

where $e^{-2\pi i \nu \tau_g} = e^{-2\pi i \nu \vec{b}\cdot\hat{s}/c}$ is known as the ``fringe" term, due to its sinusoidal nature. We can generalize this relationship to include more than a single plane wave from direction $\hat{s}$. Many plane waves, from all directions, can be incident upon the interferometer at a given time and frequency. We can represent the power distribution on the sky as $S(\Omega)$, where $S(\Omega)$. However, no instrument is equally sensitive to radiation from every direction $\hat{s} \in \Omega$. Instead, an instrument has some sensitivity pattern -- a \textit{beam pattern} -- that tapers the power distribution on the sky into an ``observed sky",  $S'(\Omega) = A(\Omega)S(\Omega)$. 

These generalizations lead to the classical visibility equation:

\begin{equation}
V_{ij}(\nu) = \int A(\Omega, \nu) S(\Omega, \nu) e^{-2\pi i \nu \vec{b}\cdot\hat{s}/c} {\rm d}\Omega
\label{eq:classical_visibility}
\end{equation}

If we choose to represent the source direction in terms of directional cosines $\ell$ and $m$, and represent the baseline vector in units of wavelengths, $\vec{b}/\lambda=(u,v,w)$, we can perform a change of variables in Equation~\ref{eq:classical_visibility} to give

\begin{equation}
V_{ij}(u,v) = \int\int A(\ell, m)S(\ell, m) e^{-2\pi i (u\ell + vm + w\sqrt{1 - \ell^2 - m^2})} {\rm d}\ell {\rm d}m
\end{equation}

This relationship is often simplified by assuming only a small area of the sky is under observation -- that is, that $A(\ell,m)$ falls-off steeply from zenith -- and therefore $\ell^2$ and $m^2$ are small. This grants

\begin{equation}
V_{ij}(u,v) \approx e^{-2\pi i w} \int\int A(\ell, m)S(\ell, m) e^{-2\pi i (u\ell + vm)} {\rm d}\ell {\rm d}m
\end{equation}

which plainly casts $V(u,v)$ as the Fourier transform of the observed sky if $w$ is small. Modern low frequency interferometers used in this work greatly violate this approximation, the consequences of which I will discuss in the proceeding sections.

% fourier relationship, uvws, image plane, touch on deconvolution
% xx,xy,yx,yy <-> add in feed polarization indicies

\subsection{The Measurement Equation}

% building the [classical] visibility equation in the unpolarized case
% pointing vs drift-scanning
% wide field effects
% rebuilding the visibility equation for widefield, polarized instruments
% Stokes visibilities -- previous chapter defines Stokes Parameters -- discuss pseudo-ness

\section{Calibration Techniques}
\label{sec:interferometry_cal}

% basics of calibration for drift-scanning interferometers
% redundant calibration theory (more in polcal chapter)
% redundant vs imaging configurations
% CLEAN: Hogbom, 1D-CLEAN [NEEDS DELAY] (Parsons & Backer), linCLEAN

\section{Instrumental Polarization}
\label{sec:interferometry_pol}

%%% Instrumental polarization:
% explore the matrix-formalized visibility equation
% DI leakage (calibration errors)
% DD leakage (cannot calibrate away -- must model)
%
%
%
\chapter{Instruments}
\label{chapter:instruments}

In the following chapters I present data and results from a variety of configurations of two massively redundant low frequency interferometers, PAPER and HERA. In this Chapter I describe these instruments (Section~\ref{sec:used_in_this_work}), along with other current and future low frequency interferometers contributing to EoR science (Section~\ref{sec:not_used_in_this_work}).

\section{Instruments used in this work}
\label{sec:used_in_this_work}

The vision of Hydrogen Epoch of Reionization Arrays was first laid out in the \cite{HERAWhitePaper} White Paper. That work proposed three consecutive efforts, improving upon their predecessors, to construct low frequency interferometers capable of detecting the EoR. While the physical feeds and elements of low frequency interferometers were relatively simple to construct, signal processing, calibration and imaging required new hardware and software to be invented. A research community of observational cosmologists interested in cosmological {\sc hi} had to be nurtured.

The first of the three stages of Reionization Arrays was a parallel effort. The Precision Array for Probing the Epoch of Reionization (PAPER; Section~\ref{subsec:paper_instrument}) and the Murchinson Widefield Array (MWA; Section~\ref{subsec:mwa_instrument}) investigated separate approaches to\footnote{Among other things, see Section III B of \citet{HERAWhitePaper} for an enumerated list.} antenna design, array layout and calibration techniques, with the objective of setting upper limits on and perhaps detecting the power spectrum of the EoR.

The second stage of the Reionization Arrays brought together the teams from the first stage to design and construct a new interferometer based on the lessons learned from PAPER and the MWA. This new instrument, named \textit{the} Hydrogen Epoch of Reionization Array (HERA; Section~\ref{subsec:hera_instrument}) is currently under construction with a build-out schedule that brings new antennas online as they are commissioned. HERA's objective is not only the detection of the EoR power spectrum, but its characterization at very high signal-to-noise. Attempts at low-fidelity imaging of ionized bubbles will be made.

The nature of the third stage is, at the time of writing, somewhat undetermined and contingent on the next decade of funding for low frequency radio astronomy ({\color{red}CITE CITE CITE}). In the vision of \cite{HERAWhitePaper}, its objective will be to image structure evolution throughout the EoR.

\subsection{The Donald C. Backer Precision Array for Probing the Epoch of Reionization (PAPER)}
\label{subsec:paper_instrument}

Much of this thesis presents data from PAPER. PAPER was planned, like HERA, as a staged build-out to larger and larger arrays. In each build-out the correlator was replaced, and (nominally) identical antennae and signal chains were added to the existing array. The first iterations of PAPER consisted of 8 dipole antennae in Green Bank, West Virginia and 4 in Western Australia \citep{Parsons.10}. While the Australian site had a far better observing environment in terms of human-generated radio interference, the site in the USA was easier for the team to design and test on ({\color{red} check with James}). The array in Green Bank was build-up to 32 antennae \citep{Pober.12} and reconfigured from a more traditional ``imaging" configuration to a redundant grid \citep{Parsons.12b}, in order to experiment with redundant calibration and increased sensitivity to discrete Fourier modes. 

While the PAPER-32 array was under operation in Green Bank it became clear that the radio frequency interference (RFI) environment was bad enough to inhibit the science goals of the experiment. ({\color{red} check with James about history of move to SA}).

\subsubsection{The PAPER signal chain}

\subsubsection{PAPER-32}

The PAPER-32 array in South Africa used a highly redundant configuration in order to take the measurements resulting in, at the time, the strongest upper limits on the EoR power spectrum \citep[][sections of \citet{Moore.17} are presented in Chapter~\ref{chapter:ionosphere}]{Parsons.14, Jacobs.15, Moore.17}. The array was in redundant configuration from December 2011 to February 2012. For three nights in September 2011, the 32 elements were reconfigured into an polarized imaging configuration. The results from this deployment were used to make the first 2D power spectra of polarization, presented in \cite{Kohn.16} and in Chapter~\ref{chapter:eor_window_paper32img}. For images and a brief description, see Figure~\ref{fig:instruments_psa32layout}.

\begin{figure}
\centering
\includegraphics[width=0.9\textwidth]{chapters/instruments/figures/psa32_layouts.pdf}
\caption[The array layouts of the PAPER-32 element deployment in South Africa.]{The array layouts of the PAPER-32 element deployment in South Africa. \textit{Left}: The redundant grid. Four rows, with each element 30\,m from the next in the East-West direction, and closely-packed ($\sim$4\,m) in the North-South direction. Figure taken from \cite{Parsons.14}. \textit{Right}: The polarized imaging array. Elements were arranged in a pseudo-random scatter.}
\label{fig:instruments_psa32layout}
\end{figure}

\subsubsection{PAPER-64}

During the PAPER-32 EoR integration, there were actually 64 antennas present in the Karoo, South Africa. However, the correlator at that time could only process 64 voltage streams -- enough for 32 dual-polarization antennae. This is why 64 element single-instrument-polarization imaging results were published prior to any PAPER-32 studies \citep{Jacobs.13, Stefan.13}. EoR integrations in the 64 element redundant configuration were only possible after a new correlator was produced in 2012. Results from this array were published in \cite[][{\color{red}Cheng et al. \textit{in prep.}, Kolopanis et al. \textit{in prep.}}]{Ali.15, Pober.15}. Diagnostic results from the PAPER-64 redundant configuration that informed studies of time-averaged visibilities are presented in Chapter~\ref{chapter:TAV}. The array layouts are shown in Figure~\ref{fig:instruments_psa64layout}.

\begin{figure}
\centering
\includegraphics[scale=0.9\textwidth]{chapters/instruments/figures/psa64_layouts.pdf}
\caption[The array layouts of the PAPER-64 deployment.]{The array layouts of the PAPER-64 deployment. \textit{Left}: The redundant grid, built-out from the PAPER-32 grid. Figure taken from \cite{Ali.15}, which highlights the baseline-types used for power spectrum measurements. \textit{Right}: the imaging array, used by \cite{Jacobs.13} to set an absolute flux scale for PAPER experiments. Figure taken from \cite{Jacobs.13}.}
\label{fig:instruments_psa64layout}
\end{figure}

\subsubsection{PAPER-128}

The culmination of the PAPER experiment was the 128 element deployment. There were two observing seasons recorded: November 2013 to March 2014, and July 2014 to January 2015. In this configuration, 112 antennas were laid-out in a redundant grid with 15\,m East-West spacings and 4\,m North-South spacings. The remaining 16 antennas were arranged in `out-rigger' and `in-rigger' positions to increase \textit{uv}-coverage and enable some level of imaging. Results from the first observing season of this array are presented in Chapters~\ref{chapter:data_prep_and_proc}, \ref{chapter:polcal} and \ref{chapter:psa128}. The array layout is shown in Figure

\begin{figure}
\centering
\includegraphics[width=0.9\textwidth]{antlayout_psa128.png}
\caption[The PAPER-128 array layout.]{The PAPER-128 array layout. 112 antennas were laid-out in a redundant grid with 15\,m East-West spacings and 4\,m North-South spacings, and the remaining 16 antennas were arranged in `out-rigger' and `in-rigger' positions to increase \textit{uv}-coverage.}
\end{figure}

\subsection{The Hydrogen Epoch of Reionization Array (HERA)}
\label{subsec:hera_instrument}



\subsubsection{HERA-19 Commissioning Array}
\subsubsection{HERA-47}
\subsubsection{Future HERA Build-Outs}

\section{Other current and future low-frequency interferometers}
\label{sec:not_used_in_this_work}

\subsection{The Low Frequency Array (LOFAR)}
\subsection{The Murchinson Widefield Array (MWA)}
\label{subsec:mwa_instrument}

\subsection{Square Kilometer Array -- Low band (SKA-Low)}
\label{subsec:skalow_instrument}


%
% change rcvr -> hut to 50 Ohm
%
\part{Structure in Fourier space}

\vspace*{\fill} 
\begin{quote} 
\centering 
The Gemara asks: Do we not see that comets pass Orion? The Gemara rejects this: The aura of the comet passes Orion and it appears as though the comet itself passes. Rav Huna, son of Rav Yehoshua, said a different answer: It is merely that veil, one of the firmaments, rips and furls and the light of the next firmament is seen, and this appears like a comet. Rav Ashi said another explanation: It is not a comet that passes Orion, but a star that is uprooted from one side of Orion, and another star, from the other side of Orion, sees it and gets startled and shudders, and appears as if it is passing.\\

\textit{Talmud Berakhot (58b:11)}
\end{quote}
\vspace*{\fill}

\chapter{Peering through the EoR Window}
\label{chapter:eor_window_theory}

Chapter~\ref{chapter:eor_intro} argued the promise of direct observations of {\sc hi} during the EoR. A direct detection has not yet been made, largely due to the overwhelming power of foregrounds compared to the target signal. As shown in Chapter~\ref{chapter:astro_rad}, foreground radiation is a factor of roughly $10^4 -- 10^5$ times brighter than predicted 21\,cm anisotropies .  In the past decade, however, enormous improvements have been made in understanding how to decontaminate interferometric visibilities, and excavate the target signal. The leverage astronomers have to use is the exceptional smoothness of low-frequency synchrotron radiation, the dominant foreground emission mechanism. In this Chapter, I review the `Foreground Wedge \& EoR Window' paradigm used to delineate foreground power from noise and {\sc hi} emission. In Section

\section{The Wedge and the Window}  % ugh I don't like this section title
\label{sec:wedge_and_window}



%(Datta et al. 2010; Parsons et al. 2012b; Vedantham et al. 2012; Morales et al. 2012; Hazelton et al. 2013; Pober et al. 2013; Thyagarajan et al. 2013; Liu et al. 2014a,b).

% foreground wedge and EoR window -- Nithya predictions -- importance of beam



\subsection{Foreground avoidance}
% Foreground avoidence: delay spectrum; basic PAPER and HERA results (including Kohn et al. 2016, 2018)
\subsubsection{Power spectra}
% Define power spectra in the delay paradigm
\subsubsection{Noise spectra}
% define noise spectrum

\subsection{Foreground subtraction}
% Foreground subtraction: theory and LOFAR results

\subsection{Hybrid methods}
% Hybrid approach: MWA

\section{The Problem of Polarization}
% polarization in wedge space -- my nice freq vs fourier space diagram? -- basically everything from Moore 2013


In this Part, I have recorded my efforts to reduce interferometric observations from PAPER and HERA, taking as much care as possible to avoid introducing spectral structure to the visibilities. My work has centered around the problem of polarization in the EoR window paradigm. In the following Chapters, I discuss quality assurance metrics and data compression (Chapter~\ref{chapter:data_prep_and_proc}), polarized calibration (Chapter~\ref{chapter:polcal}), the effect of the ionosphere on polarized power spectra (Chapter~\ref{chapter:ionosphere}), and three successively deeper integrations on polarized power, concentrating on successively thinner slices of $k_{\perp}$ (Chapters~\ref{chapter:eor_window_paper32img} -- \ref{chapter:eor_window_psa128}).
%
%
%
\chapter{Data Preparation and Processing}
\label{chapter:data_prep_and_proc}
The data volume of interferometric measurements inherently scale as the square of the number of antennas in the array ($N_{\rm ant}$). Not only does the sheer volume of data from large-$N_{\rm ant}$ arrays pose a problem for data storage, but also it requires precise and efficient efforts to quality assure (QA) the data. 

In this chapter, I will outline some of the efforts involved in data preparation, preprocessing and QA that are required for an EoR power spectrum estimate.

\section{Data Compression}
\label{sec:data_compression}

The PAPER-128 correlator produced 288 {\sc miriad} files per night. Each of these contained 8126 baselines, and each baseline contained visibilities over 1024 $98$\,kHz frequency channels and 56 $10$\,s time integrations. The four instrumental polarizations were in separate files. In sum, each file was 4.2 GB which meant that each night 1.2 TB of data were recorded.

In order to efficiently transport the data over Gigabit Ethernet from the Karoo Radio Quiet Zone (KRQZ) to Cape Town, and from Cape Town under transatlantic cables to Philadelphia, some compression was required. It was also required that such a compression, while lossy, did not effect the targeted cosmological signal.

\subsection{Delay--Delay-Rate Filtering}

The compression algorithm implemented for PAPER observations, Delay--Delay-Rate (DDR) filtering, was introduced in \cite{ParsonsBacker.09} described in \cite{Parsons.14}, and we briefly review it below.

The geometric delay of a celestial signal, originating form direction $\hat{s}$, incident on an interferometric baseline described by vector $\vec{b}$, is

\begin{equation}
\tau_g = |\vec{b} \cdot \hat{s}|/c
\end{equation}

where $c$ is the speed of light. This relationship implies that $\tau_g$ is bounded for a given baseline

\begin{equation}
- |\vec{b}|/c \leqslant \tau_g \leqslant |\vec{b}|/c
\label{eq:geometric_delay_bound}
\end{equation}

Equation~\ref{eq:geometric_delay_bound} therefore gives the maximum value of $|\tau_g|$ physically meaningful for a given array -- the maximum baseline length in that array, divided by $c$. For PAPER, the maximum baseline length is 300\,m, corresponding to $\max(|\tau_g|) = 1\mu$s. 
As reviewed in Chapter~\ref{chapter:eor_window_theory}, the delay axis may be accessed by Fourier transforming a visibility along the frequency axis. Once in delay space, power at delays larger in magnitude than $1 \mu$s could be removed. With a sufficiently large frequency bandwidth, this would not produce aliased signal, according to the critical Nyquist rate. By using the $1 \mu$s as a delay bound for all visibilities, the frequency axes of all compressed visibilities remained the same (reduced in number from 1024 to 203), which while sub-optimal from a compression point of view, allowed for ease of programming at later stages.

A similar geometric bound can be obtained by Fourier transforming the time axis of visibilities, provided that they were obtained in drift-scan mode (see Chapter~\ref{chapter:interferometry}). \cite{ParsonsBacker.09} showed that the rate at which the geometric delay on an interferometric baseline changes is governed only by the position of the array on Earth, and the Earth's rotation:

\begin{equation}
\dot{\tau}_g = -\frac{\omega_{\Earth} \cos\delta}{c} \left( b_x\sin\alpha + b_y\cos\alpha\right)
\end{equation}

where $\omega_{\Earth}$ is the angular frequency of the Earth's rotation, $\alpha$ and $\delta$ are the hour-angle and declination of a point on the celestial sphere, respectively, and $\vec{b}=(b_x,b_y,b_z)$ is the baseline vector expressed in equatorial coordinates.

For arrays not close to the geographic poles, $|b_y| \gg |b_x|$, there is a maximum rate of change (corresponding to ($\alpha$, $\delta$) = (0, 0)), producing a bound on $\dot{\tau}_g$:

\begin{equation}
- \omega_{\Earth}|b_y|/c \leqslant \dot{\tau}_g\leqslant \omega_{\Earth} |b_y|/c
\label{eq:geometric_delay_rate_bound}
\end{equation}

for a 300\,m East-West baseline, the maximum delay-rate is approximately $\max(|\dot{\tau}_g|) = 0.07$\,ns\,s$^{-1}$. This delay-rate was not Nyquist sampled by a single PAPER file: requiring the previous and next files generated for that polarization to be appended on either side of each visibility's time axis to prevent aliasing from the decimation. For the large scale processing of months of data, this required a software pipeline described in Section~\ref{subsec:compression_software}.

There are also other issues with DDR compression, largely associated with instrument systematics. Delay transforms rely on the fact that the bright foregrounds that dominate the measured signal are spectrally smooth, and that the frequency response of the instrument is also spectrally smooth: this of course is the basis for the EoR window paradigm reviewed in Chapter~\ref{chapter:eor_window_theory}. Likewise, delay-rate filtering assumes temporal smoothness. Radio Frequency Interference (RFI) signals created by human communications violate both models of smoothness, since they are typically confined to narrow bandwidths (creating sharp spikes along the frequency axis) and may be transient (creating sharp spikes along the time axis). This requires steadfast identification and flagging algorithms for RFI (see Section~\ref{sec:RFI}), and some variety of interpolation, fitting, or CLEANing across the flagged regions prior to compression.

By DDR filtering of PAPER-128 data using a 300\,m baseline to set the width of the filters we were able to reduce the volume of the data by an approximate factor of 70.

\subsection{Software Implementation}
\label{subsec:compression_software}

The first season of PAPER-128 data, due to a variety of circumstances, required compression on the computing cluster at the University of Pennsylvania. The raw data were stored on a high-volume drive that was able to connect with the cluster via a low-speed switch. The hardware capable of performing any sort of high-performance processing (i.e. holding the data in RAM) were ten ``compute nodes" connected to the cluster via a high-speed switch, and mounted in an NFS architecture. The compute nodes could only hold $\sim 10$ PAPER-128 files in storage. 

The processing stages for compression of a night of PAPER data, described below, required knowledge of the location and compression state of not only individual files, but also the neighbors-in-time of the file in question, in order to implement the DDR filter described above. To supervise the compression we created a MySQL database, which we interacted with via Shell and Python scripts. The database contained a table for the data files under processing and their compression state, a table of neighbor-relations, a table of file details, and a table of the processing nodes available. The schema of this database is shown in Figure~\ref{fig:database_schema}.

\begin{figure}[h]
\centering
\includegraphics[scale=0.7]{chapters/data_processing/figures/RTP_diagram.pdf}
\caption[The schema of the database used to organize and implement PAPER data compression.]{The schema of the database used to organize and implement PAPER data compression.}
\label{fig:database_schema}
\end{figure}

To implement the compression, per file, the following steps were required:

\begin{enumerate}
\item Copying the file from the storage volume to the cluster. For a single night of data, this required roughly 8 hours.
\item Copying the file from the cluster to the compute node. This required roughly 5 minutes.
\item Generate copy of the file, with metadata corrections. This required roughly 1 minute.
\item Delete the raw file.
\item RFI-flag the high frequency-resolution data. This required roughly 2 minutes.
\item Delete the metadata-corrected file.
\item Acquire time-neighbors to the file in question, and bring them to the RFI-flagged stage. The time required for this stage varied with cluster activity, but usually required roughly 20 minutes.
\item DDR filter the RFI-flagged data, using an high-tolerance iterative CLEAN. This required roughly 20 minutes.
\item RFI flag the compressed data (coarse flagging), saving the flags to a separate file. This required roughly 1 minute.
\item Apply the coarse RFI flags to the \textit{uncompressed}, RFI-flagged data. This required roughly one minute.
\item DDR filter the now twice-RFI-flagged data, using a low-tolerance iterative CLEAN. This required roughly 120 minutes.
\item Copy compressed data to the cluster.
\item Delete the twice-RFI-flagged data.
\item If the once-RFI-flagged data are not required as neighbors, delete them.
\item Delete the compressed data from the compute node.
\item If neighbors have already been compressed, delete them, otherwise begin their compression.
\item After all files are compressed, delete the uncompressed files from the cluster.
\end{enumerate}

In total, this meant that across ten compute nodes, and efficient use of the fact that the neighbors could progress through the processing stages while the central file was being compressed, meant that it took roughly 20 to 24 hours to compress a night of observations. 

\section{Radio Frequency Interference}
\label{sec:RFI}

As noted above, RFI was able to introduce spectral and temporal structure that would cause ringing in the data during compression if it was not flagged. This meant that both identification and characterization of RFI was crucial to the scientific goals of the PAPER and HERA experiments. In Section~\ref{subsec:rfi_paper128}, I present characterization of RFI in the second season of PAPER-128 data. By averaging flags in local time I was able to investigate ``repeat offender" frequency bands and identify outlying ``quiet" and ``loud" days.
In Section~\ref{subsec:rfi_hera19paper19} I analyze RFI flags from the first Internal Data Release (IDR1) of HERA commissioning data, which contained 19 HERA feeds suspended above 5\,m dishes in a close-packed hexagon, and 19 PAPER feeds in the same positions as the central dishes, allowing us to investigate the difference in flagging between feeds at different altitudes.

\subsection{PAPER-128}
\label{subsec:rfi_paper128}

The PAPER-128 2014 observation season ran from 18th June 2014 through the 30th April 2015. During this run, some 150 nights of data were recorded. A ``night", which I will refer to using the JD at the start of observations, consists of twelve hours of observation from 6pm to 6am South African Standard Time (SAST). Observations as processed by the PAPER correlator are recorded in {\sc miriad} uv files. These files contain visibilities for each antenna pair in the array. Each integration is 20 seconds long over 1024 frequency bins from 100 to 200\,MHz. Each uv file contains 56 integrations per antenna pair, and 72 uv files are recorded per linear polarization (xx, xy, yx, yy) per night.

Early in the PAPER data compression process, visibilities are flagged for RFI. This is accomplished by the {\tt aipy} script \textit{xrfi\_simple.py}, which takes the derivative of the frequency axis of all baselines associated with a single antenna, and flags any frequencies with a derivative $\geqslant 6\sigma$ above the mean. We always flag the band-edges ($\sim$7\,MHz on each side), since these frequencies are not useful to us, and always flag the 137$\pm$0.6\,MHz band associated with ORBCOMM satellite network transmissions. This process is repeated per integration within each uv file and stored in a Python numpy zip (npz) file. This means that any baseline associated with antenna 1 can contribute a flag to the resultant npz file, which in turn is applied to the data.

The result is 280 files of high-time and -frequency resolution files per night per linear polarization containing information about the RFI environment of the HERA site. I report on the properties of these flags in time- and frequency-space over the 2014 observation season. This section is organized it as follows: in Section~\ref{subsubsec:avgprops}, I analyse the average properties of RFI over the season by stacking flags in local time and normalizing appropriately. In Section~\ref{subsubsec:indivprops} I address nights with particularly strange RFI properties. I discuss the implications of my findings in Section~\ref{subsubsec:rfi_paper128_conc}.

\subsubsection{Average Properties}
\label{subsubsec:avgprops}

In order to assess the average properties of the RFI environment, I calculated a weighted average of flags over the season. Over 150 nights, one-time occurrences are washed-out beneath the 1\% level, allowing me to assess persistent issues.

Nominally, each night should grant 3920 integrations-worth of flags over 1024 frequency bins, per linear polarization. In reality, most of the time this holds true, but occasionally not all files are compressible (hence failing to generate flags) or observations fail to start at the correct time (so there are no data to flag). Also, in the event of an X-engine failure within the correlator, contiguous chunks of the band (in eighths, i.e. 25\,MHz across) are flagged-out, usually for the rest of the night.

For this reason, I calculated a weighted average of the flags across the season, but neglected nights with correlator failures or late starts. Weights were simply the number of nights that contained that integration-bin in SAST. The resultant ``flag density waterfall" is shown in Figure~\ref{fig:rfi_psa128_waterfall}. The color scale is indicative of flagging frequency across the season, and line plots above and to the right of the the waterfall showing the percentage of times and frequencies that were flagged, respectively. 

A summary of the persistent (flagged $\geq1\%$ of the time per channel) RFI frequencies can be found in Table~\ref{tab:rfi_psa128}. I have investigated each frequency and tried to find the most likely source for each. In most cases, this required looking at the properties in time as well as frequency. Others were more obvious from frequency alone, e.g. the 149.8\, MHz transmission frequency from the International Space Station (ISS). Still others I could not track down a convincing explanation for, and these are listed with a `?'. A `?' next to a possible cause indicates that the listed cause is the most prevalent at that frequency, but that the temporal properties of that cause do not necessarily make sense. Many of the characterizations arise from the South African Table of Frequency Allocations (SATFA; \cite{SAFreqTable}).

\begin{figure}[h!]
\centering
\includegraphics[width=\textwidth]{chapters/data_processing/figures/RFI_all_128.pdf}
\caption[A waterfall plot of RFI flags averaged over 150 days of PAPER-128 data.]{A waterfall plot of RFI flags averaged over 150 days of data. The gridding process is described in the text. Above the waterfall I show the percentage of the season each frequency is flagged, and to the right I show the percentage of frequencies that are flagged per integration.}
\label{fig:rfi_psa128_waterfall}
\end{figure}

\begin{deluxetable}{llll}
\centering
\label{tab:rfi_psa128}
\tablewidth{0pt}
\tablecaption{PAPER-128 RFI frequencies and brief characterization for the averaged flags.}
\tabletypesize{\footnotesize}
\tablehead{
\colhead{$\nu$} & \colhead{Flagged} & \colhead{Cause} & \colhead{Notes or} \\
\colhead{MHz} & \colhead{\%} & \colhead{(Possible)} & \colhead{Time (SAST) Characterization} \\
}
\startdata
103	$\pm$	3	&	100	&	BAND EDGE	&	Built-in to flagger.	\\
107.25	$\pm$	0.25	&	2.6	&	FM radio	&	Constant background at 2\% level	\\
107.55	$\pm$	0.05	&	1.9	&	FM radio	&	Constant background at 2\% level	\\
108.1	$\pm$	0.4	&	9	&	FM radio?	&	Rises with time, peaking at midnight and 4am	\\
109	$\pm$	0.4	&	11.5	&	FM radio?	&	Rises with time, peaking around 4am	\\
112.8	$\pm$	0.1	&	1.4	&	Aircraft?	&	Constant background at 1\% level	\\
114.05	$\pm$	0.85	&	3.7	&	?1	&	Decreases till midnight; peak at 4am	\\
116.55	$\pm$	0.35	&	2.2	&	?2	&	Peak at midnight	\\
120.15	$\pm$	0.35	&	3.2	&	Aircraft	&	Roughly follows CPT$\leftrightarrow$JNB flight times	\\
124.95	$\pm$	0.35	&	5.5	&	Aircraft	&	Roughly follows CPT$\leftrightarrow$JNB flight times	\\
130.25	$\pm$	0.55	&	4.3	&	?3	&	Falls (7pm) and rises (3am) steeply 	\\
131.75	$\pm$	0.35	&	10.3	&	Aircraft?	&	Peaks at 6:30, 7:30, 8:30, 9:30, 10 and then a steep falloff	\\
136.05	$\pm$	0.45	&	33.1	&	Radar?	&	Decreases over night	\\
137.35	$\pm$	0.85	&	100	&	ORBCOMM	&		\\
141.45	$\pm$	0.35	&	2.1	&	Mobile phones?	&	High until 9pm, then at background 1\% level	\\
145.85	$\pm$	0.45	&	10.7	&	Amateur radio& Strong 9pm-1am -- this is the official downlink for ISS-HAM	\\
149.75	$\pm$	0.55	&	90.7	&	ISS	&	``Beeps", but in stacked data peaks 2am	\\
175.15	$\pm$	0.35	&	20.5	&	VHF TV	 (video) &	 Channel 4. Peaks at 8:30pm, then falls to background 7\%	\\
181.15	$\pm$	0.15	&	1.6	&	VHF TV (audio) & Channel 4. 2\% level turns-off at 10pm	\\
182.15	$\pm$	0.35	&	75	&	?4	&	Decreases until 10pm (to ~15\%), when it begins a slow rise again	\\
183.2	$\pm$	0.5	&	89.7	&	VHF TV (video)	&	Channel 5. Rises throughout night. 	\\
186.25	$\pm$	0.35	&	4.6	&	?5	&	Extreme turn-off at 9:45	\\
189.15	$\pm$	0.35	&	41.4	&	VHF TV (audio)	&	Channel 5. Rises throughout night. 	\\
189.9	$\pm$	0.4	&	100	&	VHF TV 	&	Channel 6. Built-in to flagger. \\
191.1	$\pm$	0.3	&	100	&	VHF TV 	&	Channel 7. Built-in to flagger. \\
196	$\pm$	4	&	100	&	BAND EDGE	&
\enddata
\end{deluxetable}

Figure~\ref{fig:rfi_psa128_freqflags} shows the detail of the top panel of Figure~\ref{fig:rfi_psa128_waterfall}. This figure highlights the broad swath of the band from roughly 150 to 180\,MHz that was, on average, clear of RFI. This roughly corresponds to 21\,cm redshifts $z=6.9$ to 8.5. This is one of the reasons that the \cite{Parsons.14} and \cite{Ali.15} limits on the 21\,cm power spectrum concentrated on this redshift range -- there were simply more unflagged data to average-down with. \cite{Furlanetto.06} show that the $z\sim8$ universe can be considered roughly coeval over an $\sim$8\,MHz bandwidth. As such, the 30\,MHz chunk could be used to create $\sim$3 power spectra, as demonstrated in \cite{Jacobs.15}. As we show below, the deactivation of VHF TV broadcasts could enable measurements up to the band edge.

\begin{figure}[h!]
\centering
\includegraphics[width=\textwidth]{chapters/data_processing/figures/RFI_149days_freq.png}
\caption{The percentage of time that each frequency was flagged over the season.}
\label{fig:rfi_psa128_freqflags}
\end{figure}

\subsubsection*{FM Radio}

SATFA lists the frequency band 87.5--108\,MHz as available for FM radio broadcasts, leading me to postulate that the low-level RFI we observed in the 107.25	$\pm$	0.25 and 107.55	$\pm$	0.05\,MHz bands had FM radio as the leading cause. The 108.1	$\pm$	0.4 and 109	$\pm$	0.4\,MHz bands were outside of the official range, and exhibit odd temporal properties for human activity -- two peaks at midnight and 4am -- with a increasing number of flags throughout the average night (see Figure~\ref{fig:rfi_psa128_FMradio}). \\

\begin{figure}[h!]
\centering
\includegraphics[width=0.4\textwidth]{chapters/data_processing/figures/FB_71_77.png}
\includegraphics[width=0.4\textwidth]{chapters/data_processing/figures/FB_77_79.png}
\includegraphics[width=0.4\textwidth]{chapters/data_processing/figures/FB_80_87.png}
\includegraphics[width=0.4\textwidth]{chapters/data_processing/figures/FB_89_97.png}
\caption[Possible FM radio contamination.]{Possible FM radio contamination in the \textit{Top, left to right}: 107.25$\pm$0.25 and 107.55$\pm$0.05\,MHz bands, and \textit{Bottom, left to right}: 108.1$\pm$0.4 and 109$\pm$0.4\,MHz bands.}
\label{fig:rfi_psa128_FMradio}
\end{figure}

\subsubsection*{Aircraft communications}
It was difficult to argue that the 112.1$\pm$0.1\,MHz signal is caused by aircraft communications since it maintained a constant background level. However, SATFA listed this frequency as reserved for aircraft communications and it has been used in the past as a calibration frequency for aircraft instruments \cite{AircraftCalibrationFreqs}.\\

The other aircraft frequencies were obvious, because they closely traced the 2-hour flight from Cape Town to Johannesburg\footnote{Credit to Danny Jacobs for first spotting this and noting it in an internal PAPER circular in December 2009.}. 
An example (120.15$\pm$0.35\,MHz) is shown in Figure~\ref{fig:rfi_psa128_aircraft}. SATFA reserved frequencies 108--117.975\,MHz for aeronautical radionavigation and 117.975--137\,MHz for aeronautical mobile. In Table~\ref{tab:rfi_psa128} I listed 131.75$\pm$0.35\,MHz as caused by aircraft since it falls in the aeronautical mobile band, but it does not follow the flight patterns as closely as the other bands.\\

\begin{figure}[h!]
\centering
\includegraphics[width=\textwidth]{/Users/saulkohn/Documents/thesis/chapters/data_processing/figures/airlines.png}
\caption[Flights from Cape Town to Johannesburg correspond to RFI in the 120.15$\pm$0.35\,MHz channels.]{
Flights from Cape Town to Johannesburg correspond to RFI in the 120.15$\pm$0.35\,MHz channels. Vertical dashed lines indicate a flight leaving Cape Town (flights from Johannesburg are roughly concurrent) and the flight code is listed. The transparency of a line is inversely proportional to how many days a week that flight is scheduled for. The flight is 2 to 2.5 hours long -- and about 2 hours after the last flight of the day, the flags fall to background level (but notably, not always to zero).}
\label{fig:rfi_psa128_aircraft}
\end{figure} 

\subsubsection*{Orbital communications}
ORBCOMM Inc.'s constellation of 29 LEO communication satellites is a well-known contaminant of the low-frequency sky, dominating over any astronomical signal at 137--138\,MHz (although each satellite emits within a 20\,kHz band). For this reason there was built-in flagging at 137.35$\pm$0.85\,MHz within the compression pipeline.

The largest contaminant without built-in flags in the pipeline were communications from the ISS. The 149.75$\pm$0.55\,MHz transmissions were semi-regular in time; they `beep'. 

Onboard the ISS are HAM radio devices. Some countries have also launched satellites with these onboard, one of the purposes of which is to provide HAM radio operators something in space to communicate with. These devices are licensed to operate at 145.2 and 145.8\,MHz, and SATFA listed the 144-146\,MHz band as reserved for `Amateur--Satellite' communications. We detected RFI at 145.85$\pm$0.45\,MHz, although strong signal across $\sim$10\% of the season that occurs 9pm-1am argues against human operation.

\subsubsection*{Mobile phones and VHF TV}
A weak RFI signal at 141.45$\pm$0.35\,MHz was within the `mobile 1 BTX'  and aeronautical mobile band in SATFA, but other than this single listing I did not build a strong case for the signal's cause.

VHF TV is broadcast over specifically-spaced video and audio frequencies. The strong signals at 183.2$\pm$0.5\,MHz and 189.15$\pm$0.35\,MHz had almost identical gradients for the percentage of flagging as a function of time of night. These frequencies corresponded exactly to Channel 5 of South African System I 625-line VHF TV signals for video and audio transmission, respectively. Similarly, the weaker signals at 175.15$\pm$0.35	 and 181.15$\pm$0.15\,MHz corresponded to Channel 4's video and audio transmission, respectively, but they did not share the same temporal properties.

\subsubsection*{Unidentified sources}

There were 5 RFI frequencies in the averaged data that I could not identify the sources of:  weak emissions (flagged $<5\%$ of the season) at 114.05$\pm$0.85, 116.55$\pm$0.35, 130.25$\pm$0.55 and 186.25$\pm$0.35\,MHz, and one strong emission at 182.15$\pm$0.35\,MHz. The variation of each source with time is shown in Figure~\ref{fig:rfi_psa128_unidentified}. The 186.25$\pm$0.35\,MHz had a sharp turn-off around 9:45pm each night, suggesting that it originated from some kind of automated device.\\

\begin{figure}[h!]
\centering
\includegraphics[width=0.4\textwidth]{chapters/data_processing/figures/FB_135_142.png}
\includegraphics[width=0.4\textwidth]{chapters/data_processing/figures/FB_166_173.png}
\includegraphics[width=0.4\textwidth]{chapters/data_processing/figures/FB_304_315.png}
\includegraphics[width=0.4\textwidth]{chapters/data_processing/figures/FB_837_844.png}
\includegraphics[width=0.4\textwidth]{chapters/data_processing/figures/FB_882_884.png}
\caption[The temporal profile of the 5 RFI frequencies with unidentified causes.]{The temporal profile of the 5 RFI frequencies with unidentified causes. \textit{Top, left to right:} 114.05$\pm$0.85 and 116.55$\pm$0.35\,MHz. \textit{Middle, left to right:} 130.25$\pm$0.55 and 182.15$\pm$0.35\,MHz. \textit{Bottom:} 186.25$\pm$0.35\,MHz. The 182.15$\pm$0.35\,MHz frequency is flagged a large amount of the time, making it our most-offending unidentified source.}
\label{fig:rfi_psa128_unidentified}
\end{figure}

\subsubsection{Individual Properties}
\label{subsubsec:indivprops}

Using the flags per night, I was able to assess the total number of flags as a percentage of the waterfall (i.e. $N_{\rm flags}/(3920\times1024)$). The average flagging per night was 19.2$\pm$0.5\%, which was dominated by the permanent flagging of ORBCOMM and band edges. Four nights deviated from the average by a $\geq 2\sigma$ excess: JDs 2456965, 2456732, 2456958 and 2457038. Their flag waterfalls are shown in Figure~\ref{fig:rfi_psa128_worst} (2456732, 2456958 and 2457038) and Figure~\ref{fig:rfi_psa128_wandering} (2456965). While the strange nature of night 2456965 is discussed below, the three others followed the pattern of having strong contamination from FM and aircraft communication bands, but also had broadband `pulses' up to about 20 minutes in length. The source of these broadband pulses is not well understood, although it was clear that ORBCOMM tends to spill outside of its allocated band on occasion.

\begin{figure}[h!]
\centering
\includegraphics[width=0.3\textwidth]{chapters/data_processing/figures/2456732RFI.png}
\includegraphics[width=0.3\textwidth]{chapters/data_processing/figures/2456958RFI.png}
\includegraphics[width=0.3\textwidth]{chapters/data_processing/figures/2457038RFI.png}
\caption[Waterfalls of RFI flags for nights 2456732, 2456958 and 2457038.]{\textit{Left to right:} waterfalls of flags for nights 2456732, 2456958 and 2457038. These three nights, along with 2456965, are $>$20.2\% flagged; $>2\sigma$ above the average flagging amount per night.}
\label{fig:rfi_psa128_worst}
\end{figure}


JD 2456965 was easily the worst offender, and it exhibited a strange signal that wanders in frequency and time close to the ISS band. An event of note on this date (23rd November 2014) was a Soyuz FG launch that docked with the ISS -- this may have been a signature of their transmissions\footnote{The internet also suggests... less plausible explanations: \url{https://www.youtube.com/watch?t=11&v=VtZx8iPO4zs}. }.
Similar signals were seen on 2456898 (28th August 2014; although only at the beginning of the night) and 2456924 (23rd September 2014). There was no listed orbital or suborbital activity for 2456898. There was an US ICBM test off of the coast of Virginia on 2456924, but this was probably not the cause of the RFI. The flag waterfalls for these nights are shown in Figure~\ref{fig:wandering}.\\

\begin{figure}[h!]
\centering
\includegraphics[width=0.3\textwidth]{chapters/data_processing/figures/2456898RFI.png}
\includegraphics[width=0.3\textwidth]{chapters/data_processing/figures/2456924RFI.png}
\includegraphics[width=0.3\textwidth]{chapters/data_processing/figures/2456965RFI.png}
\caption[Waterfalls of RFI flags for nights 2456898, 2456924 and 2456965]{\textit{Left to right:} waterfalls of flags for nights 2456898, 2456924 and 2456965. These three nights exhibit a strange behaviour of RFI that changes in frequency and time. JD 2456965 is by far the worst, and during this night as well as 2456898, we see a broadband `comb' of flagged frequencies near the band edges}
\label{fig:rfi_psa128_wandering}
\end{figure}

Another property that the flag waterfalls in Figures~\ref{fig:rfi_psa128_worst} and \ref{fig:rfi_psa128_wandering} highlight is the presence of broadband RFI signals, typically present at frequencies lower than the ORBCOMM band. However, while we flagged at the low-end of the band (which had higher noise levels to begin with), it is likely that such broadband pulses dominated the band at those times, and that we failed to flag all of the integrations. Our flagging routine \textit{xrfi\_simple.py} does contain a thresholding option for flagging the entire integration given some arbitrary number of frequencies flagged during that integration: some experimentation will be required to decide if that threshold should change.

\subsubsection{Discussion}
\label{subsubsec:rfi_paper128_conc}

Based on my findings, I was able to recommend some actions that could be taken in the KRQZ to enable better measurements:
\begin{itemize}
\item  Steps to reduce and ideally eliminate the VHF TV transmissions in the area would be very helpful, since these were clearly interfering with our measurements in the high-end of the band.
\item The ISS 149.75$\pm$0.55\,MHz band should be permanently flagged within the compression pipeline.
\item Pursuing re-routing of flight paths will not do much to help: we see aircraft signals for the duration of their flight, not just when they're over the Karoo.
\item A lower threshold for identifying broadband RFI should be optimized.
\end{itemize}

A new, lower-frequency feed is currently under development by the HERA analog group. This would nominally allow measurements to be taken in the range 50--250\,MHz, allowing science observations of the Dark Ages and the post-reionization Universe. It should be noted that at the lowest frequencies FM radio will be a constant harassment to these measurements. At the higher frequencies, VHF TV will be the primary contaminant, but should be much easier to remove as it is both narrow-band and within the KRQZ's power to shut off.

\subsection{HERA-19 and PAPER-19}
\label{subsec:rfi_hera19paper19}

The HERA-19 IDR1 consisted of four subarrays: the HERA-19 hexagon of dishes (the `HERA Hex'), a hexagon of 19 PAPER dipoles in the exact positions of not-yet-constructed HERA dishes (the `PAPER Hex'), an imaging array and an experimental array for polarization measurements. I will concentrate on the two Hexes in this Section. I analyzed RFI as flagged in the linear \textit{xx}-polarization only. Asymmetric beams can in principle receive different RFI events for different linear polarizations, but analysis of that was outside the scope of this diagnostic study.

IDR1 consisted of one `golden day', JD 2457458. This ran from 6pm on March 10th 2016 to 6am the following day. This gave, per baseline, roughly 4000 integrations of 10 seconds each over 1024, 100~kHz frequency channels from 100 to 200~MHz.

In order to flag RFI I used the {\tt aipy} script \textit{xrfi\_simple.py}. I took the union of all baseline flags as data to analyze. Unlike in Section~\ref{subsec:rfi_paper128}, these data did not have a-priori flagging of band edges, which allowed me to make a more complete study of RFI in the HERA band. I did have to implement custom flags in order to get more than a zeroth-order view of the RFI (since these would dominate the flagging routine unless they are flagged already), but my results from Section~\ref{subsec:rfi_paper128}  gave a better idea of what was flagged to get there. 

Below I present measurements of high-power, mostly narrow-band RFI channels as flagged in HERA Hex data and PAPER Hex data separately. In both cases, I was able to list any channels that are flagged for $\geqslant 1\%$ of the night. I could then compare the flagging Hex-to-Hex, and to PAPER-128.

\subsubsection{HERA Hex RFI}
\label{subsubsec:rfi_herahex}

Table~\ref{tab:rfi_herahex} shows all narrowband frequency ranges flagged in HERA-19 visibilities, with columns of the frequency range in MHz, \% flagging over time, plausible identification, whether or not it was identified in PAPER-128 data, and other notes (often details of the possible identification). Frequencies with 100\% flagging indicate manual flags required for {\tt xrfi\_simple} to work on the rest of the channels. 

Clearly, the low-end of the band was swamped by FM radio broadcasts. One notable frequency was the 109.2$\pm$0.3~MHz band, which was heavily flagged in HERA visibilities, but was only flagged a few percent in PAPER-128 data.

As seen before, ORBCOMM satellite emissions spilled out of their allocated 137-138~MHz band down to 136.3~MHz.

There were many narrowband RFI channels, across the band, that PAPER-128 did not pick-up. Most of these were flagged only at low levels, with two exceptions: 111.3$\pm$0.2~MHz and 113.5$\pm$0.1~MHz. Both of these were in the aircraft navigation band. There is some evidence \citep{airforce} that 111.3~MHz band is fir air force communications. The 113.5~MHz band is a known band for radionavigation beacons (`VOR navaids') \citep{navaid}.

A particularly annoying `new' emitter was in the 153.8$\pm$0.2~MHz region, which is close to the center of our nominal EoR band. It could correspond to mobile phones being used close to site.

\begin{deluxetable}{lllll}				
\centering														
\label{tab:rfi_herahex}
\tablewidth{0pt}
\tablecaption{RFI as flagged by HERA}
\tabletypesize{\footnotesize}
\tablehead{
\colhead{$\nu$} & \colhead{Flagged} & \colhead{Cause} & \colhead{Seen by} & \colhead{Notes} \\
\colhead{MHz} & \colhead{\%} & \colhead{(Possible)} & \colhead{PAPER-128} &\colhead{} \\
}
\startdata													
100.7	$\pm$	0.2	&	50	&	FM Radio	&	n/a	&	RSG ``Dis Die Een" Prieska		\\
101.5	$\pm$	0.3	&	36	&	FM Radio	&	n/a	&	RSG ``Dis Die Een" Calvinia	\\
102.4	$\pm$	0.1	&	100	&	FM Radio	&	n/a	&	RSG ``Dis Die Een" Carnarvon	\\
102.8	$\pm$	0.3	&	57	&	FM Radio	&	n/a	&	RSG ``Dis Die Een" Pofadder	\\
104.2	$\pm$	0.1	&	100	&	FM Radio	&	n/a	&	SAfm Prieska	\\
105.1	$\pm$	0.2	&	100	&	FM Radio	&	n/a	&	SAfm Calvinia	\\
106.2	$\pm$	0.3	&	100	&	FM Radio	&	n/a	&	SAfm Carnarvon	\\
106.9	$\pm$	0.1	&	15	&	FM Radio	&	n/a	&	Sentech	\\
107.2	$\pm$	0.1	&	18	&	FM Radio	&	Yes	&		\\
107.8	$\pm$	0.2	&	15	&	FM Radio	&	Yes	&		\\													
108.3	$\pm$	0.1	&	31	&	FM Radio?	&	Yes	&			\\
109.2	$\pm$	0.3	&	93	&	FM Radio?	&	Yes...&	...but not to this degree	\\
111.3	$\pm$	0.2	&	25	&	Air force?	&	No	&		\\
112.5	$\pm$	0.1	&	5	    &	Aircraft?	&	No	&	\\
113.5	$\pm$	0.1	&	21	&	Aircraft    &	No	&	VOR navaids		\\
115.5	$\pm$	0.1	&	3	    &	Navaids?	    &	No	&	\\
115.9	$\pm$	0.1	&	3	    &	Navaids?	    &	No	&	\\
116.6	$\pm$	0.2	&	9	   &	Aircraft?	    &	Yes	&	VOR-DME navaids		\\
120.1	$\pm$	0.2	&	5	&	Aircraft	&	Yes	&	CPT$<->$JNB		\\
125.0$\pm$	0.2	&	6	&	Aircraft	&	Yes	&	CPT$<->$JNB		\\
130.0	$\pm$	0.2	&	4	&	Aircraft	&	No	&	Communication		\\
131.6	$\pm$	0.2	&	15	&	Aircraft	&	Yes	&	KLM OPS		\\
136.4	$\pm$	0.1	&	9	&	ORBCOMM	&	Yes	&					\\
136.7	$\pm$	0.1	&	10	&	ORBCOMM	&	Yes	&					\\
137.4	$\pm$	0.4	&	100	&	ORBCOMM	&	Yes	&				\\
145.7	$\pm$	0.4	&	18	&	ISS/Amateur Radio band	&	Yes	&			\\
149.9	$\pm$	0.1	&	100	&	ISS	&	Yes	&			\\
153.8	$\pm$	0.2	&	7	&	Mobile phones?	&	No	&			\\
175.0	$\pm$	0.1	&	100	&	VHF TV	&	Yes	&	Channel 4 Video		\\
178.3	$\pm$	0.2	&	8	&	VHF TV	&	No	&	Channel 7?		\\
181.2	$\pm$	0.1	&	100	&	VHF TV	&	Yes	&	Channel 4 Audio		\\
182.2  $\pm$	0.2	&	9	&		&	Yes	&			\\
183.5	$\pm$	0.6	&	100	&	VHF TV	&	Yes	&	Channel 5 Video		\\
184.1	$\pm$	0.1	&	2	&	VHF TV?	&	Yes	&	Channel 5?	\\
184.7	$\pm$	0.1	&	6	&	Broadcasting	&	No	&		\\
187.8	$\pm$	0.1	&	4	&		&	No	&		\\
189.1	$\pm$	0.1	&	52	&	VHF TV	&	Yes	&	Channel 5 Audio	\\
190.1	$\pm$	0.3	&	13	&		&	n/a	&		\\
191.1	$\pm$	0.1	&	100	&	VHF TV	&	n/a	&	Channel 7		\\
197.2	$\pm$	0.2	&	18	&		&	n/a	&		\\
199.4	$\pm$	0.5	&	100	&	BAND EDGE	&	n/a	&		\\
\enddata
\end{deluxetable}

\subsubsection{PAPER Hex RFI}
\label{subsubsec:rfi_paperhex}

Table~\ref{tab:rfi_paperhex} has the same description as Table~\ref{tab:rfi_herahex}, but for the PAPER Hex. There were far fewer RFI frequencies flagged in PAPER visibilities, almost all of which were seen by HERA. The only RFI seen by the PAPER Hex and not the HERA Hex was the 123.5$\pm$0.1~MHz emission, which I could find a plausible identification for.

\begin{deluxetable}{lllll}													
\centering														
\label{tab:rfi_paperhex}
\tablewidth{0pt}
\tablecaption{RFI as flagged by the PAPER Hex}	
\tabletypesize{\footnotesize}
\tablehead{
\colhead{$\nu$} & \colhead{Flagged} & \colhead{Cause} & \colhead{Seen by} & \colhead{Notes} \\
\colhead{MHz} & \colhead{\%} & \colhead{(Possible)} & \colhead{PAPER-128} &\colhead{} \\
}
\startdata	
100.0	$\pm$	0.1	&	100	&	BAND EDGE	&	n/a	&		\\		
100.7	$\pm$	0.1	&	11	&	FM Radio	&	n/a	&	RSG ``Dis Die Een" Calvinia			\\		
101.6	$\pm$	0.2	&	6	&	FM Radio	&	n/a	&	RSG ``Dis Die Een" Calvinia			\\		
102.4	$\pm$	0.1	&	100	&	FM Radio	&	n/a	&	RSG ``Dis Die Een" Carnarvon		\\		
102.7	$\pm$	0.1&	100	&	FM Radio	&	n/a	&	RSG ``Dis Die Een" Pofadder		\\		
104.2	$\pm$	0.2	&	100	&	FM Radio	&	n/a	&	SAfm Prieska		\\		
105.1	$\pm$	0.2	&	100	&	FM Radio	&	n/a	&	SAfm Calvinia	\\		
106.2	$\pm$	0.3	&	100	&	FM Radio	&	n/a	&	SAfm Carnarvon	\\		
																
108.2	$\pm$	0.1	&	3	&	FM Radio?	&	Yes	&		\\		
109.1	$\pm$	0.1  &	26	&	FM Radio?	&	Yes	&		\\		
113.6	$\pm$	0.1	&	2	&	Airplane Communications	&	No	&	VOR navaid			\\		
																
120.2	$\pm$	0.3	&	3	&	Aircraft	&	Yes	&	CPT$<->$JNB		\\		
123.5	$\pm$	0.1	&	1	&		&	No	&	Not seen by HERA			\\		
125.0	$\pm$	0.2	&	6	&	Aircraft	&	Yes	&	CPT$<->$JNB		\\		
130.0	$\pm$	0.3	&	3	&		&	No	&			\\		
131.7	$\pm$	0.2	&	14	&	Aircraft	&	Yes	&				\\		
136.4	$\pm$	0.2	&	6	&	ORBCOMM	&	Yes	&					\\		
136.7	$\pm$	0.2	&	6	&	ORBCOMM	&	Yes	&				\\		
															
137.4	$\pm$	0.4	&	100	&	ORBCOMM	&	Yes	&				\\		
145.8	$\pm$	0.3	&	14	&	ISS/Amateur Radio band	&	Yes	&			\\		
149.9	$\pm$	0.1	&	100	&	ISS	&	Yes	&			\\		
153.8	$\pm$	0.2	&	3	&	Single frequency mobile phones?	&	No	&				\\		
																
175.1	$\pm$	0.2	&	100	&	VHF TV	&	Yes	&	Channel 4 Video				\\		
178.3	$\pm$	0.2	&	100	&	VHF TV	&	No	&	Channel 7?		\\		
181.2	$\pm$	0.1	&	100	&	VHF TV	&	Yes	&	Channel 4 Audio	\\		
183.2	$\pm$	0.2	&	100	&	VHF TV	&	Yes	&	Channel 5 Video			\\		
189.2	$\pm$	0.1	&	100	&	VHF TV	&	Yes	&	Channel 5 Audio			\\		
191.2	$\pm$	0.1	&	100	&	VHF TV	&	n/a	&	Channel 7			\\		
199.8	$\pm$	0.2	&	100	&	BAND EDGE	&	n/a	&			\\
\enddata
\end{deluxetable}

\subsubsection{Hex-to-Hex Comparisons}
\label{subsubsec:hex2hex_comparison}

As mentioned above, the PAPER Hex saw far fewer narrowband RFI channels than HERA does. This highlighted an interesting trade-off between dipoles and dishes: at first glance, one might have expected PAPER dipoles to be more susceptible to RFI given their broader effective beams. However, HERA dipoles are lifted several meters above the ground, and this change in height may have been the source of the greater susceptibility to RFI. RFI comes from the horizon, which would be more easily received in the far sidelobes of the beam.

\begin{figure}[h!]
\centering
\includegraphics[width=0.45\textwidth]{chapters/data_processing/figures/RFI_HH_spec.pdf}
\includegraphics[width=0.45\textwidth]{chapters/data_processing/figures/RFI_PH_spec.pdf}
\caption[Frequency vs. percentage flagging for the HERA Hex and PAPER Hex.]{Frequency vs. percentage flagging for the HERA Hex (\textit{left}) and PAPER Hex (\textit{right}). Any band with greater than 1\% flagging is reported in Tables~\ref{tab:rfi_herahex} and \ref{tab:rfi_paperhex}.}
\label{fig:rfi_spec_hex_comparison}
\end{figure}

Even for the RFI channels they did share, HERA flagged them more often. Taking the difference in percentage-flagging for the common RFI channels (think of the left panel subtracted from the right panel for common channels in Figure~\ref{fig:rfi_spec_hex_comparison}), those channels had an average of 8\% more flagging in HERA visibilities. The difference was particularly high in the aeronautical radionavigation bands, where HERA had on average 38\% more flagging than the PAPER Hex.

Figure~\ref{fig:rfi_hex_waterfalls} shows the flags on a per-sample basis (these were averaged over time to create Figure~\ref{fig:rfi_spec_hex_comparison}). Most apparent was the occupancy of the HERA plot compared to the PAPER Hex one. An important component of this plot is the averages over frequency in the right-hand panels. We saw that the average flagging for a given time sample was about 5\% higher for HERA than for the PAPER Hex, mostly due to the higher occupancy of the FM band. But we also saw something new; HERA appeared to be much more sensitive to broadband bursts of RFI. The PAPER Hex caught one of these events (around 1.30am SAST) at high significance, but most of them hardly rose above average flagging. HERA saw five to seven bursts across the night. 

\begin{figure}
\centering
\includegraphics[width=0.45\textwidth]{chapters/data_processing/figures/RFI_HH_wf_tzoom.pdf}
\includegraphics[width=0.45\textwidth]{chapters/data_processing/figures/RFI_PH_wf_tzoom.pdf}
\caption[RFI flag waterfalls of frequency vs. South Africa Standard Time for the HERA Hex and PAPER Hex.]{RFI flag waterfalls of frequency vs. South Africa Standard Time for the HERA Hex (\textit{left}) and PAPER Hex (\textit{right}). The top panels show the average over time (identical to Figure~\ref{fig:spec}), while the right panels show the average over frequency.}
\label{fig:rfi_hex_waterfalls}
\end{figure}

\subsubsection{Comparison PAPER-128 stacked flags}
Section~\ref{subsec:rfi_paper128} presented RFI flags stacked over 150 days of observations. This method washed-out single events that effect analysis on a single-night basis, but was sensitive to repeatedly offending frequencies. Due to the PAPER-128 analysis pipeline, many channels were automatically flagged (particularly large portions of the band edges), which artificially boosted the average flagging per time and did not allow for closer inspection of the ends of the band. There was some evidence of broadband emission (see Figure~\ref{fig:rfi_psa128_waterfall}) but the band was largely free of RFI in the middle of the night. Obviously, the data presented in this section shows a less-clean band, but it also only concentrated on a single night's data, so it may be that IDR1 was conspicuous compared to an `average' night of observations.

Traits shared between the two analyses were:
\begin{itemize}
\item Aircraft communications disrupting data until around local midnight.
\item ORBCOMM spilling out of it's band.
\item VHF TV frequencies emitting throughout the night in the high end of the band.
\end{itemize} 

Table~\ref{tab:rfi_herahex} highlighted many frequencies seen by HERA and not by PAPER-128. Again, given the fact that flags were stacked and averaged in Section~\ref{subsec:rfi_paper128}, these may not have been `new', but they could have been. Particularly conspicuous were the emissions in the aeronautical radionavigation band.

\subsubsection{Discussion}
\label{subsubsec:rfi_herapaper_conc}

I have presented a first look at RFI in HERA-19 Commissioning data. Probably due to the height of the receiving element on HERA versus PAPER dipoles, much more RFI was apparent, especially on the low and high ends of the band. Luckily, the EoR band was largely clean of RFI, except for an emitter at about 154~MHz, which could have corresponded to single-frequency mobile phone communications. Such communications are officially banned in the SKA Radio Quiet Zone \citep{SAKRQZ}, which HERA is in the center of.

Only looking at a single night of RFI flags limited the predictive power of this study. More data will be required to establish whether or not this is level of RFI was `normal'. Broadband RFI bursts require closer investigation.

Efforts to extend the HERA band to lower and possibly higher frequencies are currently under way. The FM radio band extends to around 65~MHz, while the VHF TV band extends to around 230~MHz, so the RFI environment should be a consideration for these efforts.

Meanwhile, I note that the RFI flagging routine used here, {\tt xrfi\_simple}, is indeed `simple'. More advanced RFI flagging algorithms such as {\tt AOFlagger} \citep{AOflag} should be tested in later studies.

\section{Pre-Redundant Calibration QA}
% should I wait for figures from the Ali RTP paper??

\section{Post-Redundant Calibration QA}
% should I wait for figures from the Ali RTP paper??
%
%
%
\chapter{Polarimetric Calibration}

\section{Redundant Calibration}

\section{Imaging Calibration}

%
% only use the high band in power spectra
%
\chapter{The Ionosphere}
\label{chapter:ionosphere}

The ionosphere is a section of Earth's atmosphere composed of several layers, between 60 and 1000\,km in altitude. It overlaps the Troposphere, Stratosphere, Mesosphere, Thermosphere and Exosphere. The ionosphere is an ionized plasma, composed of ions from molecules in the atmospheric layers it overlaps that are ionized by solar radiation. The ionization state of the ionosphere can be quantified by the Total Electron Content (TEC) -- an integral of electron count in a given direction -- among other metrics. 

Spatiotemporal variations of the TEC are tied to solar activity, and therefore largely both diurnal and seasonal. More ionization, and therefore a larger TEC, is to be expected in the day time and closer to the summer solstice. The Solar Cycle also influences TEC, with more sunspots proportional with a higher TEC; at solar maximum, this effect dominates the seasonal variation \citep{Sotomayor-Beltran.13}. Ionospheric variations are typically described as Kolmogorov turbulence (i.e. small scale motions are isotropic in their direction and scale with wavenumber; \citealt{Zolesi.14}), however, LOFAR observations report deviations from isotropy in their observations \citep{Intema.09, Mevius.16}. Regions of the ionosphere that can be assumed to be constant in density and shape at a given time are referred to as ``isoplanatic patches''. At 74\,MHz, these patches are observed to be $1^{\circ}-2^{\circ}$ in radius \citep{Cotton.02}.

The ionosphere is composed of three main layers: D, E and F, which vary according to the day-night cycle. These are summarized in Section~\ref{tab:ionosphere_layers} (which summarizes Chapter 2 of \citealt{Zolesi.14}). At night, there are not enough high-energy electrons to penetrate to lower altitudes, causing the D layer to recombine. The E layer increases in altitude at night due to a similar effect. The E and F layers persist at all times, but during daylight the F layer is divided into two sub-layers, F$_1$ and F$_2$. 

\begin{deluxetable}{lllll}
\centering
\label{tab:ionosphere_layers}
\tablewidth{0pt}
\tablecaption{Ionospheric Layers}
\tabletypesize{\footnotesize}
\tablehead{
\colhead{Layer} & \colhead{Time} & \colhead{Altitude} &\colhead{Components} & \colhead{Electron Density}\\
\colhead{} & \colhead{} & \colhead{km} & \colhead{} & \colhead{$e^-\,m^{-3}$}
}
\startdata
D & Day & 60--90 & NO$^+$, N$_2$, Ar, O$_2^-$ & $10^8 - 10^9$  \\
E & Day/Night & 90--150 & NO$^+$, O$_2^+$, O$^+$, N$_2^+$ & $10^{11}$ \\
F$_1$ & Day & 140--600 & NO$^+$, O$_2^+$, O$^+$, N$^+$ & $10^{11}$\\
F$_2$ & Day/Night & 220--800 & O$^+$, H$^+$, He$^+$ & $10^{10} - 10^{13}$\\
\enddata
\end{deluxetable}

The diurnal nature of the ionosphere is important to radio propagation. During the day, the D layer reflects radio transmissions much closer to the Earth than during the night, when the E and F layers reflect. This leads to longer-range transmissions being possible after sunset\footnote{This effect was first observed by E. V. Appleton \citep{Appleton.46}, confirming the ionosphere's existence, for which he was awarded the 1947 Nobel Prize in Physics.}.

The relevance of the ionosphere to this work is its coupling with Earth's magnetic field. Recall that, as mentioned in previous chapters, a linearly polarized electromagnetic wave, propagating through an ionized plasma which has an incident magnetic field, will experience Faraday Rotation of its original polarization angle $\chi$:

\begin{equation}
\chi_{\rm obs} = \chi + \phi\lambda^2
\end{equation}

where $\lambda$ is the wavelength, and

\begin{equation}
\phi(\hat{s}) \approx 0.81 \int^{\rm obs}_{\rm source} n_e(\hat{s}) \vec{B}(\hat{s}) \cdot d\vec{s}
\label{eq:ionopshere_rm}
\end{equation}

where the source of the electromagnetic wave is in direction $\hat{s}$ on the sphere, $n_e$ is the electron density scalar field and $\vec{B}$ is the magnetic vector field. The Rotation Measure (RM) $\phi$ is the integral of the product along the line of sight, and has units of rad\,m$^-2$. Since the ionosphere is capable of imparting an additional RM to polarized radio waves, inducing spectral structure to interferometric visibilities, understanding it is crucial to quantifying the effect of polarization on EoR measurements.

In this chapter, I review historical measurements of the ionospheric TEC and RM distributions in Section~\ref{sec:historicalTEC} and modern observations in Section~\ref{sec:lowfreqionosphere}. In Section~\ref{sec:widefieldRMionosphere} I present our work on the role of the ionosphere in PAPER and HERA measurements, and software we developed to quantify those effects.

\section{Historical measurements of TEC and RMs}
\label{sec:historicalTEC}

The existence and layered nature of the ionosphere was confirmed between the 1920s and the 1940s. Measurements of the TEC and RM distributions came later, once radio-communications satellites were put in orbit, and are closely tied to the Global Positioning System (GPS) launched in the late 1970s (called the NAVSTAR system). NAVSTAR GPS satellites transmit at two narrow frequency bands, centered about 1.2276\,GHz (`L$_2$') and 1.57542\,GHz (`L$_1$'). Encoded in these transmissions are the local clock times per satellite (precisely calibrated with one another and with ground clocks) and their positions. With four satellites in view of a receiver, one is capable of computing their three-dimensional position and their local clock relative deviation from the satellite clock time. 

\cite{Macdoran.89} showed that one could use a frequency-dependent time delay induced by the ionospheric plasma \citep{Klobuchar.83, Brunner.93}:

\begin{equation}
\Delta t_{\rm iono} = \frac{40.3}{c\nu^2}{\rm TEC}
\label{eq:delta_iono}
\end{equation}

to calculate an estimate of the TEC in the direction of a GPS satellite. Their approach has been continously refined. 
Using an estimate of the polarization angle of the emitted L$_{1,2}$ transmissions, \cite{Titheridge.72} and \cite{Royden.84} presented measurements of TEC by measuring the Faraday Rotation induced and worked towards an estimate of the TEC based on the RM.
\cite{Lanyi.88} showed that the more accurate method was calculation of the TEC using $\Delta t_{\rm iono}$ from Equation~\ref{eq:delta_iono}.
\cite{Mannucci.98} introduced the Ionosphere Map Exchange Format (IONEX): a method and file format for storing TEC measurements using GPS beacons across the globe, allowing the first global TEC maps to be calculated. IONEX files contain global TEC measurements with a 2 hour cadence and generally 5$^\circ$ by 2.5$^\circ$ resolution in longitude and latitude respectively. They neglect the layered nature the ionosphere, modelling it as a thin sheet.
\cite{Iijima.99} provided a server that automatically pushed IONEX files to the World Wide Web as soon as they could be constructed.
\cite{Komjathy.05} presented the first measurements with over 1000 GPS stations.
Recently, \cite{Erdogan.16} presented a method for time-series forward modelling of the TEC distribution using IONEX files.

Meanwhile, many generations of the International Geomagnetic Reference Field \citep[IGRF][]{Finlay.10} have continually improved the model of the Earth's magnetic field. This model is composed by spatial interpolation of magnetic field measurements (in up to 13th-order spherical harmonic coefficients) reported by institutions around the world.

Combining these two measurements -- IONEX and IGRF data -- can provide a map of RM distribution above any given position on Earth to moderate precision (better in the Northern Hemisphere than the Southern one, based on the number of GPS beacons in each). \cite{Afaraimovich.08} offered the first such software implementation, with the objective of using it to track Solar Activity\footnote{\cite{Erickson.01} were the first to present software capable of calculating ionospheric RMs using the IGRF, but they used local GPS beacons instead of IONEX files}. \cite{Sotomayor-Beltran.13} introduced the {\tt ionFR} package, which calculated ionospheric RMs towards a given position on the sky. We generalized their approach for the wide-field measurements in our {\tt radionopy} software package, which we present in Section~\ref{sec:widefieldRMionosphere}.

\section{Low frequency observations: discoveries and challenges}
\label{sec:lowfreqionosphere}

Low frequency interferometric observations are effected in two main ways by ionospheric turbulence: scintillation in Stokes I observations, and Faraday Rotation in Stokes Q and U observations. 

TEC variations introduce a variable index of refraction across a field of view. Stokes I signal from a point source will scintillate, change position, by an amount \citep[e.g.][]{TMS}:

\begin{equation}
\Delta\theta = - \frac{1}{8\pi^2} \frac{e^2}{\epsilon_0 m_e}\frac{1}{\nu^2} \nabla_{\perp}({\rm TEC})
\label{eq:ionosphere_scintillation}
\end{equation}

at the observed frequency $\nu$, where $\nabla_{\perp}$ is the transverse gradient in TEC towards the direction of the source. The time, space and frequency dependence of this effect causes difficulty for long integrations, since the scintillation will cause averaging of point sources with empty space, spreading-out their signal over a $\sim\Delta^2\theta$ area. This can be interpreted as an additional source of noise in a Stokes I map. \cite{Vedantham.15} showed that this scintillation noise can be much larger than image noise for baselines longer than $\sim 200$\,m. \cite{Vedantham.16}, extending the previous analysis to the Fourier domain, showed that this noise does not pose large issues to HERA or SKA-Low EoR efforts, since realistic amounts scintillation were not sufficient to wash-out EoR signals on large scales (their dense cores of relatively short baselines also help). However, it could pose large issues for point-source calibration and subtraction methods -- as emphasized in a public SKA memo by \cite{Cornwell.16}.

\cite{Loi.15} used MWA observation snapshots to map the scintillation as a function of space and time, resulting in the discovery of ``tubes'' of plasma density waves across the Southern Hemisphere in lines of roughly constant latitude. Comparing the sources in their snapshot images to source positions in the NRAO VLA Sky Survey \citep[NVSS][]{Condon.98} they were able to calculate displacement vectors, and showed that they were strongly aligned to Earth's magnetic field.

The literature surrounding ionospheric Faraday Rotation is less extensive than work focussing on the unpolarized component. \cite{Lenc.16} showed that MWA measurements of diffuse foregrounds could provide a map of ionospheric spatiotemporal variance as their RM changed throughout a series of observations. \cite{Lenc.17} showed that point source power could be seen ``twinkling'' in and out of polarized intensity maps due to ionospheric activity.

\section{Relevance for PAPER and HERA EoR measurements}
\label{sec:widefieldRMionosphere}

Within the PAPER and HERA power spectrum pipelines, many tens to hundreds of days of visibilities are averaged over during binning in LST. The the ionosphere-induced spatial and temporal fluctuations in RM could produce sufficient phase scrambling of the celestial Faraday-rotated, polarized signal to suppress a fraction of any polarized signal leaked by some mechanism into Stokes I measurements. The fringe size of the 30\,m baselines used in power spectrum analyses is large enough that scintillation effects are negligible.

This effect was first investigated in \cite{Moore.17}. Using the {\tt ionFR} package \citep{Sotomayor-Beltran.13} we calculated the RM distribution at a single zenithal pointing throughout the PAPER-32 observation season. This was a vast simplification given the PAPER primary beam was much larger than a typical isoplanatic patch. Shown in Figure~\ref{fig:ionosphere_psa32hist}, there was a large spread of ionospheric RMs for each LST. There was a decrease in the average magnitude of the RM as LST increased. This was expected, given the strong correlation between the day/night cycle and TEC values \citep[e.g.][]{Tariku.15}, and given that for this observing season, LST=4 hr corresponded to observations taken shortly after sunset, while LST=8 hr was always well into the night.

\begin{figure}
\centering
\includegraphics[width=0.9\textwidth]{chapters/ionosphere/figures/MooreHist.png}
\caption[Distribution of zenithal ionospheric RMs for 3 LSTs in the PAPER-32 observing season]{Distribution of zenithal ionospheric RMs for 3 LSTs in the PAPER-32 observing season. From top to bottom: a histogram of the zenith ionospheric RMs over the season, for the transit of LSTs 4, 6, and 8 hr. Taken from \cite{Moore.17}.}
\label{fig:ionosphere_psa32hist}
\end{figure}

Treating the single pointing as constant over the sky, we calculated the expected attenuation of polarized signal, leaked into pseudo-Stokes I visibilities, that would be averaged over varying ionospheric conditions during LST binning. These attenuation factors were 43$\pm$6\% at 165\,MHz and 7$\pm$5\% at 126\,MHz.

To build on this result, we required more sophisticated simulations of the interaction of the polarized sky with the instrument and whole-sky maps of the ionosphere. To accomplish the latter, we developed the open-source Python package {\tt radionopy}\footnote{\url{https://github.com/UPennEoR/radionopy}}. Like {\tt ionFR}, {\tt radionopy} uses GPS-derived TEC maps from IONEX files and the IGRF to estimate the value of ionospheric RM at a given latitude, longitude and date. Unlike its predecessor, {\tt radionopy} does not necessarily calculate an RM at a given pointing, but instead is capable of calculating the ionospheric RM over a {\tt HEALPix} grid of the sky \citep{healpix}. Such an expression of ionospheric variation is natural to wide-field, drift-scanning EoR arrays, and reflects the format of the IONEX input measurements, which are given in their spherical harmonic decompositions. {\tt radionopy} is vectorized, leading to efficient generation of full-sky ionospheric maps, and object-oriented, allowing for easier collaborative development. Additionally we implemented the interpolation scheme recommended in the IONEX documentation to obtain `best-guess' full-sky maps for arbitrary times between the 2-hour time resolution of IONEX data. 

An example output from {\tt radionopy} is shown in Figure~\ref{fig:ionosphere_radionopy_example} as a {\sc HEALPix} grid of the hemisphere observable from the PAPER site in the Karoo. In Figure~\ref{fig:ionfr_compare} we show {\tt radionopy} and {\tt ionFR} output for a single pointing towards Cassiopeia A (Cas A; RA=23$^{\rm h}$23$^{\rm m}$27.9$^{\rm s}$, Dec=$+58^{\circ}48'42.4''$) from the LOFAR Core site in the Netherlands. The two codes gave qualitative agreement. Slight offsets at the highest RM values that day could be attributed to differences in our interpolation schemes.

\begin{figure}
\centering
\includegraphics[width=0.9\textwidth]{chapters/ionosphere/figures/widefield_RM_snap.pdf}
\caption{An example of widefield ionospheric RMs calculated by {\tt radionopy}.}
\label{fig:ionosphere_radionopy_example}
\end{figure}

\begin{figure}
\centering
\includegraphics[width=0.9\textwidth]{chapters/ionosphere/figures/ionFRcompare.png}
\caption[The RM of Cas A as viewed from the LOFAR Core site in the Netherlands on April 11th, 2011, according to {\tt ionFR} and {\tt radionopy}.]{The RM of Cas A as viewed from the LOFAR Core site in the Netherlands on April 11th, 2011, according to {\tt ionFR} and {\tt radionopy}. The two codes show quantitative agreement, demonstrating that radionopy can be used for single-pointing as well as full-sky RM measurements.}
\label{fig:ionfr_compare}
\end{figure}

{\color{red}Martinot et al. (in prep.)} investigated the full interaction of the polarized sky with the ionosphere, using realistic polarized sky models and fully-polarized HERA beam models (see Chapter~\ref{chapter:eor_window_HERA} for an example). Their work revealed that the \cite{Moore.17} analysis overestimated the ionospheric attenuation due to their single-pointing and simple beam models. Realistic levels of attenuation for a 100 day HERA integration can be expected to reach a factor of $\leqslant 0.1$. If polarization leakage occurs close to the EoR level, this is sufficient to recover the EoR power spectrum. However, if it is above the EoR level (as expected by \citealt{Nunhokee.17}), the ionosphere alone will not be sufficient to rule out polarization leakage being detected before the EoR can be recovered.

% 
% bridge to EoR window studies?
%
\chapter{A view of the EoR window from the PAPER-32 imaging array}
\label{chapter:eor_window_paper32img}

In this Section, we present 2D power spectra created from data taken by the PAPER-32 imaging array in Stokes I, Q, U and V. 

The PAPER 32-antenna array relied on its highly redundant configuration in order to take the measurements resulting in the strong upper limits on the 21 cm power spectrum \citep{Parsons.14, Jacobs.15, Moore.17}. However, for three nights in 2011 September, the 32 elements were reconfigured into an polarized imaging configuration. 

Power spectra allowed us to observe and diagnose systematic effects in our calibration at high signal-to-noise within the Fourier space most relevant to EoR experiments. We observed well-defined windows in the Stokes visibilities, with Stokes Q, U and V power spectra sharing a similar wedge shape to that seen in Stokes I.  With modest polarization calibration, we saw no evidence that polarization calibration errors moved power outside the wedge in any Stokes visibility, to the noise levels attained.  Deeper integrations will be required to confirm that this behavior persists to the depth required for EoR detection.

The layout of this Chapter is as follows.  In Section~\ref{sec:psa32_obs} we provide a brief description of the PAPER array in its imaging configuration, the data from which this paper is based, and describe its calibration and reduction. We also describe the method used to create 2D power spectra in this section. We analyze the power spectra in Section~\ref{sec:psa32_res}, and discuss the implications of our findings and conclude in Section~\ref{sec:psa32_disc}.

\section{Observations \& Reduction}
\label{sec:psa32_obs}
We present measurements taken overnight on 2011 September 14--15 over local sidereal times (LSTs) 0--5 hr.

\begin{figure}[h!]
\centering
\includegraphics[width=0.9\columnwidth]{chapters/eor_window_PAPER/figures/new_antenna_config.pdf}
\includegraphics[width=0.9\columnwidth]{chapters/eor_window_PAPER/figures/uv_coverage_exclbad_overlaid-compressed.png}
\caption[The PAPER-32, dual-pol antenna imaging configuration and \textit{uv} distribution.]{The PAPER-32, dual-pol antenna imaging configuration (top). They were arranged in a pseudo-random scatter within in a $\sim$300\,m diameter circle to maximize instantaneous $uv$ coverage (bottom). $uv$ coverage is shown for 100--200\,MHz over 203 channels in blue, and 146--166\,MHz over 20 channels in red (the latter being the frequencies used in our power spectrum analysis). Malfunctioning antennae identified during calibration are overlaid with red crosses (and are excluded from the $uv$ coverage map).\\}
\label{fig:psa32img_config}
\end{figure}

Antennae were arranged in a pseudo-random scatter within a 300\,m-diameter circle, the layout of which is shown in Figure~\ref{fig:psa32img_config}. This allowed us to obtain resolutions between 15' and 25' across the bandwidth (100--200 MHz nominally, although in reality this extends 110--185 MHz due to band edge effects and VHF TV). Drift-scan visibilities were measured every 10.7 s, and divided into datasets about 10 minutes in length. We express an interferometric visibility $V^{pq}_{ij}$ between antennae $i$ (with dipole arm $p$, which can be $x$ (East-West) or $y$ (North-South) for PAPER dipoles), and $j$ (with dipole arm $q$), in directional cosines $l$ and $m$ for frequency $\nu$ at time $t$, as:

\begin{equation}
\begin{aligned}
V_{ij}^{pq}(\nu,t) = g^p_i g^{q*}_j \exp(-2\pi i \nu \tau_{pq}) \times \\ \int d\Omega \, A^{pq}(\Omega, \nu)
S(\Omega, \nu) \exp\left(\frac{ -i\nu}{c} \vec{b}(t) \cdot \hat{s}(\Omega)\right)
\end{aligned}
\label{eq:psa32_visibility}
\end{equation}

where the $g$ terms represent the complex gains for each antenna and dipole arm, $A^{pq}$ is the polarized beam and $S$ is the sky. The product $\vec{b}(t) \cdot \hat{s}(\Omega)$ represents the projection of the baseline between $i$ and $j$ with respect to an arbitrary location on the sky. 
The motivation for including the term for the delay between dipole arms $p$ and $q$, $\tau_{pq}$, is given in Section \ref{subsubsec:psa32_polcal}. This delay is clearly zero if $p=q$.

Visibilities were obtained from correlating both $x$ and $y$ dipoles, forming V$^{xx}$, V$^{xy}$, V$^{yx}$ and V$^{yy}$. Frequencies from 100 to 200\,MHz were sampled into 2048 channels.
Data were delay-filtered to 203 frequency channels \citep[see the Appendix of][]{Parsons.14} and Chapter~\ref{chapter:data_prep_and_proc}.  Cross-talk was modelled and removed by subtracting the average power over the 5 hours of observation, which extended across LST=0h--5h. An initial RFI-flagging removed any outliers more than $6\sigma$ from a spectrally smooth profile.

\subsection{Calibration}

Calibration took place in three stages, detailed below: a first-order delay-space calibration for the initial gains and phases with respect to Pictor A, an absolute calibration using imaging with respect to Pictor A and Fornax A, and a polarimetric correction for the $\tau_{xy}$ phase term in the V$^{xy}$ and V$^{yx}$ visibilities. 
Traditional polarimetric calibration proceeds by observing a source with a known polarization angle, and solving for up to seven direction-independent terms in the Jones matrix \citep[e.g.][]{TMS, HBS.1.96}, as well correcting for the effects of the primary beam. 
Given the dearth of suitable calibrators at our observing frequencies, especially at the relatively low resolution and sensitivity of the array, we proceeded with polarized calibration using different techniques, as described in Section~\ref{subsubsec:psa32_polcal} below.

\subsubsection{Initial calibration}
A first-order gain and phase calibration was performed by a similar approach to \citet{Jacobs.13}. Each 10~minute drift-scan dataset was phased to the known position of Pictor A using {\tt aipy} routines.

The gain term in Equation \ref{eq:visibility} was approximated as 
\begin{equation}
 g^p_i = G^p_i \exp(-2\pi i \nu \tau_{ip})
\end{equation}
and the required delay $\tau_{ip}$ offset of the uncalibrated delay tracks to the real position on the sky solved for to obtain a phase calibration; the absolute flux calibration $G^p_i$ was found by isolating the tracks of Pictor A in delay space, and applying the required flux scale across the band \citep[for a discussion of delay-space calibration, see][and Figure~\ref{fig:delay_spectra}]{Parsons.12a}.

\subsubsection{Absolute Calibration}
\label{subsubsec:psa32_abscal}

Visibilities were converted to {\tt CASA} Measurement Sets to be further calibrated using a custom pipeline developed around {\tt CASA} libraries. Snapshot images were generated for each 10~minute observation by Fourier transforming the visibilities. 
We used uniform weights and the multi-frequency synthesis algorithm to further improve the $uv$~coverage. 
Dirty images were deconvolved down to a 5~Jy threshold using the Cotton-Schwab algorithm. 
The sky model generated by the CLEAN components was used to self-calibrate each snapshot over the full bandwidth, using a frequency-independent sky-model and averaging over the 10~minute observation.
We corrected for residual cable length errors by computing antenna-based phase solutions for each frequency channel for each snapshot observation. 
After self-calibration, snapshot visibilities were again Fourier transformed into images and deconvolved down to a 2~Jy threshold to form the final sky models. 
These final sky models were used to solve for a frequency independent, diagonal, complex Jones matrix \citep{HBS.1.96,Smirnov.11} for each antenna in order to calibrate gain variations from snapshot to snapshot. 
We make no attempt to correct sky models for polarized primary beams and, therefore, our gain solutions incorporate both the direction independent and the direction dependent responses of the two gain polarizations. 
This is a reasonable approximation for the scope of the paper, as, eventually, wide-field polarization corrections cannot be implemented directly in the per-baseline 
power spectrum estimation (see Section~\ref{sec:psa32_res}).

The average correction in magnitude through this second-order calibration was a $\pm$6\% change for $x$ gains and $\pm$7\% for $y$ gains from those derived in the initial delay-space calibration.
If the gain on an antenna deviated by more than 30\% from image-to-image during this analysis, it was discarded from future processing stages, since it was likely malfunctioning. This was true for 3 antennae (see the top panel of Figure~\ref{fig:psa32img_config}). 

\begin{figure*}[h!]
\centering
\includegraphics[width=0.4\textwidth]{chapters/eor_window_PAPER/figures/zen_2455819_50285_orig-I-image.pdf}
\includegraphics[width=0.4\textwidth]{chapters/eor_window_PAPER/figures/zen_2455819_50285_orig-Q-image.pdf}
\includegraphics[width=0.4\textwidth]{chapters/eor_window_PAPER/figures/zen_2455819_50285_orig-U-image.pdf}
\includegraphics[width=0.4\textwidth]{chapters/eor_window_PAPER/figures/zen_2455819_50285_orig-V-image.pdf}
\noindent\rule{14cm}{0.6pt}
\includegraphics[width=0.4\textwidth]{chapters/eor_window_PAPER/figures/zen_2455819_50285-I-image.pdf}
\includegraphics[width=0.4\textwidth]{chapters/eor_window_PAPER/figures/zen_2455819_50285-Q-image.pdf}
\includegraphics[width=0.4\textwidth]{chapters/eor_window_PAPER/figures/zen_2455819_50285-U-image.pdf}
\includegraphics[width=0.4\textwidth]{chapters/eor_window_PAPER/figures/zen_2455819_50285-V-image.pdf}
\caption[Snapshot images of Stokes parameters before and absolute calibration.]{\textit{Above:} Example of a Stokes I snapshot image (top left) with corresponding Stokes Q (top right), Stokes U (bottom left) and Stokes V (bottom right) images before absolute calibration. \textit{Below:} The same organization as above, after absolute calibration. No primary beam correction was applied. The Stokes I image was deconvolved down to 5~Jy~beam$^{-1}$ whereas the other images were not deconvolved. We note that the Stokes Q image is relatively featureless apart from a few faint sources that appear instrumentally polarized. Stokes U and Stokes V images are, instead, dominated by Fornax A that shows instrumental polarization leaked from total intensity. Units are Jy~beam$^{-1}$; note the change in scale between polarizations and calibration stages.}
\label{fig:psa32_sky_image}
\end{figure*}

The final gain amplitude calibration was carried out similarly to \citet{Ali.15}. We generated single channel images between 120 and 174 MHz for each snapshot and deconvolved each of them down to 10 Jy. For each snapshot, a source spectrum is derived for Pictor A by fitting a two dimensional Gaussian the source using the {\tt PyBDSM}\footnote{\url{http://www.lofar.org/wiki/doku.php?id=public:user software:pybdsm}} source extractor \citep{pybdsm}. Spectra were optimally averaged together by weighting them with the primary beam model evaluated in the direction of Pictor A. To fit the absolute calibration, we divided the model spectrum \citep{Jacobs.13} by the measured one and fit a 6th order polynomial over the 120-174 MHz frequency range. This procedure was repeated using Fornax A with the only difference that a taper was applied to the visibilities (120\,m) in order to reduce Fornax A to a point-like source and use the model spectrum from \citet{Bernardi.13}. The best fit coefficients for Pictor A and Fornax A were averaged together to obtain the final absolute flux density calibration. Snapshots of fully CASA-calibrated data are shown in Figure~\ref{fig:psa32_sky_image}.

\subsubsection{Polarimetric factors}
\label{subsubsec:psa32_polcal}
Standard full polarization calibration involves correcting for leakage of Stokes~$I$ into the $V_{ij}^{xy}$ and $V_{ij}^{yx}$ visibilities and leakage of polarized signal into total intensity (the so called Jones $D$ matrices or $D$-terms; e.g. \citet[][]{TMS, HBS.1.96}), and an unknown phase difference between the $x$ and $y$ feeds \citep[e.g.][]{Sault.96}. 

We attempt no $D$ matrix calibration in this paper, as there is not a dominant source to be used for such calibration: the limited sensitivity of our observations does not offer good signal-to-noise ratio on PMN~J0351-2744, the only  polarized source at low frequencies known so far in our survey area.  In addition, $D$-term calibration would require determination of the primary beam Mueller matrices beyond our current accuracy. The consequences of this limitation are discussed in the analysis of our power spectra in Section~\ref{sec:psa32_res}.  

As a intermediate measure compatible with these limitations, we therefore adopted a minimization of the phase difference between the $V_{ij}^{xy}$ and $V_{ij}^{yx}$ visibilities, minimizing a sum of squared weighted residuals $w$:

\begin{equation}
w(\nu,\,t,\,\tau_{xy}) = \sum_{ij} | V_{ij}^{xy} - V_{ij}^{yx}\exp(-2\pi i \nu \tau_{xy}) |^2
\end{equation}
to find an estimated value of $\tau_{xy}$ for the array at each $(\nu,\,t)$ sample. This is equivalent to assuming that the sky is intrinsically not circularly polarized at the frequencies observed by PAPER.

We choose not to correct for ionospheric Faraday rotation in our calibration. Not only is this difficult to do for widefield instruments, but also the ionosphere was relatively stable during the observations, so we expect little incoherent averaging during the power spectrum stage below. We calculated the stability of ionospheric RM ($\phi_{\rm iono}$) using the {\sc ionFR} software \citep{Sotomayor-Beltran.13}, which calculates the $\phi_{\rm iono}$ for a given longitude, latitude and time by interpolating values of GPS-derived total electron content maps and the International Geomagnetic Reference Field \citep{Finlay.10}. The values of $\phi_{\rm iono}$ for different lines of sight are shown in Figure~\ref{fig:psa32img_ionosphere}. 
Fluctuations of $\phi_{\rm iono}$ will cause incoherent time-averaging and subsequent loss of polarized signal. Using the formalism of \citet{Moore.15} to calculate the attenuation factor, we found that none of the lines of sight (except for the 21h,0$^{\circ}$ one which goes beneath the horizon) shown are responsible for attenuating signal by $>\,20\%$ in power-spectrum space (see Section~\ref{subsec:psa32_create_pspec}).

\begin{figure}
\centering
\includegraphics[scale=0.45]{chapters/eor_window_PAPER/figures/ionosphere4casacal_morelines.pdf}
\caption[The values of ionospheric RM for different lines of sight a range of LSTs.]{The values of ionospheric RM for different lines of sight the range of LSTs in this analysis, as calculated by {\sc ionFR} \protect\citep{Sotomayor-Beltran.13}. The 21h,0$^{\circ}$ line of sight goes beneath the horizon after LST=3h, and therefore has fewer data points.\\ }%As an LoS gets closer to the horizon, the value of $\phi_{\rm iono}$ becomes more uncertain as projection effects become larger.}
\label{fig:psa32img_ionosphere}
\end{figure}


We form linear combinations of the instrument visibilities, the so-called pseudo-Stokes visibilities \citep[see e.g.][]{Moore.13} $V^I,\,V^Q,\,V^U$ and $V^V$ as:

\begin{equation}
\left(\begin{array}{c}
V^{I}\\
V^{Q}\\
V^{U}\\
V^{V}\end{array} \right)
=
\left( \begin{array}{cccc}
1 & 0 & 0 & 1 \\
1 & 0 & 0 & -1 \\
0 & 1 & 1 & 0 \\
0 & -i & i & 0 \end{array} \right) 
\left(\begin{array}{c}
V^{xx}\\
V^{xy}\\
V^{yx}\\
V^{yy}\end{array} \right) 
\label{eq:psa32_stokes}
\end{equation}

Data that were reduced, calibrated, and formed into Stokes visibilities were separated into delay spectra inside and outside of the horizon for each baseline. We used a 50\,ns margin for what was considered `inside' the horizon, in order to confine all supra-horizon emission \citep[e.g.][]{Parsons.12a, Pober.13} to the foreground component of the data. We implemented a one-dimensional CLEAN \citep{ParsonsBacker.09, Parsons.12b} with a Blackman-Harris window to a tolerance of $10^{-9}$. RFI is more easily identified in foreground-removed data, so we RFI-flagged again on the background data deviations greater than $3\sigma$. We then added the inside- and outside-horizon visibilities back together; RFI flags were preserved in the process. 

The effect of our calibration is shown in the delay-transformed visibilities in Figure~\ref{fig:psa32_delay_spectra}. As is apparent in Figure~\ref{fig:psa32_sky_image}, after improved calibration there are fewer delay tracks (i.e. sources) in the Stokes Q visibilities, while there is little overall change in Stokes U. The minimization of Stokes V, performed after the imaging calibration stage, moves power from Stokes V into Stokes U, effectively accounting for part of a $D$-term correction. But without an accurate $D$-term calibrator, Stokes U exhibits additional (and dominant) $D$-term leakage from Stokes I, 
in this case due to Pictor A. Pictor A is the brightest source in Stokes I in our observed field, and thus dominates the visibility shown.  There is no reason to suppose that Pictor A is pure Stokes U (compare also Figure \ref{fig:psa32_sky_image}), and thus the bulk of this emission must be leakage.

\begin{figure}[h!]
\centering
\includegraphics[width=\columnwidth]{chapters/eor_window_PAPER/figures/delayfalls_wedgeres.pdf}
\caption[The absolute value of delay-transformed visibilities over the bandwidth (146--166\,MHz) used to create the power spectra shown in this Chapter.]{The absolute value of delay-transformed visibilities over the bandwidth (146--166\,MHz) used to create the power spectra shown in this Chapter. The left and right columns show the visibilities before and after absolute calibration (and for Stokes U and V, the application of the $\tau_{xy}$ parameter), respectively, for baseline formed by antennae 6 and 14 ($\sim 250$\,m in length, approximately East-West). The flux scale in the left column as been boosted for a more fair comparison to the absolute-calibrated data. From top to bottom, the rows correspond to Stokes I, Q, U and V. 
The horizon limit is marked by white dashed lines. 
}
\label{fig:psa32_delay_spectra}
\end{figure}

\subsection{Creating power spectra}
\label{subsec:psa32_create_pspec}
 

Expressing the visibility $V_{ij}^{pq}(\nu,t)$ observed at time $t$ (see Equation~\ref{eq:psa32_visibility}) in terms of the geometrical delay $\tau_g=\vec{b}(t) \cdot \hat{s}(l,m)/c$ for the baseline $ij$, \citet{Parsons.12a} define the delay transform as the Fourier transform of the visibility along the frequency axis:
\begin{equation}
\tilde{V}_{ij}^{pq}(\tau,t) = \int d\nu \, V_{ij}^{pq}(\nu,t) e^{2\pi i \nu \tau}
%\int dl\, dm\, d\nu\, A^{pq}_{ij}(l, m, \nu) S^{pq}_{ij}(l, m, \nu) e^{-2\pi i \nu (\tau_g - \tau)}
\end{equation}

We can represent the power at each frequency and baseline in an array as a power spectrum in terms of their respective Fourier components $k_{\parallel}$ and $k_{\perp}$ as:
\begin{equation}
P(k_{\parallel},k_{\perp}) \approx |\tilde{V}_{ij}^{pq}(\tau,t)|^2 \frac{X^2Y}{\Omega B}\left(\frac{c^2}{2k_B\nu^2}\right)^2
\label{eq:psa32_pspec}
\end{equation}
where $B$ is the bandwidth, $\Omega$ is the angular area (i.e. proportional to the beam area), and X and Y are redshift-dependent scalars calculated in \citet{Parsons.12b}. 

To form $|\tilde{V}_{ij}^{pq}(\tau,t)|^2$, consecutive integrations were cross-multiplied, phasing the zenith of latter to the former i.e.:
\begin{equation}
|\tilde{V}_{ij}^{pq}(\tau,t)|^2 \approx |{V}_{ij}^{pq}(\tau,t)\times {V}_{ij}^{pq}(\tau,t+\Delta t)e^{i\theta_{ij,{\rm zen}}(\Delta t)}|^2
\end{equation}
where $\Delta t$=10.7 seconds and $\theta_{ij,{\rm zen}}(\Delta t)$ is the appropriate zenith rephasing factor. This method should avoid noise-biased power spectra except on very long baselines, which the PAPER configuration does not contain, while sampling essentially identical $k$-modes.  Note that this is the same method used by \citet{Pober.13} in their investigation of the unpolarized wedge.


%
%
%
\chapter{A view of the EoR window from the HERA-19 commissioning array}
\label{chapter:eor_window_HERA}

As emphasized in Chapter~\ref{chapter:eor_window_paper32img}, it is important to constrain intrinsic and leaked polarized signal for any {\sc hi} EoR experiment. 
The objective of this Chapter was an exploration of eight nights of data from the Hydrogen Epoch of Reionization Array (HERA) 19-element commissioning array, coupled with simulations of the instrument, in order to forecast how much of a problem polarization would pose for this interferometer. 
This work also represents the first power spectral analysis from HERA. While not in the realm of an EoR-level integration, we were able to offer some initial expectations for this new instrument's performance in the Fourier domain.

This work is organized as follows: in Section~\ref{sec:hera19_leak} we review the theory behind polarization leakage into unpolarized signal and simulate the effect for a model of HERA. In Section~\ref{sec:hera19_obs} we describe the HERA data that we used, its calibration and reduction to power spectra. We present our results, and discuss the implications for HERA's EoR measurements, in Section~\ref{sec:hera19_results}, and conclude in Section~\ref{sec:hera19_conc}. We assume the cosmological parameters reported by \cite{Planck.16} throughout.

\section{Polarization Leakage Simulations}
\label{sec:hera19_leak}

In Chapter~\ref{chapter:interferometry}, we presented direction dependent and independent ways for polarized power to ``leak'' between visibilities. Direction dependent leakage arises because dipole arm `n' is sensitive to electromagnetic radiation with polarization axis aligned with perpendicular dipole arm `e'. Forming pseudo-Stokes visibilities from those in the instrumental basis,

\begin{equation}
\left(\begin{array}{c}
V^{I}\\
V^{Q}\\
V^{U}\\
V^{V}\end{array} \right)
= \frac{1}{2}
\left( \begin{array}{cccc}
1 & 0 & 0 & 1 \\
1 & 0 & 0 & -1 \\
0 & 1 & 1 & 0 \\
0 & -i & i & 0 \end{array} \right) 
\left(\begin{array}{c}
V^{nn}\\
V^{ne}\\
V^{en}\\
V^{ee}\end{array} \right) .
\label{eq:pseudo-stokes}
\end{equation}
results in each pseudo-Stokes visibility containing a direction-dependent mix of the `true' Stokes parameters on the sky. Using simulations of the HERA feed, faceted parabolic dish and analog signal chain \citep{Fagnoni.16}, we proceeded to simulate pseudo-Stokes visibilities using the polarized formalism in Chapter~\ref{chapter:interferometry} (Figure~\ref{fig:interferometry_Mab} shows the direction-dependent leakage matrices used here). 
We simulated visibilities for the HERA-19 commissioning array, described below, using an \textit{unpolarized} model of the low frequency sky from the Global Sky Model \citep[GSM;]{GSM.08, pygsm, GSM.17} at the appropriate R.A. range to match our observations. Forming power spectra and images from these visibilities allowed for a comparison of our data to a `leakage from Stokes I only' regime. At the low frequencies and large scales probed by HERA, Stokes I is extremely bright compared to the other Stokes parameters (e.g. Chapter~\ref{chapter:astro_rad}, \citet{Kohn.16}), so this regime is realistic for the measurements in question.

In Chapter~\ref{chapter:interferometry}, we also presented a formalism for propagating direction-independent calibration errors into polarization leakage. We did not include calibration errors in our simulations, allowing us to build intuition around power spectrum estimates for a ``perfectly behaving'' instrument.

\section{Observations \& Reduction}
\label{sec:hera19_obs}

In this work we used eight nights of observations from the HERA-19 commissioning array. HERA is a low-frequency interferometer composed of 14\,m-diameter dishes arranged in a close-packed hexagonal array of 14.7\,m spacing. The commissioning array consists of nineteen dishes (see Figure~\ref{fig:hera19_antpos}); HERA is being constructed in staged build-outs, and upon completion will consist of 350 dishes in a fractured hexagon configuration \citep[see][]{Dillon.16, deBoer.17}. A feed cage containing two dipole feeds (recycled from the PAPER array, see \citealt{Parsons.10}), oriented in North-South and East-West directions, is suspended above each dish \citep{Neben.16,Ewall-Wice.16.HERA_Dish,Thyagarajan.16}.

HERA only observes in drift-scan mode. The observations we used were eight nights, from Julian Date (JD) 2457548 to 2457555; LSTs 10.5 -- 23 hr. Drift-scan visibilities were recorded every 10.7 seconds for 1024 evenly-spaced channels across the 100-200\,MHz bandwidth. These data were divided into {\sc miriad} data sets roughly 10 minutes long. A night's observation lasted 12 hours in total (6pm to 6am South African Standard Time; SAST); of these we used the central 10 hours, to avoid the thermal effects of the Sun.

\begin{figure}
\centering
\hspace{-0.5cm}\includegraphics[scale=0.6]{chapters/eor_window_HERA/figures/antpos_hera19.pdf}
\caption[The perimeter of each dish in the HERA-19 array.]{The perimeter of each dish in the HERA-19 array.  A red ``X'' marks antennae that were identified during preprocessing and calibration as malfunctioning and were excluded from further analysis.}
\label{fig:hera19_antpos}
\end{figure}

To identify samples contaminated by radio frequency interference (RFI), a two-dimensional median filter in time and frequency was applied to the visibility data to smooth out high pixel-to-pixel variations, and remove significant outliers that were likely unphysical. The variance of the resulting data was computed, and points with a $z$-score greater than 6 (i.e., points where the value is more than 6$\sigma$ away from the mean) were flagged as initial seeds for RFI extraction. A two-dimensional watershed algorithm was applied using these seeds as starting points, enlarging the regions of RFI-contamination to neighboring pixels with z-scores greater than 2, until all such pixels were flagged. Figure~\ref{fig:hera19_rfi} shows the fractional RFI flag occupancy per time (displayed in LST) and frequency across the 8 days of observations. The majority of the band is relatively clear of RFI. Some clear features are: the FM radio band (below 110 MHz), ORBCOMM satellite communications (137 MHz), an ISS downlink (150 MHz) and VHF TV channels (above 170 MHz)\footnote{For an extended discussion of RFI as seen by HERA, see the public HERA Memo \# 19}.
The Galaxy, when transiting zenith at LST$\approx$17.75 hours, is so bright that it appears to degrade our ability to flag RFI.

\begin{figure}
\centering
\includegraphics[scale=0.6]{chapters/eor_window_HERA/figures/frac_occ.pdf}
\caption{Fractional RFI flag occupancy per time and frequency over the eight days of observations.}
\label{fig:hera19_rfi}
\end{figure}

\subsection{Calibration}
\label{subsec:hera19_cal}

HERA is designed to be calibrated using redundant calibration techniques \citep{Dillon.16}, but for this preliminary view of HERA commissioning data, we used image-based calibration. Future studies with deeper integrations targeting EoR detections will take advantage of redundancy to obtain more precise calibration solutions \citep{deBoer.17}. We used the {\sc CASA} \citep{casa} package for calibration, taking advantage of its CLEAN, {\tt gaincal} and {\tt bandpass} functions.

To enable the use of {\sc CASA}, we first converted from native {\sc miriad} to a {\sc uvfits} file format which could be ingested by {\sc CASA} using {\sc pyuvdata} \citep{pyuvdata}. 
Using LSTs in which the Galactic center (GC; $\alpha, \delta$ = 17h 45m 40.04s,​ ​-29d 0m 28.12s) was transiting, we built a CLEAN model which modeled the GC as an unpolarized point source of strength 1 Jy and flat spectrum, which could be scaled appropriately later (see Equation~\ref{eq:freq_scale}). 
Clearly, this is an incomplete calibration model. However, as the objective of this work is to explore the response of the instrument in power spectrum space without combining baselines of different lengths, most of the purpose of the calibration is correcting an initial large cable delay per antenna. 
Treating the GC as unpolarized is adequate for this study. The large optical depth towards the GC \citep{Oppermann.12} results in large amounts depolarization in the plane of the Galaxy \citep{Wolleben.06}. Moreover, we expected non-negligible amounts of beam depolarization due to the large solid angle of the synthesized beam.

For each night of observations, we used the {\sc CASA} {\tt gaincal} and {\tt bandpass} functions to obtain frequency-dependent phase and amplitude solutions for each antenna and dipole arm. Four antennae had very deviant solutions, and their inclusion resulted in low-quality images. These were omitted from further analysis (and are marked with red ``X''s in Figure~\ref{fig:hera19_antpos}).  Before calibration, we manually flagged the edges of the band (below 110 MHz and above 190 MHz), where spectral behavior is dominated by the high and low pass filtering in the HERA signal chain \citep{deBoer.17}.

In Figure~\ref{fig:hera19_GCimage}, we show images formed from the simulated pseudo-Stokes visibilities (top panels) and our observations (bottom panels). These are multi-frequency synthesis images, where we used all unflagged frequencies on either side of the band edges; 115 MHz to 188 MHz. We do not specify a beam model during imaging. At HERA's position ((latitude, longitude) = (-30:43:17.5, 21:25:41.9)) the Galactic Center transits 2$^{\circ}$ from zenith, while the HERA primary beam has a FWHM of $\sim5^{\circ}$ at 150\,MHz \citep{Neben.16}. For the simulated visibilities, we flagged the same antennae as in the data. As expected for a compact array, the Stokes I images capture only a low-resolution view of the Galactic Center. The simulated and observed visibilities form remarkably similar images in Stokes I, Q and U, but the simulation under-predicts pseudo-Stokes V power. We defer further discussion to Section~\ref{sec:hera19_results}.

\begin{figure*}
\centering
\includegraphics[width=0.7\textwidth]{chapters/eor_window_HERA/figures/sim4pol.pdf}
\par\noindent\rule{0.8\textwidth}{0.4pt}
\includegraphics[width=0.7\textwidth]{chapters/eor_window_HERA/figures/new4pol.pdf}
\caption[Multi-frequency synthesis pseudo-Stokes images formed from simulation and data.]{
\textit{Above}: Multi-frequency synthesis pseudo-Stokes images formed from simulation, where only a Stokes I sky was used; any polarized power is due to direction-dependent polarization leakage (see Section~\ref{sec:hera19_leak}).
\textit{Below}: Multi-frequency synthesis pseudo-Stokes images formed from observed visibilities on JD 2457548.
Both sets of panels show the Galactic Center (our calibrator source) close to transit in pseudo-Stokes I, Q, U and V visibilities (\textit{top left, top right, lower left, lower right}). A Briggs-weighting with robustness 0 was used when gridding into the image plane. No deconvolution was performed. The colorbar is in units of Jy/Beam.
A separate color scale is used for Stokes I for suitable dynamic range. An R.A., Dec. grid is shown, illustrating the wide-field nature of HERA observations.
}
%The color scale is normalized to the peak flux of the Galactic Center in Stokes I.} %The excess at the center of the Stokes Q image suggests a $\sim 1\%$ error in the gain solutions, while the excess in Stokes V suggests D-terms at the $\sim 1\%$ level \citep{TMS}.}
\label{fig:hera19_GCimage}
\end{figure*}

Example bandpass solutions from JD 2457548 are shown in Figure~\ref{fig:hera19_bandpass}. Although some residual RFI remains obvious, the derived bandpasses were smooth.  Thus, even though the gains were imprecise, we expected that using them should not add additional spectral structure.  %Spectrally non-smooth gain errors would have the effect of coupling significant amounts foreground signal into the EoR window; see Section \ref{subsec:pspec} below.
% It's not whether the bandpass is smooth, but if *after correcting for the bandpass*, the errors are unsmooth.

\begin{figure}
\centering
\includegraphics[scale=0.5]{chapters/eor_window_HERA/figures/h19_2457458_abs_smallzoom_nolegend.png}
\caption[Bandpass solutions for the North-South dipole orientation obtained for the functioning antennae in the array on JD 2457548.]{Bandpass solutions for the North-South dipole orientation obtained for the functioning antennae in the array on JD 2457548. Differences in line color and style is merely to distinguish different antennae. Shaded regions indicate the effective sub-bands used for power spectrum analysis.}
\label{fig:hera19_bandpass}
\end{figure}

The complex gain solutions were subsequently applied to the {\sc miriad} files. Figure~\ref{fig:hera19_phasecal} shows the effect of calibration on the visibilities of three nominally redundantly-spaced baselines. Shown in that figure are the phases of three $V^{nn}$ visibilities from 14.7\,m baselines before and after calibration. There were no shared antennae between the visibilities shown. The qualitative agreement is obvious, providing a consistency check on the solutions.

\begin{figure}
\centering
\includegraphics[width=0.9\textwidth]{chapters/eor_window_HERA/figures/phases_pre_post_abscal_h19_cyclic_grey.pdf}
\caption[The effect of calibration on the phases of visibilities from three redundantly-spaced 14.7\,m baselines.]{The effect of calibration on the phases of visibilities from three redundantly-spaced 14.7\,m baselines; \textit{nn} polarization. The color scale is cyclic; black is $\pm\pi/2$ and white is 0 and $\pm\pi$. \textit{Above}: before calibration; \textit{below}: after calibration. A simple sky model was sufficient to enforce redundancy for redundant baselines.}
\label{fig:hera19_phasecal}
\end{figure}

We did not attempt to calibrate \textit{D}-terms in this work.
%This limited our interpretive power, which we discuss in Section~\ref{sec:results}.

We down-selected to two relatively RFI-free 20 MHz sub-bands (Figure~\ref{fig:hera19_rfi}); 115 to 135 MHz and 152 to 172 MHz, henceforth referred to the ``low band'' and the ``high band''. As we discuss in Section~\ref{subsec:hera19_pspec}, these bands were multiplied by a Blackman-Harris window, centered on their central frequencies, before Fourier transforming in order to minimize side-lobes. This windowing lead to an noise-effective bandwidth of 10 MHz, appropriate for EoR analyses since the {\sc hi} signal is to a reasonable approximation coeval over the corresponding redshift range \citep{Furlanetto.06}.

Pseudo-Stokes visibilities were formed from the instrumental polarizations. These visibilities were then scaled to the appropriate amplitude using a model for the GC spectrum 

\begin{equation}
S_{\rm Sgr A^*}(\nu) \approx  3709 {\rm\, Jy} \times (\nu/408 {\rm \,MHz})^{-0.5}
\label{eq:freq_scale}
\end{equation}
drawn from the Global Sky Model \citep[GSM;][]{GSM.08,pygsm,GSM.17}. Note that the GSM is inherently $\sim 5\%$ uncertain at these frequencies. We note that this scaling is heavily resolution dependent; we are treating the Galactic Center as a point source when it is extended in reality. However, Section~\ref{sec:hera19_results} we show that we obtain sensible power levels for the foregrounds and noise, lending confidence to our overall scaling.

\subsection{Forming power spectra}
\label{subsec:hera19_pspec}

Power spectra were formed according to the method used in \cite{Pober.13} and \cite{Kohn.16}, which we briefly review here. All Fourier transforms were windowed using a Blackman-Harris window at the center of the sub-band, which minimized sidelobes. \cite{Parsons.12a} define the \textit{delay transform} as the Fourier transform of a visibility for baseline $ij$ and pseudo-Stokes parameter $P$ along the frequency axis

\begin{equation}
\tilde{V}_{ij}^{P}(\tau, t) = \int {\rm d}\nu \tilde{V}_{ij}^{P}(\nu, t)e^{2\pi i \nu \tau}.
\end{equation}

We note that using a Blackman-Harris window will induce a correlation between consecutive $\tau$ modes. The Fourier transform of the window function in frequency will be sharply peaked in the delay space, and can be ignored to some extent. Hence the self-correlation of $V_{ij}^{P}(\tau, t)$ can be used to define the power spectrum, although the small correlation of different $\tau$ modes could effect the variance of the power spectrum \citep{Parsons.14}.

The power at each delay-mode and baseline can be represented in terms of their respective Fourier components $k_{\parallel}$ and $k_{\perp}$ \citep{Parsons.12a, Nithya.15b}:
\begin{align*}
P(k_{\parallel},k_{\perp}) &\approx | \tilde{V}_{ij}^{P}(\tau) |^2 \frac{X^2 Y}{\Omega B} \left(\frac{c^2}{2k_B\nu^2}\right)^2 , \\\\
k_{\parallel} &= \frac{2\pi \nu_{\rm 21cm} H(z) %H_0 \sqrt{\Omega_m (1+z)^3 + \Omega_k (1+z)^2 + \Omega_{\Lambda}} 
}{c (1+z)^2}\tau, \\\\
k_{\perp} &= \frac{2\pi}{D(z) \lambda} b\\
\end{align*}
for: bandwidth $B$, angular area of the beam $\Omega$, $\nu_{\rm 21cm}\approx$1420 MHz, baseline length $b$, wavelength of observation $\lambda$, Hubble parameter $H(z)$, transverse comoving distance $D(z)$ and redshift-dependent scalars X and Y \citep{Parsons.12b}. Note that the angular area of the beam refers to the diagonal components of the Mueller matrices shown in Figure~XXX. For further discussion of forming polarized power spectra in $k$-space, refer to \cite{Nunhokee.17}.

To avoid a noise-bias when forming the $ |\tilde{V}_{ij}^{P}(\tau, t) |^2$ term, we cross-multiplied consecutive integrations, rephasing the zenith angle of the latter to the former:

\begin{equation}
 | \tilde{V}_{ij}^{P}(\tau, t) |^2 \approx | \tilde{V}_{ij}^{P}(\tau, t) \times \tilde{V}_{ij}^{P}(\tau, t+\Delta t)e^{i\theta_{ij,\rm zen}(\Delta t)}|
\end{equation}
where $\theta_{ij,\rm zen}(\Delta t)$ was the appropriate phasing for baseline $ij$ and $\Delta t = 10.7$ seconds.

Pseudo-stokes power spectra were formed for each pair of integrations, for every baseline. After forming power spectra, baselines of identical lengths were averaged together. Appealing to cosmological isotropy, baselines of the same length but different orientation should be sampling the same cosmological structure. These 2D power spectra were averaged over our 8 days of observations. Note that all averaging was performed after forming power spectra; this incoherent averaging was non-optimal from a signal-to-noise perspective outside the wedge, slightly reducing our sensitivity in the EoR window. However, the intention of this investigation was not a deep integration on noise; we were more interested in the polarized response of the instrument. As such, the power spectra presented in the Section below should be interpreted as approximate.

\section{Results \& Discussion}
\label{sec:hera19_results}

%We formed two-dimensional power spectra (that is, power gridded into the ($k_{\perp}$, $k_{\parallel}$) plane) for each JD and pseudo-Stokes parameter, and averaged those power spectra together. Averaging in the squared power between days is non-optimal in terms of attempting an EoR detection, but we were concerned with foregrounds in $k$-space for this study, and it reduced the variance of foreground power within the ``pitchfork" region.

Power spectra are shown for the high and low bands in Figure~\ref{fig:hera19_pitchforks_highband} and Figure~\ref{fig:hera19_pitchforks_lowband}, respectively, where white dotted lines mark the boundary of the EoR window on the 2D plots.  The same data are presented in middle and lower panels, with the latter overlaid as lines to emphasize common features of the power spectra with respect to baseline length. 

Theoretical noise levels for the high and low bands were between 
$P_{\rm noise}(k)\approx$ 1.7$\times$10$^8$ \,mK$^2$Mpc$^3$h$^{-3}$ and 3.4$\times$10$^9$ \,mK$^2$Mpc$^3$h$^{-3}$ in the high band, and between 2.3$\times$10$^8$ \,mK$^2$Mpc$^3$h$^{-3}$ and 6.1$\times$10$^9$ \,mK$^2$Mpc$^3$h$^{-3}$ in the low band. These estimates used using a temperature model of the sky

\begin{equation}
T_{\rm sky} = 180\,{\rm K} \left(\frac{\nu}{180\,{\rm MHz}}\right)^{-2.55},
\end{equation}
assume receiver temperatures of 300\,K and 600\,K for the high band and low band, respectively \citep[][also see the public HERA Memo \# 16]{deBoer.17}, and were calculated according to the formalism for noise power spectra in \cite{Parsons.12a}, with the inclusion of a baseline-number dependence (to account for different occupancies in each $k_{\perp}$ bin).
These noise power estimates were roughly corroborated by our observations (see Figure~\ref{fig:hera19_highband_cuts_per_day}). We observe excess noise on the shortest baselines (also obvious in the lower panels of Figures~\ref{fig:hera19_pitchforks_highband} and \ref{fig:hera19_pitchforks_lowband}). 

\subsection{General features of the power spectra}
\label{subsec:general_features}
The most striking feature of these power spectra is the degree of foreground isolation achieved in all pseudo-Stokes parameters. In similar studies of 2D polarized power spectra, both PAPER \citep{Kohn.16} and LOFAR \citep{Asad.17} measurements found ``filled'' regions of Fourier space out to the edge of the EoR window (in the delay-spectrum paradigm, this corresponds to the horizon; zenith angle $\pm$90$^{\circ}$), with some supra-horizon leakage \citep{Pober.13} into the EoR window itself. The power spectra in Figures~\ref{fig:hera19_pitchforks_highband} and \ref{fig:hera19_pitchforks_lowband} show no such behavior; all foreground emission appears to be contained within a narrow region around $k_{\parallel}=0$ h/Mpc. This behavior was predicted for an array of HERA-like dishes by \citealt{Nithya.15b} (although that study only concentrated on the Stokes I component). 

Power at horizon delays, as predicted by \cite{Nithya.15b} and \cite{Neben.16}, was not observed. This was likely a resolution effect. To resolve horizon-delay power, one would need to sample many periods of $\tau_h=b/c$, where $b$ is the magnitude of the baseline vector. The maximum length baseline in the HERA-19 array was 58.4\,m, corresponding to a $\sim$5 MHz period: barely sampled by the 10\,MHz windows we use in this study. The lack of horizon power is corroborated by the simulations of the HERA delay response in \cite{Ewall-Wice.16.HERA_Dish} and \cite{Thyagarajan.16}, although those studies used a different windowing function for the delay transform. Their simulations also predict a high degree of foreground isolation: the presence of noise in our data of course meant that we do not realize the 11 dex of isolation that can be achieved in simulation, but the $\sim$8 dex we do see, without any foreground subtraction and a simple calibration, speaks to the power of HERA's future capabilities.

\begin{figure*}[h]
\centering
\includegraphics[scale=0.45]{chapters/eor_window_HERA/figures/timeavg_SIM_high.pdf}
\includegraphics[scale=0.45]{chapters/eor_window_HERA/figures/timeavg_high.pdf}
\includegraphics[scale=0.3]{chapters/eor_window_HERA/figures/timeavg_1d_high.pdf}
\caption[Power spectrum results from the high-band (157--167 MHz).]{Results from the high-band (157--167 MHz). White dotted lines indicate the boundary of the pitchfork and the EoR window. A black dotted line indicates the $k_{\parallel}=0$\,h/Mpc line. \textit{Top}: Simulated power spectra in Stokes I, Q, U and V, following the formalism in Section~\ref{sec:hera19_leak} -- no polarized sky model was used, so power in Stokes Q, U and V was only due to direction-dependent leakage from Stokes I. No instrumental noise was included in the simulation. \textit{Middle}: Eight-day average power spectra from data. \textit{Bottom}: The same data as shown in the middle panel, but with each baseline length overlaid on one another to allow shared features to be more easily identified.}
\label{fig:hera19_pitchforks_highband}
\end{figure*}

\begin{figure*}[h]
\centering
\includegraphics[scale=0.45]{chapters/eor_window_HERA/figures/timeavg_SIM_low.pdf}
\includegraphics[scale=0.45]{chapters/eor_window_HERA/figures/timeavg_low.pdf}
\includegraphics[scale=0.3]{chapters/eor_window_HERA/figures/timeavg_1d_low.pdf}
\caption[Power spectrum results from the low-band (120--130 MHz).]{Results from the low-band (120--130 MHz), arranged in the same format as Figure~\ref{fig:hera19_pitchforks_highband}.}
\label{fig:hera19_pitchforks_lowband}
\end{figure*}

Visible in the observational data, but not in the simulation, is an excess of power at $k_{\parallel}=0.04$\,h/Mpc, corresponding to a delay of 100\,ns, which is independent of baseline length. This suggests that its origins are in the HERA signal chain. There are 15\,m coaxial cables at one stage of the signal chain from the HERA dishes to the correlator\footnote{This stage of the signal chain is only present in the commissioning array. Future HERA build-outs will transition to a different architecture \citep{deBoer.17}.}. In the limit of little delay induced by the cable and our limited delay resolution, a reflection along this stage of the signal chain would produce an alias of the foreground signal at a $\tau \approx 100$\,ns \citep{Beardsley.16, Ewall-Wice.EoX}.

\subsection{Day-to-day variability}

The foreground and EoR window power levels appeared to be relatively stable between days, with variation most likely due to the incomplete sky model used for gain calibration. Figure~\ref{fig:hera19_highband_cuts_per_day} shows power as a function of baseline length for $k_{\parallel}=0$ h/Mpc (solid lines) and $k_{\parallel}=0.2$ h/Mpc (dot-dashed lines). Deviations from the mean at $k_{\parallel}=0$ h/Mpc may be a limitation imposed by our simplistic sky model. Since the noise levels in the EoR window region remained noise-like throughout our observations, the uncertainty in the absolute gain scale did not have a large impact on our largely-diagnostic investigation.

\begin{figure}
\centering
\includegraphics[scale=0.5]{chapters/eor_window_HERA/figures/highband_by_day.pdf}
\caption[High band power as a function of baseline length for the center of the pitchfork ($k_{\parallel}=0$ h/Mpc) and in the EoR window ($k_{\parallel}=0.2$ h/Mpc) for each JD of observation. ]{High band power as a function of baseline length for the center of the pitchfork ($k_{\parallel}=0$ h/Mpc; solid lines) and in the EoR window ($k_{\parallel}=0.2$ h/Mpc; dot-dashed lines) for each JD of observation. The black dashed line represents the approximate noise power assuming a receiver temperature of 300\,K. A very similar relationship is shown in the low band, but with a higher noise floor, which is consistent with system temperature as a function of frequency. The noise level climbs with baseline length as the compact nature of the array gives more short baselines to average-over in a given $(k_{\parallel},k_{\perp})$ bin than longer ones.}
\label{fig:hera19_highband_cuts_per_day}
\end{figure}

\subsection{Polarimetric results}
\label{subsec:polarimetric_results}

Figures~\ref{fig:hera19_pitchforks_highband} and~\ref{fig:hera19_pitchforks_lowband} qualitatively illustrate that the simulations described in Section~\ref{sec:hera19_leak} reproduced the main features of the observed power spectra. The simulations were run only with a Stokes I sky component and no simulated calibration errors, so the only signal in the polarized power spectra was from wide-field beam leakage (Figure~XXX). An example comparison between simulation and observation in the image plane is shown in Figure~\ref{fig:hera19_GCimage}.

In Figure~\ref{fig:hera19_bl0_cuts_vs_sim} we show the power levels observed on the shortest baseline (14.7\,m) compared to our simulations for each band. The simulations used an unpolarized diffuse sky model \citep[the most recent version of the GSM;][]{GSM.17}, which should be accurate at the scales probed by a 14.7\,m baseline. Inset panels zoom-in on the region around $k_{\parallel}=0$\,h/Mpc, where most of the foreground power was concentrated.
We saw that the simulations reproduced $\sim75\%$ of the foreground power observed in pseudo-Stokes I in the high band, and over-predicted foreground power by $\sim35\%$ in the low band. This could have been due to unrealistic frequency scaling of the diffuse foregrounds in the GSM. 

For pseudo-Stokes Q and U, the simulations accounted for $\sim 60-75\%$ of power seen within the pitchfork region, suggesting that most of the power seen in these power spectra, at least for the shortest baselines, can be mostly attributed to direction-dependent leakage effects. As noted in Section~\ref{sec:hera19_leak}, residual gain and phase errors are able to leak a fraction of pseudo-Stokes I into Q and U, but some fraction of the observed power ($\leqslant 25\%$) may have been due to linearly polarized foregrounds. This is corroborated by residual power close to the location of the Galactic Center, and increased power over the sky, in the observed pseudo-Stokes Q and U skies compared to the simulated ones in Figure~\ref{fig:hera19_GCimage}. As the Galactic Center is the highest-amplitude source of power, we expect residual gain errors to be most obvious in the same position as it is in the pseudo-Stokes I image. Such an excess is present in the observed pseudo-Stokes Q and U images, but absent in the simulated ones -- pointing to direction-independent gain errors being present. However, the simulated pseudo-Stokes Q and U images contain only direction-dependent leakage from Stokes I. Since they reproduce most of the features seen in the observed data, pseudo-Stokes Q and U are clearly dominated by direction dependent leakage.

\cite{Lenc.16} observed linearly polarized emission from diffuse structure with $\sim 1.6 - 4.5\%$ fractional polarization at 150\,MHz, corresponding to power levels of $\sim10^5$ mK$^2$Mpc$^3$h$^{-3}$. This power level is similar to expected EoR power levels \citep[e.g.][]{Lidz.07, Moore.13, Nunhokee.17}; a detection of a power spectrum of polarized galactic synchrotron will require much deeper integrations.

\begin{figure*}
\centering
\includegraphics[width=0.9\textwidth]{chapters/eor_window_HERA/figures/highband_bl0_4pol_with_zoom.pdf}\\
\includegraphics[width=0.9\textwidth]{chapters/eor_window_HERA/figures/lowband_bl0_4pol_with_zoom.pdf}
\caption[Simulated and observed power as a function of $k_{\parallel}$ for the shortest baseline (14.7\,m).]{Simulated and observed power as a function of $k_{\parallel}$ for the shortest baseline (14.7\,m). \textit{Right to left}: pseudo-Stokes I, Q, U and V; \textit{above}: the high band; \textit{below}: the low band. The simulations were noiseless and used an unpolarized sky model. Inset panels zoom-in on the peak region. They capture the foreground power levels in pseudo-Stokes I, Q and U, suggesting all power in Q and U is due to leakage from Stokes I. The power level in V is highly discrepant, however, suggesting some sort of beam-independent instrumental leakage.}
\label{fig:hera19_bl0_cuts_vs_sim}
\end{figure*}

The observed pseudo-Stokes V power spectrum was more poorly modelled by our simulation. In both bands we observed $\sim$20 dB more power in pseudo-Stokes V at $k_{\parallel}=0$\,h/Mpc than predicted by our simulations. The peak power observed in pseudo-Stokes V was roughly 0.1\% of the peak power observed in pseudo-Stokes I. Likewise in the sky images shown in Figure~\ref{fig:hera19_GCimage}, there is little pseudo-Stokes V power in the simulated images, compared to observation. This suggests that most or all of the power in pseudo-Stokes V is due to direction independent leakage. While the leakage appears localized in Figure~\ref{fig:hera19_GCimage}, we see in Figure~\ref{fig:hera19_bl0_cuts_vs_sim} that it is statistically similar to pseudo-Stokes Q and U in power.
Since $D$-terms cause direction-independent leakage from pseudo-Stokes I to pseudo-Stokes V, the excess power we observed could be interpreted as an approximate $D$-term level of $\sim$1\% \citep{TMS}. This is similar to $D$-term levels from other low frequency instruments such as MWA-32, which was found to have $\sim$2\% $D$-terms (G. Bernardi, private communication). 
The under-prediction of pseudo-Stokes V from the simulation could, of course, also be due to some unmodelled direction-dependent instrumental effect.

To understand which effect, if either, is dominant, a precise $D$-term calibration of HERA is required. This effort is underway with data taken with bright polarized point sources in transit, and will be presented in future work. Another potential cause of the discrepancy could have been that our simulations under-predicted Stokes V power, due to lack of accounting for some variety of instrumental circular polarization.

In Section~\ref{subsec:general_features} we noted the presence of excess power at $k_{\parallel}=\pm 0.04$\,h/Mpc ($\pm$100\,ns) that was independent of baseline length, suggesting that it was due to a reflection along 15\,m cables. Figure~\ref{fig:hera19_bl0_cuts_vs_sim} shows that power at this delay is not consistent between polarizations. Stokes U and V power only exhibited excess signal at -100\,ns in the high band, and in the low band, it was only Stokes U that did not exhibit that excess at +100\,ns. This may be a clue about the polarization state of cable reflections, perhaps as a function of frequency, but we defer this to future work -- noting it as a point of interest here.

\section{Conclusions}
\label{sec:hera19_conc}

In this work we have presented polarized power spectra from the HERA-19 commissioning array. With modest calibration, HERA is able isolate total intensity and polarized foregrounds to within the ``pitchfork'' region of \textit{k}-space, as predicted by \cite{Nithya.15b}, lending confidence to its future performance as an instrument capable of both detecting and characterizing the EoR power spectrum. Of course, the array used in this study had just 19 antennae, 15 of which were used for analysis -- future build-outs of HERA with up to 350 antennae will require strong quality-assurance efforts.

Simulations of the polarized response of the instrument, mapped into the same Fourier space as the data, suggest that most or all of the polarized power observed in pseudo-Stokes Q and U power spectra is due to direction-dependent beam leakage from pseudo-Stokes I. Residual gain and phase errors could account for the rest of the power, but some fraction of the total ($\leqslant 25\%$) may be due to linearly polarized foregrounds. Excess power in pseudo-Stokes V may be due to $D$-terms at the 1\% level, but a full image-based calibration with a polarized point source is required to confirm this. The general accuracy of our simulations suggests current modelling of the complex HERA beam is accurate.
%
%
%
\chapter{Deep integrations on polarization with PAPER-128}
\label{chapter:eor_window_psa128}

In Chapter~\ref{chapter:eor_window_paper32img}, I presented polarized power spectra from a short integration -- a few hours of one night -- over a wide range of $k_{\perp}$-modes probed by the PAPER-32 polarized imaging array \citep{Kohn.16}. Chapter~\ref{chapter:eor_window_HERA} presented the first power spectral results from HERA. The HERA-19 commissioning array was small and dense, meaning that only a few $k_{\perp}$-modes were accessible. For that study, we averaged over 10 hours per night, for 8 consecutive nights \citep{Kohn.18}. This Chapter presents results from the PAPER-128 array. In this Chapter I present roughly one quarter of the total number of observations recorded by this interferometer (Section~\ref{sec:psa128_obs}). I show results of a deep integration on a very narrow range of $k_{\perp}$-modes (corresponding to $\sim$30\,m spacings of the redundant grid; Section~\ref{sec:psa128_results}) and discuss the implications for deep, fully-polarized integrations with large interferometers (Section~\ref{sec:psa128_conc}).

\section{Observations \& Reduction}
\label{sec:psa128_obs}
 
PAPER-128 was the largest build-out of the PAPER experiment. As described in Chapter~\ref{chapter:instruments},  PAPER-128 consisted of 128 antennas, 112 of which were arranged in a redundant grid. An annotated photograph of the array is shown in Figure~\ref{fig:psa128photo}. In this section, I review the PAPER-128 campaign (Section~\ref{subsec:psa128_obs_overview}) and the subsequent reduction of roughly one quarter of the total number of observations (Section~\ref{subsec:psa128_s1e2_reduction}).

\begin{figure}
\includegraphics[width=0.8\textwidth, angle=270]{chapters/psa128_pol/figures/array_photo_diagram.pdf}
\caption[An annotated photograph of the PAPER-128 array.]{An annotated photograph of the PAPER-128 array, looking to the East. Highlighted are the 112 antenna redundant grid, with 15\,m East-West spacings between each row; outlier antennas from the main grid used to increase \textit{uv}-coverage; coaxial cables running to the receiverators and correlator (see Chapter~\ref{chapter:instruments}). An inset panel shows a PAPER sleeved dipole. Photo credit: J. E. Aguirre. Figure credit: C. D. Nunhokee; \citep{Nunhokee_thesis}.}
\label{fig:psa128photo}
\end{figure}

\subsection{Overview of PAPER-128 observations}
\label{subsec:psa128_obs_overview}
Observations were recorded for two years, with first light on November 20th 2013 and final readings on January 27th 2015. However, these observations were not always contiguous. Human errors, experimentation and malfunctioning electronics required the correlator and connected electronics to be turned off and restarted, altering the characteristic phasing and gain scale of the array. Each of these restarts constituted the beginning of a new ``Epoch" of the array which required different quality assurance steps and initial calibration stages. Table~\ref{tab:seasons_psa128} summarizes the length and nature of these Epochs. 

\begin{deluxetable}{lllll}
\centering
\label{tab:seasons_psa128}
\tablewidth{0pt}
\tablecaption{PAPER-128 Observing Seasons \& Epochs}
\tabletypesize{\footnotesize}
\tablehead{
\colhead{Season} & \colhead{Epoch} & \colhead{Julian Dates} & \colhead{Calendar Dates} & \colhead{Notes} \\
}
\startdata
1 & 1 & 2456617 - 2456673 & Nov 20, 2013 - Jan 15, 2014 & 1/8 F-Engine failure \\
   & 2 & 2456678 - 2456724 & Jan 20, 2014 - Mar 7, 2014 &  Good \\
2 & 1 & 2456625 - 2456732 & Mar 8, 2014 - Mar 7, 2014 & Too few data \\
   & 2 & 2456843 - 2456873 & Jul 4, 2014 - Aug 3, 2014 & Uninteresting LST range \\
   & 3 & 2456881 - 2456928 & Aug 11, 2014 - Sep 27, 2014 &  Many malfunctioning antennas \\
   & 4 & 2456942 - 2457008 & Oct 11, 2014 - Dec 11, 2014 &  Good \\
   & 5 & 2457030 - 2457050 & Jan 7, 2015 - Jan 27, 2015 &  Many malfunctioning antennas\\
\enddata
\end{deluxetable}

Figure~\ref{fig:psa128_epochs} illustrates the challenge imposed by the correlator restarts. Epoch changes were characterized by large shifts in the overall phase of visibilities, of course leading to changed magnitudes of the real and imaginary parts of the visibilities recorded. The quality assurance metrics described in Chapter~\ref{chapter:data_prep_and_proc} were sensitive to these changes. Of course, absolute calibration -- that is phasing to the correct point on the sky and scaling the visibilities from arbitrary to physical flux density units -- had to be run separately on individual Epochs.

\begin{figure}
\centering
\includegraphics[width=0.7\textwidth]{chapters/psa128_pol/figures/S2_chan100_hist.pdf}
\includegraphics[width=0.5\textwidth]{chapters/psa128_pol/figures/S2_chan100_phasegrid.pdf}
\caption[The challenge of Epoch changes.]{The challenge of Epoch changes. Shown are all of the Season 2 time samples of the visibilities recorded by the 30\,m baseline between antennas 1 and 4, LSTs 0--5, for only the 150\,MHz frequency bin. The above panel shows a histogram of the real part of the visibilities as a function of LST -- there is a dramatic change in magnitude with respect to LST. Likewise, the lower panel shows the phase of each visibility sample (color axis) as a function of LST (vertical axis) and day of observation (x axis). There are obvious large shifts in phase, which require separate calibration stages.}
\label{fig:psa128_epochs}
\end{figure}

As shown in Table~\ref{tab:seasons_psa128}, Season 2 had a larger number of observed nights than Season 1. However, the analysis of Season 2 was especially challenging due to large numbers of malfunctioning antennas. This may have been due to the antennas ageing past a critical point. Nothing in the array was replaced during build-outs except for the correlator; 32 of the antennas had been out in the desert for 4--5 years (PAPER-32) and another 32 for 3--4 years (PAPER-64). Most of the time, these were the antennas that were identified as malfunctioning.

Season 2 Epoch 4 was relatively well-behaved, and may be analyzed in the future. For this work, we concentrated our analysis on Season 1. Season 1 Epoch 2 was ten days shorter than Season 1 Epoch 1, but Epoch 1 had two major challenges associated with it: a data loss event, and an F-engine failure. Due to human error, Epoch 1 data was deleted and had to be restored and recompressed (see Chapter~\ref{chapter:data_prep_and_proc}). This was almost entirely successful, at the loss of one week's worth of observations. The F-engine failure was more critical. We discovered during out analysis that exactly one eighth of the antennas in the array produced noise-like visibilities with the rest of the array, but normal correlations between one another. These antennas had ``seceded" from the array. They shared the characteristic of all being attached to the same F-engine (of which there were eight; see Figure~\ref{fig:instruments_PAPER_Signal_Chain}). This suggested that there was a clock-offset on that F-engine, resulting in no correlation between the signal from those antennas and those running through the other in-sync F-engines.

This left Season 1 Epoch 2 as the most well-characterized and well-behaved Epoch of PAPER-128 observations, and we focused on this Epoch alone from now on. Cross-polarization metrics identified seven incorrectly-rotated antennas, which were corrected during initial processing. Eight antennas were identified as malfunctioning by mean visibility amplitude metrics, including one of those that was incorrectly-rotated. The state of the array for Season 1 Epoch 2 is summarized in Figure~\ref{fig:s1e2_array}.

\begin{figure}
\centering
\includegraphics[width=0.6\textwidth]{chapters/psa128_pol/figures/s1e2_array.png}
\caption[Good, bad and rotated antennas in Season 1 Epoch 2.]{Good, bad (red crosses) and rotated (blue circles) antennas in Season 1 Epoch 2, as identified by the cross-polarization and visibility amplitude metrics defined in Chapter~\ref{chapter:data_prep_and_proc}.}
\label{fig:s1e2_array}
\end{figure}

\subsection{Reduction of Season 1 Epoch 2 data}
\label{subsec:psa128_s1e2_reduction}

After correcting for cross-polarized and malfunctioning antennas, we were able to begin processing the data using the redundant calibration techniques described in Chapter~\ref{chapter:polcal}. We used the phase-flattening algorithm described in Section~\ref{sec:polcal_data} to solve for phase slopes for each antenna. This was performed for the North-South and East-West feed arms separately using the linear instrumental polarizations (`nn' and `ee'), as this method is sensitive to signal-to-noise. We performed a single calculation of these overall delays at the start of the Epoch, and applied those to the rest of the days observed. These overall delays are largely due to electrical delays along the cables leading to the receiverators and the correlator, so we did not expect them to vary a by large amount.

\subsubsection{Redundant calibration}

After phase-wraps were flattened, the {\sc omnical} algorithm could be invoked safely. For this study, we implemented the \textit{4pol+minV} calibration method. For a full exploration of {\sc omnical}ibration methods, see Chapter~\ref{chapter:polcal}. Briefly, the \textit{4pol+minV} calibration scheme redundantly solves for diagonal gains for the North-South and East-West feed arms at the same time, \textit{and} imposes that the `ne' and `en' visibilities are equal. That is, it produces a redundant calibration which minimizes pseudo-Stokes V. Figures~\ref{fig:psa128_pre_post_abs} and \ref{fig:psa128_pre_post_phs} show the successful results of the \textit{4pol+minV} {\sc omnical}ibration, where visibilities from 30\,m East-West baselines showed a high degree of redundancy in all pseudo-Stokes polarizations, and pseudo-Stokes V is almost complete noise-like.

\begin{figure}
\centering
\includegraphics[width=0.8\textwidth]{chapters/psa128_pol/figures/nocal_IQUV_6680_0-5.pdf}
\includegraphics[width=0.8\textwidth]{chapters/psa128_pol/figures/omnical_IQUV_6680_0-5.pdf}
\caption[Amplitudes of three redundant visibilities before and after {\sc omnical}ibration using the \textit{4pol+minV} scheme.]{Amplitudes of three redundant visibilities before (above) and after (below) {\sc omnical}ibration using the \textit{4pol+minV} scheme. All four pseudo-Stokes polarizations attained a high degree of redundancy in magnitude after calibration. The color axis is logarithmic and spans 5 orders of magnitude in arbitrary data units.}
\label{fig:psa128_pre_post_abs}
\end{figure}

\begin{figure}
\centering
\includegraphics[width=0.8\textwidth]{chapters/psa128_pol/figures/nocal_IQUV_6680_phase.pdf}
\includegraphics[width=0.8\textwidth]{chapters/psa128_pol/figures/omnical_IQUV_6680_phase.pdf}
\caption[The same visibilities as shown in Figure~\ref{fig:psa128_pre_post_abs}, but showing their phases instead of their amplitudes.]{The same visibilities as shown in Figure~\ref{fig:psa128_pre_post_abs}, but showing their phases instead of their amplitudes. Again, a high degree of redundancy is obtained between baselines for all polarizations. The color axis is linear and spans $\pi$ (red) to $-\pi$ (blue) radians.}
\label{fig:psa128_pre_post_phs}
\end{figure}

After calibration, we down-selected to only the baselines we sought to analyze for our power spectrum studies. This was partly a utilitarian step: reducing the entire data set would represent a large feat of data processing, since Season 1 Epoch 2 was $\sim2.5$ TB in size if all baselines were retained, and several processing stages that would duplicate the data were still required. The baselines kept were the 30\,m East-West type and their closest diagonals -- that is, 30\,m East-West \& $\pm$4\,m North-South baselines ({\color{red} e.g. Kolopanis et al. (2018)}).

\subsubsection{Foreground removal}
% - linCLEAN
To remove foreground signal from the data, we implemented a variation of the 1D-CLEAN \citep{ParsonsBacker.09} used by past PAPER studies \citep[][{\color{red}; Kolopanis et al. 2018}]{Parsons.14, Ali.15, Jacobs.15, Moore.17, Kerrigan.18}. Instead of performing an iterative CLEAN we used a linear least-squares approach which we termed {\tt linCLEAN}.

Using {\tt linCLEAN}, we endeavoured to model the foreground component of the visibilities using the finite number of Fourier modes that exist within the foreground wedge of the EoR window paradigm. The number of these modes was set by the frequency resolution of the instrument, and the baseline length (see Chapter~\ref{chapter:eor_window_theory}).

We constructed a visibility $\vec{m}(\nu)$ as a model, per time integration, of an individual visibility $\vec{d}(\nu)$ (which we represent as a vector along the frequency axis), seeking to minimize the chi-square

\begin{equation}
\chi^2(\nu) = (\vec{d}(\nu) - \textbf{A}(\nu',\tau)\vec{m}(\nu))^T \textbf{W}(\nu',\nu)(\vec{d}(\nu) - \textbf{A}(\nu,\tau)\vec{m}(\nu)).
\label{eq:linclean_chisquare}
\end{equation}

In the above equation, the matrix $\textbf{A}$ had dimensions ``number of frequency channels" by ``number of allowed delay modes". An allowed delay mode $\tau$ was within the interval [0,  $\frac{|\vec{b}|}{c} + t_{SH}$], for baseline vector $\vec{b}$, speed of light $c$ and an allowed ``supra-horizon leakage" term $t_{SH}$ \citep{Pober.13}, which we set to 15ns.

The contents of $\textbf{A}$ were the concatenation of matrices $\textbf{C}$ and  $\textbf{S}$:
\begin{eqnarray}
\textbf{C}_{ij} &=  \cos(2\pi \nu_i \tau_j)\\
\textbf{S}_{ij} &=  \sin(2\pi \nu_i \tau_j)\\
\end{eqnarray}
for frequency channel $i$ and delay bin $j$.
The matrix $\textbf{W}$ was diagonal, and assigned weighting per frequency channel. With an estimated system temperature one could implement an inverse variance weighting per frequency channel, but we pursued a simpler scheme where the entries were zero for RFI-flagged channels, and unity otherwise.

This system was least-squares solvable for the $\tau$ modes, and granted a foreground model

\begin{equation}
\vec{m}(\nu) = (\textbf{A}^T \textbf{W} \textbf{A})^{-1}\textbf{A}^T\textbf{W}\vec{d}(\nu)
\end{equation}
which could subsequently be subtracted from the $\vec{d}(\nu)$, leaving only the noise-like backgrounds.

\subsubsection{Binning in LST}

After running {\tt linCLEAN} on the Epoch, we implemented a round of RFI flagging that clipped any samples that represented $>4\sigma$ fluctuations above the average, where averages were performed along the time and frequency axes.
We could then average-down on the noise by binning visibilities according to the LST they were observed at. The LST bin size used was 41s long, and we split the Epoch into even and odd days, constructing two separate LST-binned data sets. Cross-multiplying these allowed us to construct an unbiased power spectrum estimate \citep[e.g.][{\color{red}; Cheng et al. 2018}]{Parsons.14}.
Unflagged RFI events would dominate any other signal in a given LST bin. To avoid binning RFI with sky signal, before averaging we computed the median of all observations in a given LST bin and flagged any observations with amplitude $>3\sigma$ above the median. This clipping narrowed the distribution of visibilities about the median, altering the thermal noise variance, but leaving the expectation value unchanged, so we expect little loss of signal due to this step.

For this analysis, we created two LST-binned data sets (each split into even and odd days): a set constructed from the foreground subtracted visibilities, and another set with the foregrounds included. We used the latter to calculate an absolute calibration of both data sets.

\subsection{Absolute calibration \& fringe-rate filtering}
% - (unpolarized) PicA (Jacobs13) CASA cal on lstbin_fg
% - Stokes I FRF on all 4pols of fgsub
We formed images of the Pictor A transit in order to derive an absolute calibration for the `n' and `e' feed arms in the fashion described in Chapter~\ref{chapter:polcal}. We converted the {\sc miriad} files of the LST-binned foreground data into {\tt CASA} \citep{casa} Measurement Sets at LST$\approx$5.3 hours (the relevant LST for the transit of Pictor A). 
Our sky model consisted only of Pictor A as a unpolarized point source. Because we had already down-selected to the power spectrum baselines, Pictor A completely dominated the signal. The images of the four pseudo-Stokes parameters we obtained are shown in Figure~\ref{fig:psa128_abscal_images}. In that image, pseudo-Stokes I is on a color scale with double the dynamic range of Q, U and V -- that is, pseudo-Stokes Q, U and V were almost completely noise-like. In pseudo-Stokes I, Pictor A dominated the field. These images were low quality because only the power spectrum baselines were used to produce the image, leading to large grating lobes and an elongated point-spread function; shown in Figure~\ref{fig:psa128_psf}.

\begin{figure}
\hspace{-2cm}\begin{tabular}{ll}
\includegraphics[clip, trim=0.5cm 9cm 5cm 5cm, width=0.6\textwidth]{chapters/psa128_pol/figures/pretty_I_0-022.pdf} &
\includegraphics[clip, trim=0.5cm 9cm 5cm 5cm, width=0.6\textwidth]{chapters/psa128_pol/figures/pretty_Q_0-011.pdf}\\
\includegraphics[clip, trim=0.5cm 9cm 5cm 5cm, width=0.6\textwidth]{chapters/psa128_pol/figures/pretty_U_0-011.pdf} &
\includegraphics[clip, trim=0.5cm 9cm 5cm 5cm, width=0.6\textwidth]{chapters/psa128_pol/figures/pretty_V_0-011.pdf} \\
\end{tabular}
\caption[Images of Pictor A, used to calibrate the absolute scaling to physical units.]{Images of Pictor A, used to calibrate the absolute scaling to physical units. These images are multi-frequency syntheses, but we used per-channel information for the final calibration (see Figure~\ref{fig:psa128_bandpases}). 
Pseudo-Stokes I is on a linear scale of twice the dynamic range of pseudo-Stokes Q, U and V. A beam ellipse is shown in the South-West corner of the images.
The poor quality of the images is a result of using only the baselines that go into the power spectrum estimates. Pictor A dominates the field in pseudo-Stokes I, and is of the morphology expected given the PSF shown in Figure~\ref{fig:psa128_psf}.}
\label{fig:psa128_abscal_images}
\end{figure}

\begin{figure}
\centering
\includegraphics[clip, trim=0.5cm 9cm 5cm 5cm, width=0.6\textwidth]{chapters/psa128_pol/figures/pretty_PSF_0-1.pdf}
\caption[The point-spread function of an array consisting only of the $\sim$30\,m power spectrum baselines.]{The point-spread function of an array consisting only of the $\sim$30\,m power spectrum baselines. The sparseness and asymmetry of the array lead to an elongated PSF with large grating lobes.}
\label{fig:psa128_psf}
\end{figure}

We used the {\tt CASA} {\tt bandpass} routine to derive an overall frequency-dependent scaling that brought the North-South and East-West dipole arms to a physical scale that minimized pseudo-Stokes Q in Pictor A, and converted the data units to a physical level in Janskies. For the Jansky scaling we used the spectrum from \cite{Jacobs.13}. {\tt CASA} provided separate scalings for all antennas used in the analysis. We plot the average scaling for the North-South and East-West dipole arms in Figure~\ref{fig:psa128_bandpases}, shading-in the standard deviation between antennas. We implemented aggressive RFI flags, leading to large gaps in the spectrum. The higher frequency portion of the band had a very low variance its bandpass solutions, as expected given that they were {\sc omnical}ibrated and that the low band was historically poorly-behaved and characterized (e.g. Chapter~\ref{chapter:polcal}).

\begin{figure}
\centering
\includegraphics[width=0.8\textwidth]{chapters/psa128_pol/figures/avg_bandpasses.pdf}
\caption[The average bandpass scaling used for absolute calibration.]{The average bandpass scaling used for absolute calibration, with shading indicating the 1$\sigma$ deviation about the average across antennas.}
\label{fig:psa128_bandpases}
\end{figure}


\section{Results}
\label{sec:psa128_results}

% note -- use high half of the band because reasons e.g. bandpass
% - not-that-great power spectra

\section{Discussion \& Conclusions}
\label{sec:psa128_conc}
% Pitch to "future work", to end the section
% -- levels predicted for detection of Q,U: Nunhokee, Asad
% -- challenges associated with very long integrations: QUALITY ASSURANCE, instrument stability
% -- ionosphere

%
%
%
\part{Expanding the potential of EoR measurements}


\vspace*{\fill} 
\begin{quote} 
\centering 
There is one glory of the sun, and another glory of the moon, and another glory of the stars; for star differs from star in glory. \\
\textit{1 Corinthians (15:41)}
\end{quote}
\vspace*{\fill}

%
% other stuff -- needs a brief non-chapter introduction
%
\chapter{Time-Averaged Visibilities}
\label{chapter:TAV}
%
% 
%
\chapter{Higher-order correlation functions between kSZ and 21cm observations}
\label{chapter:ksz_21cm}

In Chapter~\ref{chapter:eor_intro}, it was discussed that there are many probes of the EoR beyond {\sc hi}. Secondary anisotropies of the CMB can be used as probes of reionization, mainly due to the fact that reionization represents a large source of free electrons, which CMB photons can scatter off of. The pattern of scattering -- the secondary anisotropies -- is sensitive to the topology of the {\sc hi} field. This Chapter focuses on a particular mechanism for producing secondary anisotropies; the kinetic Sunyaev-Zel'dovich effect (kSZ; CITE SOMETHING). We present novel mathematical theories for understanding the correlation between future CMB and EoR measurements, taking instrumental noise and the EoR window into account in a way not presently explored in the literature. In Section~\ref{sec:ksz-21cm} we make clear the link between the EoR and the kSZ effect, and the separate challenges to detecting either. In Sections~\ref{sec:bispec} and \ref{sec:trispec}, we present semi-analytic theory and results from simulation for the kSZ$^2$-21cm bispectrum and the kSZ$^2$-21cm$^2$ trispectrum, respectively.

\section{The kSZ-21cm connection}
\label{sec:ksz-21cm}
% note somewhere in here that we always need two lots of kSZ 

% Since the kSZ is sourced by the momentum of ionized gas, we expect it to be correlated with the ionized bubbles during reionization, and therefore anticorrelated with 21cm intensity maps.

% note link between n-point correlation functions and fourier spectra

% mathematical form of kSZ and 21cm maps

\begin{equation}
\delta_{\rm 21cm}(\vec{x}) \approx T_0^{\rm 21cm} \left\langle x_{\sc hi} \right\rangle \left[1 + \delta_{x_{\rm HI}}(\vec{x})\right] \left[1 + \delta_\rho(\vec{x})\right]
\label{eq:d21cm}
\end{equation}
where $\delta_\rho$ is fractional baryon overdensity, and $\left\langle ... \right\rangle$ indicates an average over position $\vec{x}$. To compare with the kSZ, described below, it will be useful to speak in terms of the ionization overdensity field $\delta_x(\vec{x})$, rather than the neutral overdensity one ($\delta_{x_{\sc hi}}(\vec{x})$) above. This just brings out a prefactor:

\begin{equation}
\delta_{x_{\sc hi}}(\vec{x}) = \frac{-\langle x_i \rangle}{1 - \langle x_i \rangle} \delta_x(\vec{x})
\end{equation}
where $\langle x_i \rangle$ is the spatial mean ionization fraction. So our representation of the 21\,cm temperature contrast becomes

\begin{equation}
\delta_{\rm 21cm}(\vec{x}) = T_0^{\rm21cm} (1 - \langle x_i \rangle ) \left[ 1 - \frac{\langle x_i \rangle}{1 - \langle x_i \rangle} \delta_x(\vec{x}) + \delta_{\rho}(\vec{x})  - \frac{\langle x_i \rangle}{1 - \langle x_i \rangle} \delta_x(\vec{x}) \delta_{\rho}(\vec{x}) \right]
\label{eq:d21cm_x}
\end{equation}

% assumptions


\subsection{Foregrounds}

\subsection{Noises}

\section{kSZ$^2$-21cm squeezed-triangle bispectra}
\label{sec:bispec}

In this section we present an estimate for bispectra formed from future HERA 21\,cm intensity maps (large scales; $\ell < 300$) and Stage 3 or 4 CMB maps (\citet{cmbS4.16}; small scales $\ell > 3000$). These disparate scales stretch the three-point correlation function into a `squeezed triangle' in Fourier space.

\subsection{Semi-analytic approximation}

Consider the bispectrum of two Fourier transformed kSZ maps and one Fourier transformed 21\,cm map under the Limber approximation: 
% OVERFLOW
\begin{equation}
\left\langle T_{\rm kSZ}(\ell_1) T_{kSZ}(\ell_2) T_{\rm 21cm}(\ell_3)\right\rangle = (2\pi)^2 \delta_D(\ell_1 + \ell_2 + \ell_3) \int \frac{{\rm d}\chi}{\chi^4} W_{\rm 21cm}(\chi) W_{\rm kSZ}^2(\chi) B_{\rm 21cm, kSZ, kSZ}(\ell_1/\chi, \ell_2/\chi,  \ell_3/\chi; \chi)
\label{eq:bispec_limber}
\end{equation}
where the window functions are based on global quantities associated with the maps:

\begin{equation}
W_{\rm 21cm}(\chi) = {\rm \frac{d}{d\chi}}\left(T_{\rm 21cm}(z)\right)
\label{eq:W21cm}
\end{equation}
where ${\rm d\chi}$ is the comoving distance probed by the 21\,cm map, given by the observing bandwidth, and

\begin{equation}
W_{\rm kSZ}(\chi) = T_{\rm CMB} \frac{\sigma_T n_e(z)}{c}\frac{\left\langle x_i \right\rangle e^{-\left\langle \tau(z) \right\rangle } }{1+z}
\label{eq:WkSZ}
\end{equation}
where redshift \textit{z} corresponds to a given comoving distance $\chi$, as determined by the central redshift of the 21\,cm cube, $n_e(z)$ is the average number density of electrons at that redshift, $\left\langle x_i \right\rangle $ is the average ionization fraction at that redshift, $\sigma_T$ is the Thomson cross section and $\tau (z)$ is the optical depth to redshift \textit{z}. $T_{\rm CMB} = 2.725\pm0.002$\,K \citep{Mather.99}.

Now we consider the Limber approximation of a related quantity: the 21\,cm field correlated with the square of the line-of-sight-projected momentum field:

\begin{equation}
\begin{split}
\left\langle \vec{q} \cdot \hat{n}(\vec{k}_1) \vec{q} \cdot \hat{n} (\vec{k}_2) T_{\rm 21cm}(\vec{k}_3) \right\rangle =
\int \int \frac{ {\rm d^3k' \, d^3k''} }{(2\pi)^6} (\hat{k'}\cdot\hat{n}) (\hat{k''}\cdot\hat{n}) \times \\
\left\langle 
\vec{v}(\vec{k'})\vec{v}(\vec{k''}) 
\left[ \delta_x(\vec{k}_1 - \vec{k'}) + \delta_{\rho}(\vec{k}_1 - \vec{k'})\right]
\left[ \delta_x(\vec{k}_2 - \vec{k''}) + \delta_{\rho}(\vec{k}_2 - \vec{k''})\right] 
T_{\rm 21cm}(\vec{k}_3)
\right\rangle
\end{split}
\label{eq:q_n_21}
\end{equation}
where we shifted our concentration from $\ell$ space to $\vec{k}$-space, which was more convenient to work in for the derivations below.
One of our assumptions in Section~\ref{sec:ksz-21cm} was that the velocity was coherent on large scales, and therefore should not correlate with $\delta_{x}$ or $\delta_{\rho}$. This allows us to expand the $[...]$ terms in Equation~\ref{eq:q_n_21} in their own spatial averages, and take the product of velocities into its own spatial average:

\begin{equation}
\left\langle \vec{v}(\vec{k'})\vec{v}(\vec{k''}) \right\rangle = (2\pi)^3 \delta_D(\vec{k'} + \vec{k''})P_{vv}(\vec{k'})
\label{eq:Pvv}
\end{equation}
where the $\delta_D(\vec{k'} + \vec{k''})$ in the above relation allowed us to integrate-out our $k''$ dependence. Referring to terms of the form $\langle \delta_i(\vec{k}_1 - \vec{k'}) \delta_j(\vec{k}_2 - \vec{k''}) \delta_{\rm 21cm}\rangle$ as $B_{i,j,{\rm 21cm}}$, we could express the overall correlator $\left\langle \vec{q} \cdot \hat{n}(\vec{k}_1) \vec{q} \cdot \hat{n} (\vec{k}_2) T_{\rm 21cm}(\vec{k}_3) \right\rangle$ in terms of ``component bispectra":

\begin{equation}
\begin{split}
\left\langle \vec{q} \cdot \hat{n}(\vec{k}_1) \vec{q} \cdot \hat{n} (\vec{k}_2) T_{\rm 21cm}(\vec{k}_3) \right\rangle = \\
(2\pi)^3 \delta_D(\vec{k}_1 + \vec{k}_2 + \vec{k}_3) 
\int \frac{{\rm d^3}k'}{(2\pi)^3} (\hat{k'}\cdot\hat{n})^2 P_{vv}(k') 
\left[ B_{x,x,\rm 21cm} + B_{x,\rho,\rm 21cm} + B_{\rho,x,\rm 21cm} +B_{\rho,\rho,\rm 21cm} \right].
\end{split}
\end{equation}

In the squeezed-triangle limit, where $k_3 << k_1,k_2$, this reduced to:
\begin{equation}
\begin{split}
\left\langle \vec{q} \cdot \hat{n}(\vec{k}_1) \vec{q} \cdot \hat{n} (\vec{k}_2) T_{{\rm 21cm}}(\vec{k}_3) \right\rangle = \\
(2\pi)^3 \delta_D(\vec{k}_1 + \vec{k}_2 + \vec{k}_3) \frac{v_{\rm rms}^2}{3}
\left[ B_{x,x,{\rm 21cm}} + B_{x,\rho,{\rm 21cm}} + B_{\rho,x,{\rm 21cm}} +B_{\rho,\rho,{\rm 21cm}} \right]
\end{split}
\label{eq:component_bispectra}
\end{equation}

Of course, $\delta_{\rm 21cm}$ also contains information about $\delta_x$ and $\delta_{\rho}$. Using Equation~\ref{eq:d21cm_x}, we expanded each component bispectrum as functions of $\delta_x$, $\delta_{\rho}$ and $\delta_x\delta_{\rho}$:

\begin{equation}
\begin{split}
B_{x,x,{\rm 21cm}} \propto B_{x,x,x} + B_{x,x,\rho} + B_{x,x,x\rho}\\
B_{x,\rho,{\rm 21cm}} \propto B_{x,\rho,x} + B_{x,\rho,\rho} + B_{x,\rho,x\rho}\\
B_{\rho,x,{\rm 21cm}} \propto B_{\rho,x,x} + B_{\rho,x,\rho} + B_{\rho,x,x\rho}\\
B_{\rho,\rho,{\rm 21cm}} \propto B_{\rho,\rho,x} + B_{\rho,\rho,\rho} + B_{\rho,\rho,x\rho}\\
\end{split}
\label{eq:lots_of_bispectra}
\end{equation}
where the third index corresponds to large scales, and the first and second indices are probing the same small scale. To gain intuition for what to expect from simulations, we could make some approximations that allow us to reduce each subcomponent bispectrum into power spectra, which are inexpensively estimated from simulations.

\subsubsection{$B_{x,x,\rm 21cm}$}
\label{subsec:B_xx21}
Using Equations~\ref{eq:d21cm_x} and \ref{eq:bispec_limber}, we could write:
% TOO CLOSE TO PAGE EDGE
\begin{equation}
\begin{split}
\langle \delta_x(\vec{k}_1) \delta_x(\vec{k}_2) \delta_{{\rm 21cm}}(\vec{k}_3) \rangle = 
\langle
T_0(1- \left\langle x_i \right\rangle ) \times \\
\left[
- \frac{\left\langle x_i \right\rangle}{1 - \left\langle x_i \right\rangle} \delta_x(\vec{k}_1) \delta_x(\vec{k}_2) \delta_x(\vec{k}_3) + \delta_x(\vec{k}_1) \delta_x(\vec{k}_2) \delta_{\rho}(\vec{k}_3)
 - \frac{\left\langle x_i \right\rangle}{1 - \left\langle x_i \right\rangle} \delta_x(\vec{k}_1) \delta_x(\vec{k}_2) \int \frac{{\rm d^3}k'}{(2\pi)^3} \delta_x(\vec{k}_3 - \vec{k'})\delta_{\rho}(\vec{k'})
\right] \\
\delta_D(\vec{k}_1+\vec{k}_2+\vec{k}_3) 
\rangle .
\end{split}
\label{eq:B_xx21}
\end{equation}

Taking the averages inside the square brackets reduces all the terms to the to component bispectra written in Equation~\ref{eq:lots_of_bispectra}. The first and second terms are simpler to understand, whereas the third term contains a convolution left-over from Fourier transforming $\delta_x(\vec{x})\delta_{\rho}(\vec{x})$ from Equation~\ref{eq:d21cm_x}:

\begin{equation}
\begin{split}
B_{x,x,{\rm 21cm}} = \left( -T_0 \left\langle x_i \right\rangle B_{x,x,x} + T_0(1-\left\langle x_i \right\rangle) B_{x,x,\rho} - T_0 \left\langle x_i \right\rangle \int \frac{{\rm d^3}k'}{(2\pi)^3}
\langle \delta_x(\vec{k}_1) \delta_x(\vec{k}_2) \delta_x(\vec{k}_3 - \vec{k'})\delta_{\rho}(\vec{k'}) \rangle \right)\\
\delta_D(\vec{k}_1+\vec{k}_2+\vec{k}_3)
\end{split}
\end{equation}

\subsubsection*{$B_{x,x,x}$}
\label{subsubsec:Bxxx}
Consider the bispectrum $\langle\delta_x(\vec{k}_1)\delta_x(\vec{k}_2)\delta_x(\vec{k}_3)\rangle$. In the squeezed triangle limit of $k_3 << k_1, k_2$, we concentrated on the correlator

\begin{equation}
\langle \langle \delta_x(\vec{k}_1)\delta_x(\vec{k}_2) | \delta_x(\vec{k}_3)\rangle \delta_x(\vec{k}_3) \rangle
\end{equation}
This could be interpreted as: what is the correlation between expectation value of $\delta_x(\vec{k}_1)\delta_x(\vec{k}_2)$ given that $\delta_x(\vec{k}_3)$ has some value, with the ionization overdensity field $\delta_x(\vec{k}_1)$? If $\delta_x(\vec{k}_3)$ is sufficiently small, the expectation value may be expanded as a Taylor Series:

\begin{equation}
\langle \delta_x(\vec{k}_1)\delta_x(\vec{k}_2) | \delta_x(\vec{k}_3)\rangle =
\langle \delta_x(\vec{k}_1)\delta_x(\vec{k}_2) \rangle + 
\delta_x(\vec{k}_3) \frac{{\rm d}\langle\delta_x(\vec{k}_1)\delta_x(\vec{k}_2) | \delta_x(\vec{k}_3)\rangle}{{\rm d}\delta_x(\vec{k}_3)}|_{\delta_x(\vec{k}_3)=0} + \ldots.
\end{equation}

We could evaluate the derivative by assuming that the small-scale power $P_{\delta_x,\delta_x}(\vec{k_1}) = \langle \delta_x(\vec{k}_1)\delta_x(\vec{k}_2) \rangle$ in a large-scale ionized region is \textit{identical} to a typical region some time later when $\left\langle x_i \right\rangle$ has increased. This allows us to express the subcomponent bispectrum as 

\begin{equation}
B_{x,x,x} \approx P_{\delta_x,\delta_x}(\vec{k_3}) \frac{\partial P_{\delta_x,\delta_x}(\vec{k_2}) }{\partial\delta_x}|_{x_i = \left\langle x_i \right\rangle}
\end{equation}

Using the definition of $\delta_x = (x_i - \left\langle x_i \right\rangle)/\left\langle x_i \right\rangle$, we can rewrite the derivative with respect to $\left\langle x_i \right\rangle$ and use the chain rule
\begin{equation}
B_{x,x,x} = P_{\delta_x,\delta_x}(\vec{k_1}) P_{\delta_x,\delta_x}(\vec{k_3}) \left\langle x_i \right\rangle \frac{{\rm d} \ln (P_{\delta_x,\delta_x}(\vec{k_1}))}{{\rm d}\left\langle x_i \right\rangle}
\end{equation}

\subsubsection*{$B_{x,x,\rho}$}
\label{subsubsec:Bxxrho}
This subcomponent could be neglected for our estimate, since large-scale $\delta_{\rho}$ should be negligible.

\subsubsection*{$B_{x,x,x\rho}$}
\label{subsubsec:B_xxxrho}
The third term of $B_{x,x,{\rm 21cm}}$ takes the unflattering form of

\begin{equation}
- T_0 \left\langle x_i \right\rangle \int \frac{{\rm d^3}k'}{(2\pi)^3} 
\langle \delta_x(\vec{k}_1) \delta_x(\vec{k}_2) \delta_x(\vec{k}_3 - \vec{k'})\delta_{\rho}(\vec{k'}) \rangle .
\end{equation}

Making the assumption that all our fields are Gaussian\footnote{This is a highly-idealistic assumption, but allows the mathematics to be tangible. For a fast estimator this is acceptable, but it should not be interpreted as extremely physically motivated.}, we could expand the four-point function as three products of two-point functions. Evaluating them one-at-a-time:

\begin{equation}
\int \frac{{\rm d^3}k'}{(2\pi)^3} \langle \delta_x(\vec{k}_1) \delta_x(\vec{k}_2) \rangle \langle \delta_x(\vec{k}_3 - \vec{k'})\delta_{\rho}(\vec{k'}) \rangle
\end{equation}
This vanishes, since the integration of the second term picks-out the $\vec{k}_3 = \vec{k'}$ mode. Under the squeezed triangle approximation we can send $k_3 \rightarrow 0$, and $\delta_{\rho}(k'=0)=0$.

\begin{equation}
\int \frac{{\rm d^3}k'}{(2\pi)^3} \langle \delta_x(\vec{k}_1)\delta_x(\vec{k}_3 - \vec{k'}) \rangle \langle \delta_x(\vec{k}_2) \delta_{\rho}(\vec{k'}) \rangle \approx P_{\delta_x, \delta_x}(\vec{k}_1) P_{\delta_x, \delta_{\rho}}(\vec{k}_2) 
\end{equation}
The integration of the first term selects the $\vec{k_1} + \vec{k_3} = \vec{k'}$ mode. Under the squeezed triangle approximation, $\vec{k_1} \approx \vec{k'}$.

The third integral was just a permutation of the second, above. This meant that we can write the third subcomponent bispectrum as

\begin{equation}
-T_0 \left\langle x_i \right\rangle \left(P_{\delta_x, \delta_x}(\vec{k}_1) P_{\delta_x, \delta_{\rho}}(\vec{k}_2)  + P_{\delta_x, \delta_{\rho}}(\vec{k}_1) P_{\delta_x, \delta_x}(\vec{k}_2)  \right)
\end{equation}
and the component bispectrum from this subsection can be expressed as
% OVERFLOW
\begin{equation}
B_{x,x,{\rm 21cm}} \approx 
 - T_0 \left\langle x_i \right\rangle P_{\delta_x,\delta_x}(\vec{k_1}) \left(
P_{\delta_x,\delta_x}(\vec{k_3}) \left\langle x_i \right\rangle \frac{{\rm d} \ln (P_{\delta_x,\delta_x}(\vec{k_1}))}{{\rm d}\left\langle x_i \right\rangle} 
+ P_{\delta_x, \delta_{\rho}}(\vec{k}_2)  +\frac{ P_{\delta_x, \delta_{\rho}}(\vec{k}_1)P_{\delta_x, \delta_x}(\vec{k}_2)}{P_{\delta_x,\delta_x}(\vec{k_1}) } 
\right)
\delta_D(\vec{k}_1+\vec{k}_2+\vec{k}_3)
\end{equation}

\subsubsection{$B_{x,\rho,{\rm 21cm}}$ and $B_{x,\rho,{\rm 21cm}}$}
\label{subsec:B_xrho21}

In the squeezed triangle limit, these two component bispectra are identical. Their joint contribution was:
% OVERFLOW
\begin{equation}
\begin{split}
2\langle \delta_x(\vec{k}_1) \delta_x(\vec{k}_2) \delta_{{\rm 21cm}}\rangle = \\
2T_0(1-\left\langle x_i \right\rangle) \left\langle 
- \frac{\left\langle x_i \right\rangle}{1-\left\langle x_i \right\rangle} \delta_x(\vec{k}_1) \delta_{\rho}(\vec{k}_2) \delta_x(\vec{k}_3)
+ \delta_x(\vec{k}_1) \delta_{\rho}(\vec{k}_2) \delta_{\rho}(\vec{k}_3)
- \frac{\left\langle x_i \right\rangle}{1-\left\langle x_i \right\rangle} \delta_x(\vec{k}_1) \delta_{\rho}(\vec{k}_2) \int \frac{{\rm d^3}k'}{(2\pi)^3} \delta_x(\vec{k}_3 - \vec{k'})\delta_{\rho}(\vec{k'})\right\rangle
\end{split}
\end{equation}

\subsubsection*{$B_{x,\rho,x}$}
\label{subsubsec:Bxrhox}
We could follow a similar line of reasoning as in Section~\ref{subsubsec:Bxxx} by assuming that the small-scale $\delta_x\delta_{\rho}$ cross-power in an ionized region is the same as a typical region some time later when $\langle x_i \rangle$ has increased. This allowed us to express:

\begin{equation}
B_{x,\rho,x} \approx P_{\delta_x,\delta_{\rho}}(\vec{k}_1) P_{\delta_x,\delta_x}(\vec{k}_3)\left\langle x_i \right\rangle \frac{{\rm d} \ln (P_{\delta_x,\delta_{\rho}}(\vec{k_1}))}{{\rm d}\left\langle x_i \right\rangle}
\end{equation}

\subsubsection*{$B_{x,\rho,\rho}$}
\label{subsubsec:Bxrhorho}
This subcomponent could be neglected for our estimate, since large-scale $\delta_{\rho}$ should be negligible.

\subsubsection*{$B_{x,\rho,x\rho}$}
\label{subsubsec:Bxrhoxrho}

Following the same reasoning as in Section~\ref{subsubsec:B_xxxrho}, we arrived at the expression

\begin{equation}
\delta_x(\vec{k}_1)\delta_{\rho}(\vec{k}_2) \int \frac{\rm{d}k'}{(2\pi)^3}\delta_x(\vec{k}_3-\vec{k'})\delta_{\rho}(\vec{k'})
\approx
P_{\delta_x,\delta_x}(\vec{k}_1)P_{\delta_{\rho},\delta_{\rho}}(\vec{k}_2) + P_{\delta_x,\delta_{\rho}}(\vec{k}_1)P_{\delta_x,\delta_{\rho}}(\vec{k}_2)
\end{equation}
so the component bispectrum from this subsection can be expressed as

\begin{equation}
\begin{split}
-2 T_0 \left\langle x_i \right\rangle \left( P_{\delta_x,\delta_{\rho}}(\vec{k}_1) P_{\delta_x,\delta_x}(\vec{k}_3)\left\langle x_i \right\rangle \frac{{\rm d} \ln (P_{\delta_x,\delta_{\rho}}(\vec{k_1}))}{{\rm d}\left\langle x_i \right\rangle} +  P_{\delta_x,\delta_x}(\vec{k}_1)P_{\delta_{\rho},\delta_{\rho}}(\vec{k}_2) + P_{\delta_x,\delta_{\rho}}(\vec{k}_1)P_{\delta_x,\delta_{\rho}}(\vec{k}_2) \right) \\ 
\delta_D(\vec{k}_1+\vec{k}_2+\vec{k}_3)
\end{split}
\end{equation}


\subsubsection{$B_{\rho,\rho,{\rm 21cm}}$}
\label{subsec:B_rhorho21}

This component bispectrum was much simpler to calculate, as we expected the overdensity power to be subdominant to the ionization field.

\subsubsection*{$B_{\rho,\rho,x}$}
\label{subsubsec:Brhorhox}
Following results from Section~\ref{subsubsec:B_xxxrho}, this should be negligible so long as $P_{\delta_{\rho},\delta_{\rho}}(\vec{k}_1) < P_{\delta_{x},\delta_{x}}(\vec{k}_1)$, as expected.

\subsubsection*{$B_{\rho,\rho,\rho}$}
\label{subsubsec:Brhorhorho}
This subcomponent could be neglected for our estimate, since large-scale $\delta_{\rho}$ should be negligible.

\subsubsection*{$B_{\rho,\rho,x\rho}$}
\label{subsubsec:Brhorhoxrho}

As in Sections~\ref{subsubsec:B_xxxrho} and \ref{subsubsec:Bxrhoxrho}, we could take advantage of the convolution term in the Fourier transform to obtain

\begin{equation}
\delta_x(\vec{k}_1)\delta_{\rho}(\vec{k}_2) \int \frac{\rm{d}k'}{(2\pi)^3}\delta_x(\vec{k}_3-\vec{k'})\delta_{\rho}(\vec{k'})
\approx
P_{\delta_{\rho},\delta_x}(\vec{k}_1)P_{\delta_{\rho},\delta_{\rho}}(\vec{k}_2) + P_{\delta_{\rho},\delta_{\rho}}(\vec{k}_1)P_{\delta_{\rho},\delta_x}(\vec{k}_2)
\end{equation}

So the overall component bispectrum is the above, multiplied by a factor of $-T_0\left\langle x_i \right\rangle$.

\subsubsection{Full estimator}

Under the above assumptions, and simplifying with the squeezed triangle $P_{f,f}(\vec{k}_1)\approx P_{f,f}(\vec{k}_2)$ for $f=[x,\rho]$, we obtained our estimate for the bispectrum:

\begin{equation}
\begin{split}
B_{\rm kSZ, kSZ, {\rm 21cm}}(\vec{k}_1,\vec{k}_2,\vec{k}_3) \approx \\
-2T_0\left\langle x_i \right\rangle
\left[
P_{\delta_x,\delta_x}(\vec{k}_1) \left( P_{\delta_x,\delta_x}(\vec{k}_3)\frac{\left\langle x_i \right\rangle}{2} \frac{{\rm d} \ln (P_{\delta_x,\delta_x}(\vec{k_1}))}{{\rm d}\left\langle x_i \right\rangle} + P_{\delta_x,\delta_{\rho}}(\vec{k}_1) + P_{\delta_{\rho},\delta_{\rho}}(\vec{k}_1) \right) \right. \\
\left.
+ P_{\delta_x,\delta_{\rho}}(\vec{k}_1) \left( P_{\delta_x,\delta_x}(\vec{k}_3)\left\langle x_i \right\rangle \frac{{\rm d} \ln (P_{\delta_x,\delta_{\rho}}(\vec{k}_1))}{{\rm d}\left\langle x_i \right\rangle} + P_{\delta_x,\delta_{\rho}}(\vec{k}_1) + P_{\delta_{\rho},\delta_{\rho}}(\vec{k}_1) \right)
\right]
\delta_D(\vec{k}_1+\vec{k}_2+\vec{k}_3)
\end{split}.
\end{equation}


\subsection{Counting triangles}

A closed triangle in $\ell$-space can be represented by a two-dimensional Dirac delta-distribution $\delta^{(2)}_{D}(\vec{\ell_1}+\vec{\ell_2}+\vec{\ell_3})$. A count of the different orientations of such a triangle was formulated by \cite{Joachimi.09} as:

\begin{equation}
\int^{2\pi}_0 {\rm d}\phi_{\ell1} \int^{2\pi}_0 {\rm d}\phi_{\ell2} \int^{2\pi}_0 {\rm d}\phi_{\ell3} \,\delta^{(2)}_{D}(\vec{\ell_1}+\vec{\ell_2}+\vec{\ell_3})
\end{equation}

Which can be represented as an exponential:

\begin{equation}
\begin{split}
\int^{2\pi}_0 {\rm d}\phi_{\ell1} \int^{2\pi}_0 {\rm d}\phi_{\ell2} \int^{2\pi}_0 {\rm d}\phi_{\ell3} \,\int^{\infty}_0 \frac{{\rm d^2\theta}}{(2\pi)^2}\exp\left(i(\vec{\ell_1}+\vec{\ell_2}+\vec{\ell_3})\cdot\vec{\theta}\right)\\
= (2\pi)^2\int^{\infty}_0 {\rm d}\theta \theta J_0(\ell_1\theta)J_0(\ell_2\theta)J_0(\ell_3\theta)
\end{split}
\end{equation}

Where they used the definition $J_0(x) = \int^{2\pi}_0 {\rm d}\phi e^{ix\cos\phi}/2\pi$. \cite{gradshteyn2000table} give the analytic solution for the final integral for closed-triangle configurations for triple products of any order of Bessel Function. For zeroth-order Bessel Functions, their solution reduces to the reciprocal of the area of the triangle:

\begin{equation}
(2\pi)^2\int^{\infty}_0 {\rm d}\theta \theta J_0(\ell_1\theta)J_0(\ell_2\theta)J_0(\ell_3\theta) = \left( \frac{1}{4}\sqrt{2\ell_1^2\ell_2^2 + 2\ell_1^2\ell_3^2 + 2\ell_2^2\ell_3^2 - \ell_1^4 - \ell_2^4 - \ell_3^4} \right)^{-1}
\end{equation}

In reality, measurements of any bispectrum or power spectrum will be binned in $\ell$, with central values and widths of $\bar{\ell}$ and $\Delta\ell$ respectively. The number of triangles in a kSZ$^2$-21cm bispectrum bin will be given by:

\begin{equation}
N_{\rm Tri} \approx 2\pi\Omega_S^2\bar{\ell_1}\bar{\ell_2}\bar{\ell_3}\Delta\ell_1\Delta\ell_2 \Delta\ell_3 \int^{\infty}_0 {\rm d}\theta \theta J_0(\bar{\ell_1}\theta)J_0(\bar{\ell_2}\theta)J_0(\bar{\ell_3}\theta)
\label{eq:Ntri}
\end{equation}

Signal-to-noise scales with  $\sqrt{N_{\rm Tri}}$....

\subsection{Results}

\subsubsection{Signal-to-noise estimates}
% graphs!
% but stuck at k_parallel=0...

\section{kSZ$^2$-21cm$^2$ squeezed-rectangle trispectra}
\label{sec:trispec}

\subsection{Semi-analytic approximation}

\subsection{kSZ$^2$-21cm trispectra cross-power spectrum}
%
% 
%
\chapter{Deep Learning for 21cm Observations}
\label{chapter:hera_ml}

Modern cosmological theory is capable of predicting the statistical features of many aspects of the observable Universe, using either theoretical calculations \citep[e.g.][]{Bond.91, Sheth.99} or sophisticated numerical simulations \citep[e.g.][]{Lewis.00, Vogelsberger.14}. These theories may be tested by making observations of various large-scale fields, in surveys spanning large cosmological volumes in space and time. The ultimate goal of measurements is extract from the data some parameters which are believed to describe the underlying processes, and to relate these parameters to a theoretical understanding of the physics at work. In some cases -- most conspicuously the primordial CMB -- the statistics of the fields are Gaussian, and are completely described by the two-point correlation function, or its Fourier conjugate, the power spectrum \citep[e.g.][for a review]{Liddle.00}.

A field described only by Gaussian statistics practically does not exist in cosmology beyond the CMB. For nearly every other scenario involving the non-linear interactions of gravity, radiation, and fluid mechanics, the resultant fields are non-Gaussian. Within the non-Gaussianity of these fields is encoded additional valuable information about the astrophysical processes at work, and can also serve as a cross-validation of two-point statistics of the same field \citep{Alvarez.16, Majumdar.17}. The specific details of the non-Gaussianity are not usually straightforwardly obtained from the theory, and thus devising appropriate higher-order statistics to efficiently probe the non-Gaussian information is in general a difficult problem. 

By analyzing a field using power spectra, one explicitly neglects all non-Gaussian information. In Chapter~\ref{chapter:ksz_21cm}, we presented higher-order correlation functions that are sensitive to non-Gaussian information in Fourier space. In the case of 21\,cm emission, working in Fourier space provides a natural and relatively simple way to avoid foreground contamination. Another solution could be to search for non-Gaussian information in image space, assuming some future development that could overcome the foreground challenge \citep[e.g.][]{Shaw.14, Shaw.15, Zhu.16, Patil.17}, or that we may operate on wedge-filtered image fields in a physically meaningful way \citep{Beardsley.15}. Staying in image space allows us to retain the non-Gaussian information in our data.

\section{Neural Networks}

A potential solution for parameter extraction is available due to advances in computation, allowing us to generate large numbers of numerical simulations which are realizations that capture the relevant physics of an astrophysical process \citep[e.g.][]{Mesinger.11}, and the development of deep learning algorithms which can be ``trained" to recognize patterns in data \citep[e.g.][]{Hinton.06, Hinton.12}.

Convolutional Neural Networks \citep[CNNs; e.g.][]{Lecun.95} have proven exceptionally useful for extracting non-Gaussian information from images in order to classify or extract information from their contents to a very high accuracy \citep[e.g.][]{imagenet.12}. There are many, many explanations of the inner calculus of neural networks, and the intention of this chapter is not a comprehensive review of that field. For the purposes of this chapter, a few concepts must be mentioned:

\begin{itemize}
\item Convolutional Neural Networks are systems of 1, 2 or 3-dimensional matrices that are used as convolutional kernels on an input image. An image is propagated forward through the network via consecutive convolutions by these kernels. Each kernel entry (i.e. pixel) is known as a `weight' $w$.

\item The desired output of a `training set', for example, the contents of an image, is given as a vector which the total of all the convolutions must reproduce.

\item Inevitably, if the convolutional kernels are initially randomly generated, the output vector will not contain the desired quantities. A `cost function' is a metric that specifies how `wrong' an output is. This could be the mean squared error, for example.

\item Neural networks `learn' through a process called `backpropagation'. Based on the cost function, a chain rule can be applied backwards along the network for each input, updating the values of the weights by some fraction of the user-specified `learning rate' \citep{Rumelhart.86}.

\item Associated with each weight is an `activation function', $a(x)$. The value of $a(w*x)$ (the output of the activation function given the convolved input) is actually what is handed to the next convolutional kernel along the network. Activation functions can be non-linear, allowing neural networks to learn complex decision boundaries.

\item In order to down-sample the data to a more manageable size, `pooling layers' are often implemented. These extract a moving statistic such as the moving average or maximum in a given region of the image.

\item CNNs often end with a `fully connected' or `dense' layer. These are multi-layer perceptrons \citep[e.g.][]{Rosenblatt.61} that propagate the value s of $a(wx)$ -- that is, no convolution is applied, and each layer is 1-dimensional.

\item After training on some subset of the total data (which may be done several times over), a neural network can be `tested' by forward-propagating new images, not used in training, and not backpropapating. Testing can also be implemented after some subsample of the training data has been propagated -- i.e., as the network is in the middle of training -- often called `validation'.
\end{itemize}

With this primer in mind, we will present two uses of CNNs for understanding simulated realizations of reionization: classifying the main causes (galaxies or active galactic nuclei) of reionization \citep{Sultan.18}, and regressing upon a physical parameter of interest.

\section{Reionization model classifier}

The 21\,cm power spectrum is a powerful tool for quantifying the relative clustering of large and small scaled ionized regions \citep{Hassan.17}. However, the topology of the regions themselves can provide information on the dominant mechanism of their formation. We considered two scenarios: one in which only galaxies, and the other in which only active galactic nuclei (AGN), provided ionizing photons. 

We used {\sc simfast21} \citep{Santos.10, Hassan.17.1} to generate a dark matter density field, evolve it into the non-linear regime using the Zel'dovich approximation. Dark matter halos were generated using the excursion set formalism \citep{Bond.91}. Either galaxies or AGN were placed in halos, with populations following the parametrization of \cite{Hassan.16}. Ionized regions are ``painted on top of" the dark matter halos according to parametrizations from high-resolution radiative transfer simulations and large-volume hydrodynamic simulations (see \cite{Hassan.16, Hassan.18}). An example of a galaxy-dominated and AGN-dominated reionization field is shown in Figure~\ref{fig:hassan-fields}. For this study, we focused on the field at redshift $z=8$. Galaxies produce more, small, ionized regions, whereas AGN produce larger more spherical ones. This is due to the strong clustering AGN and their harder X-ray spectrum.

\begin{figure}
\centering
\includegraphics[width=0.8\textwidth]{chapters/hera_ml/figures/hassan-field.png}
\caption[21\,cm brightness temperature fields for Galaxy-Only and AGN-only models.]{21\,cm brightness temperature fields (in arbitrary units) for Galaxy-Only (left) and AGN-only (right) models. Figure from \cite{Hassan.18}.}
\label{fig:hassan-fields}
\end{figure}

We used {\tt Tensorflow} to build a classifying CNN with 2 layers of 2-dimensional convolutional kernels interleaved with two maximum-pooling layers, a single dense layer, and an output layer. The convolutional and dense layers used the ReLU activation function, which is defined as

\begin{equation}
{\rm ReLU}(x) = 
\begin{cases} 
      0 & x < 0 \\
      x & x >0 
\end{cases}
\end{equation}

\begin{figure}
\centering
\includegraphics[width=0.8\textwidth]{chapters/hera_ml/figures/hassan-cnn.png}
\caption{The classification CNN used in this study.}
\label{fig:hassan-cnn}
\end{figure}

The network is shown in Figure~\ref{fig:hassan-cnn}. To train it, we used $\sim 1000$ images of $z=8$ realizations. Each image was a $140\times140$ greyscale image of 21\,cm brightness temperature, with a simulation box size of 75\,Mpc. Each image came from a separate simulation, which varied the photon escape fraction, X-ray spectrum of the ionizing sources and the ionizing efficiency of those sources. The testing set was $\sim 100$ additional images. To prevent over-fitting, only a random set of 75\% of neurons were used during each forward propagation (known as `dropout').

Using the {\sc 21cmSense} package \citep{Pober.14}, we could simulate the expected thermal noise of a foreground-decontaminated image cube for LOFAR, HERA-331 and SKA-Low (see Chapter~\ref{chapter:instruments}). Adding this noise to each image allowed us to make predictions of the accuracy of such a tool for predicting ionization models for actual data.

The results of training and validation are shown in Figure

\section{Reionization parameter regressor}

\section{Future directions}
This is an effort ripe for exploration...
%
% - understanding kernels
% - extraction of the bispectrum
% - cross-correlation with multiple inputs
% - visibility data (harken back to data processing chapter)
%
%
%
\chapter{Conclusions}
\label{chapter:conc}
%
%
%

\begin{appendices}
%\addcontentsline{toc}{chapter}{Appendices}
\chapter{Software}

Software engineering and maintenance of existing codebases has been, generally speaking, historically undervalued and unappreciated \citep{AstropyProblem}. In this Appendix I would like to provide a brief description of the major software packages used in this work -- without which, the work would not exist.

\section{Astronomical Interferometry in Python ({\tt aipy})}
\label{sec:aipy}

The {\tt aipy} software package \citep{aipy} was developed by a team based largely at the University of California, Berkeley and led by Aaron Parsons. Developed under NSF funding for the PAPER experiment, it provides a Python API to interact with interferometric visibilities stored in the {\sc miriad} file format \citep{miriad}. It is able to efficiently query large {\sc miriad} files due the APIs closeness to the underlying C code. It also contains calibration, deconvolution, imaging and phasing code in Python, and interfaces with {\tt HEALPix} (see Section~\ref{sec:healpix}, below) as well as other astronomical Python packages.

{\tt aipy} is maintained by the HERA software team, and can be found at: \url{https://github.com/HERA-Team/aipy}.

\section{Astronomy in Python ({\tt astropy})}

{\tt astropy} is an open-source and community-developed core Python package for Astronomy, containing a host of extremely useful utility functions and objects \citep{astropy}.

\section{Common Astronomy Software Applications ({\tt CASA})}
\label{sec:casa}

{\tt CASA} is under active development, with the primary goal of supporting the data post-processing needs of the next generation of radio telescopes. It is developed by an international consortium of scientists based at the National Radio Astronomical Observatory (NRAO), the European Southern Observatory (ESO), the National Astronomical Observatory of Japan (NAOJ), the CSIRO Australia Telescope National Facility (CSIRO/ATNF), and the Netherlands Institute for Radio Astronomy (ASTRON), under the guidance of NRAO \citep{casa}.

\section{Deep Learning packages}
\label{sec:keras_pytorch_tf}

Experimentation with deep learning analyses of 21\,cm simulated observations took place in Keras \citep{keras}, PyTorch \citep{pytorch} and Tensorflow \citep{tensorflow}.

\section{Hierarchical Equal Area isoLatitude Pixelization of the sphere ({\tt HEALPix})}
\label{sec:healpix}

The {\tt HEALPix} software, and its Python wrapper {\tt healpy}, provide a pixelization which subdivides a spherical surface into pixels which each cover the same surface area as every other pixel. Pixel centers occur on a discrete number of rings of constant latitude. This scheme makes natively spherical measurements, such as angular power spectra and wide-field images, simple and efficient to interact with \citep{healpix}.

\section{{\tt pyuvdata}}
\label{sec:pyuvdata}

{\tt pyuvdata} provides a Python interface to interferometric data. It can read and write {\sc miriad} and {\sc uvfits} file formats, as well as read {\tt CASA} measurement sets and {\tt FHD} \citep{FHD} visibility save files \citep{pyuvdata}.

{\tt pyuvdata} is maintained by the HERA software team, and can be found at: \url{https://github.com/HERA-Team/pyuvdata}.

\section{The Scientific Python Ecosystem ({\tt scipy})}
\label{sec:scipy}

Many of the above tools require at least one of the many packages under the {\tt scipy} ecosystem. It is truly foundational to almost any scientific analysis that takes place in Python \citep{ScipyEcosystem}.


\end{appendices}

\newpage
\bibliographystyle{apj_w_etal}
\addcontentsline{toc}{chapter}{Bibliography}
\bibliography{thesisbib}

\end{document}
