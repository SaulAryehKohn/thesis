\documentclass[12pt,twoside,openany]{book}
\usepackage{graphicx}
\usepackage{epsfig}
\usepackage{color}
\usepackage{amsmath,amssymb}
\usepackage{float}
\usepackage[hyphens]{url}
%\usepackage{bigints,relsize}
\usepackage{mathrsfs,mathtools,xfrac}
%\usepackage{mathtools,xfrac}
%\usepackage{subfig} % needed for proper size caption in Table A.1
\usepackage[T1]{fontenc}
\usepackage{mathptmx} % needed to make text and equations in Times New Roman 
%\usepackage{times}
\usepackage{aastex_hack,natbib}
%\usepackage{deluxetable}
%\usepackage{tabularx}
\usepackage{rotating,rotate}
\usepackage{pdflscape,longtable}
\usepackage[toc,page]{appendix}
%\usepackage{fixmainmatter} % I created this pkg to fix the blank page issue in \mainmatter
\usepackage{rotating} % Provides {sideways}{sidewaysfigure}{sidewaystable} environments
%\documentclass{aastex}
\usepackage{grffile}
%\usepackage{epstopdf}
\usepackage{lscape}
\usepackage{natbib}
\usepackage{xr}
\usepackage{url}
\usepackage{caption}
\usepackage[caption=false]{subfig}
%\usepackage{float,lscape}
%\usepackage{pdflscape}
%\usepackage{equation}
%\usepackage{fancyhdr}
%\captionsetup[deluxetable]{labelformat=empty}
\usepackage{multirow}
\usepackage{lscape}
\usepackage{bm}
\usepackage{cancel}
\usepackage{epstopdf}
\usepackage{accents}
\usepackage{multicol}
\bibstyle{aa}
\DeclareMathAlphabet{\mathcal}{OMS}{cmsy}{m}{n}
%\setlength{\topmargin}{0.9in}
\setlength{\hoffset}{0.05in}
\setlength{\textheight}{8.5in}
\setlength{\headheight}{0in}
\setlength{\headsep}{-.22in}
\setlength{\oddsidemargin}{0.55in}
\setlength{\evensidemargin}{0.55in}
\setlength{\textwidth}{5.9in}
\newcommand{\doublespaced}{\renewcommand{\baselinestretch}{2}\normalfont}
\newcommand{\singlespaced}{\renewcommand{\baselinestretch}{1}\normalfont}
\newcommand{\halfspaced}{\renewcommand{\baselinestretch}{1.5}\normalfont}
\renewcommand{\arraystretch}{0.7}
\newcommand*\rfrac[2]{{}^{#1}\!/_{#2}}
\newcommand{\avg}[1]{\ensuremath{\langle #1 \rangle}}
\newcommand{\Ang}{\; \mathring{\text{A}}}
\newcommand{\Lya}{Ly$\alpha$ }
%\newcommand{\Ang}{\mbox{ \AA}} 
\newcommand\blfootnote[1]{%
  \begingroup
  \renewcommand\thefootnote{}\footnote{#1}%
  \addtocounter{footnote}{-1}%
  \endgroup
}
\makeatletter
\newcommand{\unchapter}[1]{%
  \begingroup
  \let\@makechapterhead\@gobble % make \@makechapterhead do nothing
  \chapter{#1}
  \endgroup
}
\makeatother
%force sub subsections to be numbererd and show show up in the table of contents
\setcounter{secnumdepth}{3}
\setcounter{tocdepth}{3}
% define some shortcuts
\newcommand{\Fig}[1]{Fig.~\ref{#1}}
\newcommand{\Sec}[1]{Section~\ref{#1}}
\newcommand{\Eqn}[1]{Eq.~\ref{#1}} \newcommand{\lya}{Ly$\,\alpha$ }
\newcommand{\trec}{\ensuremath{t_{\rm rec}}}
\newcommand{\tq}{\ensuremath{t_{\rm q}}}
\newcommand{\nbar}[1]{\ensuremath{\bar{n}_{\rm #1}}}
\newcommand{\pow}[2]{\ensuremath{#1 \times 10^{#2}}}
\newcommand{\hmpc}{\ensuremath{\,h^{-1}\,{\rm Mpc}\,}}
\newcommand{\ihmpc}{\ensuremath{\,h\,{\rm Mpc^{-1}}}}
\newcommand{\K}{\mbox{ K}}
\newcommand{\bma}{\begin{math}}
\newcommand{\ema}{\end{math}}
\newcommand{\beq}{\begin{equation}}
\newcommand{\eeq}{\end{equation}}
\newcommand{\beqa}{\begin{eqnarray}}
\newcommand{\eeqa}{\end{eqnarray}}
\newcommand{\bc}{\begin{center}}
\newcommand{\ec}{\end{center}} 
\newcommand{\bit}{\begin{itemize}}
\newcommand{\eit}{\end{itemize}}
\font\BFd=cmmib10
\font\BFt=cmmib10
\font\BFs=cmmib10 scaled 700
\font\BFss=cmmib10 scaled 500
\def\bbox#1{%
\relax\ifmmode
\mathchoice
{{\hbox{\BFd #1}}}
{{\hbox{\BFt #1}}}
{{\hbox{\BFs #1}}}
{{\hbox{\BFss #1}}}
\else \mbox{#1} \fi }
\def\k{{\bbox{k}}}
\def\q{{\bbox{q}}}
\def\r{{\bbox{r}}}
\def\x{{\bbox{x}}}
\def\thetab{\pmb{\theta}}
\def\dk{\frac{d^3k}{2 \pi^3}}
\def\dq{\frac{d^3q}{2 \pi^3}}
\def\dkc{\frac{d^3k_3}{(2 \pi)^3}}
\newcommand{\MHz}{\mbox{MHz}}
%\newcommand{\vdag}{(v)^\dagger}
%\newcommand{\myemail}{tfj@sas.upenn.edu}
%\newcommand{\musicheader}[2]{\colorbox{black}{\textcolor{white}{\emph{\# #1 #2}}}}
%\newcommand{\tcode}[1]{{\tt #1}}
%\newcommand{\E}{\times 10^}
%\newcommand{\magthresh}{20.0 }
%\newcommand{\startdate}{September 1 }
%\newcommand{\findate}{February 22 }
%\newcommand{\bands}{\emph{griz}}
%\newcommand{\seeingtrigger}{1.1" }
%\newcommand{\deadtrigger}{7 days }
%\newcommand{\searchradius}{1.08 arcseconds }
%\newcommand{\assocradius}{1.80 arcseconds }
%\newcommand{\numofbfakes}{8 } 
%\newcommand{\bfakemag}{20 }
%\newcommand{\deepSNRbfakes}{80 }
%\newcommand{\shallowSNRbfakes}{20 } 
%\newcommand{\SVstart}{November 18, 2012 }
%\newcommand{\SVend}{February 22, 2013 }  
%\newcommand{\SNnum}{1200 }
%\newcommand{\hostspectra}{281 }
%\newcommand{\SNtenfivefive}{451 }
%\newcommand{\psnid}{\tcode{psnid} }
%\newcommand{\rv}{$R_{Vol}$ } 
%%%%%%%%%%%%%%%%%%%%%%%%%%%%%%%%%%%%%%%%%%%%%%%%%%%%%%%%%%%%

\newcommand{\tita}{UNDERSTANDING POLARIZATION AS A FOREGROUND FOR HI EPOCH OF REIONIZATION MEASUREMENTS}
\newcommand{\titlow}{Understanding polarization as a foreground for HI Epoch of Reionization measurements}

\begin{document}
\addcontentsline{toc}{chapter}{Title}
\frontmatter
\doublespaced
\thispagestyle{empty}
\parskip=0.3in
\begin{center}
{\tita }\\

Saul Aryeh Kohn\\

A DISSERTATION\\

in\\ 

Physics and Astronomy\\

Presented to the Faculties of the University of Pennsylvania \\
in Partial Fulfillment of the Requirements for the Degree of Doctor of Philosophy\\

2018
\end{center}
\parskip=0in

\begin{multicols}{2}
\noindent Supervisor of Dissertation \\

\begin{flushright}
\noindent Graduate Group Chairperson\\
\end{flushright}

\end{multicols}

\noindent\makebox[0in][l]{\rule[2ex]{2.8in}{.3mm}} \hspace{3.05in} \makebox[0in][l]{\rule[2ex]{2.8in}{.3mm}} 
\vspace{-.5in}
\begin{multicols}{2}
\singlespaced
\noindent James E. Aguirre\\ \small Associate Professor of Physics and Astronomy\\

\normalsize

%this is the grad chair, not the chair of the committee, right?
\begin{flushright}
Whoever the graduate chair is\\ \small Professor of Physics and Astronomy
\end{flushright}
\end{multicols}

\halfspaced
\noindent Dissertation Committee:

\noindent Adam Lidz, Associate Professor of Physics and Astronomy

\noindent Masao Sako, Associate Professor of Physics and Astronomy

\noindent another professor, Assistant Professor of Physics and Astronomy

\noindent another professor, Professor of Physics and Astronomy

\newpage

\pagestyle{plain}
\unchapter{Dedication}
\doublespaced
\vspace*{2in}
\begin{center}
{\large\emph{for my grandparents, endless sources of inspiration}}
\end{center}

\newpage

%\doublespacing

\thispagestyle{empty} % No page number as per Manual, p. 11

\vspace*{\fill}

\begin{flushleft}
{\tita }

\copyright \space COPYRIGHT
 
2018

Saul Aryeh Kohn\\[24 pt] % If traditional copyright then delete everything below here, but keep \end{flushleft}

This work is licensed under the \\
Creative Commons Attribution \\
NonCommercial-ShareAlike 3.0 \\
License

To view a copy of this license, visit

\url{http://creativecommons.org/licenses/by-nc-sa/3.0/}
\end{flushleft}


%%%%%%%%%%%%%%%%%%%%%%%%%%%%%%%%%%%%%%%%%%%%%%%%%%%%%%%%%%%%

\chapter{Acknowledgments}
\halfspaced
%blah di blah
% PAPER & HERA teams
% Danny, Carina, Matt, Josh K., Paul LP
% Friends: Ashley, Christian, Elodie, Julie, Steve K., Irteza
% James (who counts as a friend too)
% Kohns
% Goodmans
% Gabby
Acknowledgements require a certain mindset to be written well.


%%%%%%%%%%%%%%%%%%%%%%%%%%%%%%%%%%%%%%%%%%%%%%%%%%%%%%%%%%%%


\newpage
%\vspace*{.75 in}
\vspace*{.15 in}
\begin{center}
\addcontentsline{toc}{chapter}{Abstract}
{\bf ABSTRACT}\\
\tita \\
\parskip=0.2in


Saul A. Kohn\\
James E. Aguirre
\end{center}
\noindent
%350 Word Limit
%Write One
Abstracts are written last.

\vspace*{\fill}

\newpage

\singlespaced
\tableofcontents

\newpage
\phantomsection
\addcontentsline{toc}{chapter}{List of Tables}
\listoftables

\newpage
\addcontentsline{toc}{chapter}{List of Figures}
\listoffigures


\halfspaced
\setlength{\parindent}{0.25in}

%%%%%%%%%%%%%%%%%%%%%%%%%%%%%%%%%%%%%%%%%%%%%%%%%%%%%%%%%%%%


\mainmatter
\part{Introduction \& Mathematical Formalisms}
\chapter{The Epoch of Reionization} % The Epoch of Reionization
%
% what is the EoR
% why is it interesting
% current (CMB, high-z galaxies, HI limits) and future (HI, CO, C+ intensity mapping, extreme deep fields from JWST) probes
%
\chapter{Astrophysical Polarization}
 % Astrophysical Polarization
%
% what is polarization (emag)
% sources of astrophysical polarization
% [magnetic fields in SPAAAACE]
%
\chapter{Interferometry}
\label{chapter:interferometry} % Interferometry
%
% building the [classical] visibility equation in the unpolarized case
% pointing vs drift-scanning
% wide field effects
% rebuilding the visibility equation for widefield, polarized instruments
% basics of calibration for drift-scanning interferometers
% briefly, redundant calibration
% CLEAN
%
\chapter{Instrumental Polarization}
 % Instrumental Polarization
%
% explore the matrix-formalized visibility equation
% DI leakage
% DD leakage
%
\chapter{Instruments}
\section{Instruments used in this work}
\subsection{The Donald C. Backer Precision Array for Probing the Epoch of Reionization (PAPER)}
\subsubsection{PAPER-32 redundant array}
\subsubsection{PAPER-32 polarized imaging array}
\subsubsection{PAPER-64}
\subsubsection{PAPER-128}

\subsection{The Hydrogen Epoch of Reionization Array (HERA)}
\subsubsection{HERA-19 Commissioning Array}
\subsubsection{HERA-47}
\subsubsection{Future HERA Build-Outs}

\section{Other current and future low-frequency interferometers}
\subsection{The Low Frequency Array (LOFAR)}
\subsection{The Murchinson Widefield Array (MWA)}
\subsection{Square Kilometer Array -- Low band (SKA-Low)}

%
% brief overview of instruments used in this work:
%   PAPER: 32, 32-pol-img, 64 (xtlak), 128
%   HERA: 19, H1C?
% instruments of note, not used in this work, but exploring similar things and facing similar challenges:
%   LOFAR, MWA
%   SKA-Low
%
\part{Outer space in Fourier space}
\chapter{Peering through the EoR Window}

%
% -- sort of a lit review chapter --
% The issue of low frequency foregrounds for EoR measurements
% Define power spectra...
% foreground wedge and EoR window
% Foreground avoidence: delay spectrum
% Foreground subtraction: theory and LOFAR results
% Hybrid approach: MWA
%
%
\chapter{Data Preparation and Processing}
\label{chapter:data_prep_and_proc}
The data volume of interferometric measurements inherently scale as the square of the number of antennas in the array ($N_{\rm ant}$). Not only does the sheer volume of data from large-$N_{\rm ant}$ arrays pose a problem for data storage, but also it requires precise and efficient efforts to quality assure (QA) the data. 

In this chapter, I will outline some of the efforts involved in data preparation, preprocessing and QA that are required for an EoR power spectrum estimate.

\section{Data Compression}
\label{sec:data_compression}

The PAPER-128 correlator produced 288 files per night. Each of these contained 8126 baselines, and each baseline contained visibilities over 1024 frequency channels and 19 time integrations. The four instrumental polarizations were in separate files. In sum, each file was 4.2 GB which meant that each night 1.2 TB of data were recorded.

In order to efficiently transport the data over Gigabit Ethernet from the Karoo Radio Quiet Zone (KRQZ) to Cape Town, and from Cape Town under transatlantic cables to Philadelphia, some compression was required. It was also required that such a compression, while lossy, did not effect the targeted cosmological signal.

The compression algorithm implemented for PAPER observations, Delay--Delay-Rate filtering, was described in \cite{Parsons.14}, and we briefly review it below.

\section{Radio Frequency Interference}
\label{sec:RFI}


%
% compression -- DDR algorithm & software overview
% PAPER-128 RFI memo
% HERA RFI memo
% Crosstalk subtraction -- theory (PAPER-64 memo)
% PAPER QA
%
\chapter{Polarimetric Calibration}

\section{Redundant Calibration}

\section{Imaging Calibration}

%
% Omnical (2pol, 4pol, 4polminV) [omnipolcal.pdf, Dillon paper]
% Imaging calibration [briefly]
%
\chapter{The Ionosphere}

% 
% initial pass: Moore et al. 2017
% Martinot (?) et al. 2018 (?)
% radionopy
%
\chapter{A view of the EoR window from the PAPER-32 imaging array}

%
% Kohn et al. 2016
%
\chapter{A view of the EoR window from the HERA-19 commissioning array}

%
% Kohn et al. 2018 (HERA wedges)
%
\chapter{Deep integrations with PAPER-128}

%
% PAPER-128 (wherever we get to)
%
\part{Expanding the potential of EoR measurements}
% extra stuff
\chapter{Time-Averaged Visibilities}
% theory and observation of PAPER and HERA time-averaged visibilities
\chapter{Higher-order correlation functions between the kSZ and 21cm fields during the EoR}

% bispectrum & trispectrum squeezed formalisms
% noise in image space
\chapter{Deep Learning for 21cm Observations}

% CNNs on 21cmFAST - regression
% CNNs on 21cmFAST - Sultan's classification
\chapter{Conclusions}

%
% WTF how do I write this?! answer: not yet.
%


%\appendix
\begin{appendices}
\addcontentsline{toc}{chapter}{Appendices}
\chapter{Software}

Software engineering and maintenance of existing codebases has been, generally speaking, historically undervalued and unappreciated \citep{AstropyProblem}. In this Appendix I would like to provide a brief description of the major software packages used in this work -- without which, the work would not exist.

\section{Astronomical Interferometry in Python ({\tt aipy})}
\label{sec:aipy}

The {\tt aipy} software package \citep{aipy} was developed by a team based largely at the University of California, Berkeley and led by Aaron Parsons. Developed under NSF funding for the PAPER experiment, it provides a Python API to interact with interferometric visibilities stored in the {\sc miriad} file format \citep{miriad}. It is able to efficiently query large {\sc miriad} files due the APIs closeness to the underlying C code. It also contains calibration, deconvolution, imaging and phasing code in Python, and interfaces with {\tt HEALPix} (see Section~\ref{sec:healpix}, below) as well as other astronomical Python packages.

{\tt aipy} is maintained by the HERA software team, and can be found at: \url{https://github.com/HERA-Team/aipy}.

\section{Astronomy in Python ({\tt astropy})}

{\tt astropy} is an open-source and community-developed core Python package for Astronomy, containing a host of extremely useful utility functions and objects \citep{astropy}.

\section{Common Astronomy Software Applications ({\tt CASA})}
\label{sec:casa}

{\tt CASA} is under active development, with the primary goal of supporting the data post-processing needs of the next generation of radio telescopes. It is developed by an international consortium of scientists based at the National Radio Astronomical Observatory (NRAO), the European Southern Observatory (ESO), the National Astronomical Observatory of Japan (NAOJ), the CSIRO Australia Telescope National Facility (CSIRO/ATNF), and the Netherlands Institute for Radio Astronomy (ASTRON), under the guidance of NRAO \citep{casa}.

\section{Deep Learning packages}
\label{sec:keras_pytorch_tf}

Experimentation with deep learning analyses of 21\,cm simulated observations took place in Keras \citep{keras}, PyTorch \citep{pytorch} and Tensorflow \citep{tensorflow}.

\section{Hierarchical Equal Area isoLatitude Pixelization of the sphere ({\tt HEALPix})}
\label{sec:healpix}

The {\tt HEALPix} software, and its Python wrapper {\tt healpy}, provide a pixelization which subdivides a spherical surface into pixels which each cover the same surface area as every other pixel. Pixel centers occur on a discrete number of rings of constant latitude. This scheme makes natively spherical measurements, such as angular power spectra and wide-field images, simple and efficient to interact with \citep{healpix}.

\section{{\tt pyuvdata}}
\label{sec:pyuvdata}

{\tt pyuvdata} provides a Python interface to interferometric data. It can read and write {\sc miriad} and {\sc uvfits} file formats, as well as read {\tt CASA} measurement sets and {\tt FHD} \citep{FHD} visibility save files \citep{pyuvdata}.

{\tt pyuvdata} is maintained by the HERA software team, and can be found at: \url{https://github.com/HERA-Team/pyuvdata}.

\section{The Scientific Python Ecosystem ({\tt scipy})}
\label{sec:scipy}

Many of the above tools require at least one of the many packages under the {\tt scipy} ecosystem. It is truly foundational to almost any scientific analysis that takes place in Python \citep{ScipyEcosystem}.


\end{appendices}

\newpage
%\include{bibliography}
%\chapter{Bibliography}

\bibliographystyle{astron}
\addcontentsline{toc}{chapter}{Bibliography}
\bibliography{thesisbib}

\end{document}
