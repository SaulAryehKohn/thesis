\documentclass[12pt,twoside,openany]{book}
\usepackage{graphicx}
\usepackage{epsfig}
\usepackage{color}
\usepackage{amsmath,amssymb}
\usepackage{float}
\usepackage[hyphens]{url}
%\usepackage{bigints,relsize}
\usepackage{mathrsfs,mathtools,xfrac}
%\usepackage{mathtools,xfrac}
%\usepackage{subfig} % needed for proper size caption in Table A.1
\usepackage[T1]{fontenc}
\usepackage{mathptmx} % needed to make text and equations in Times New Roman 
%\usepackage{times}
\usepackage{aastex_hack,natbib}
%\usepackage{deluxetable}
%\usepackage{tabularx}
\usepackage{rotating,rotate}
\usepackage{pdflscape,longtable}
\usepackage[toc,page]{appendix}
%\usepackage{fixmainmatter} % I created this pkg to fix the blank page issue in \mainmatter
\usepackage{rotating} % Provides {sideways}{sidewaysfigure}{sidewaystable} environments
%\documentclass{aastex}
\usepackage{grffile}
%\usepackage{epstopdf}
\usepackage{lscape}
\usepackage{natbib}
\usepackage{xr}
\usepackage{url}
\usepackage{caption}
\usepackage[caption=false]{subfig}
%\usepackage{float,lscape}
%\usepackage{pdflscape}
%\usepackage{equation}
%\usepackage{fancyhdr}
%\captionsetup[deluxetable]{labelformat=empty}
\usepackage{multirow}
\usepackage{lscape}
\usepackage{bm}
\usepackage{cancel}
\usepackage{epstopdf}
\usepackage{accents}
\usepackage{multicol}
\bibstyle{aa}
\DeclareMathAlphabet{\mathcal}{OMS}{cmsy}{m}{n}
%\setlength{\topmargin}{0.9in}
\setlength{\hoffset}{0.05in}
\setlength{\textheight}{8.5in}
\setlength{\headheight}{0in}
\setlength{\headsep}{-.22in}
\setlength{\oddsidemargin}{0.55in}
\setlength{\evensidemargin}{0.55in}
\setlength{\textwidth}{5.9in}
\newcommand{\doublespaced}{\renewcommand{\baselinestretch}{2}\normalfont}
\newcommand{\singlespaced}{\renewcommand{\baselinestretch}{1}\normalfont}
\newcommand{\halfspaced}{\renewcommand{\baselinestretch}{1.5}\normalfont}
\renewcommand{\arraystretch}{0.7}
\newcommand*\rfrac[2]{{}^{#1}\!/_{#2}}
\newcommand{\avg}[1]{\ensuremath{\langle #1 \rangle}}
\newcommand{\Ang}{\; \mathring{\text{A}}}
\newcommand{\Lya}{Ly$\alpha$ }
%\newcommand{\Ang}{\mbox{ \AA}} 
\newcommand\blfootnote[1]{%
  \begingroup
  \renewcommand\thefootnote{}\footnote{#1}%
  \addtocounter{footnote}{-1}%
  \endgroup
}
\makeatletter
\newcommand{\unchapter}[1]{%
  \begingroup
  \let\@makechapterhead\@gobble % make \@makechapterhead do nothing
  \chapter{#1}
  \endgroup
}
\makeatother
%force sub subsections to be numbererd and show show up in the table of contents
\setcounter{secnumdepth}{3}
\setcounter{tocdepth}{3}
% define some shortcuts
\newcommand{\Fig}[1]{Fig.~\ref{#1}}
\newcommand{\Sec}[1]{Section~\ref{#1}}
\newcommand{\Eqn}[1]{Eq.~\ref{#1}} \newcommand{\lya}{Ly$\,\alpha$ }
\newcommand{\trec}{\ensuremath{t_{\rm rec}}}
\newcommand{\tq}{\ensuremath{t_{\rm q}}}
\newcommand{\nbar}[1]{\ensuremath{\bar{n}_{\rm #1}}}
\newcommand{\pow}[2]{\ensuremath{#1 \times 10^{#2}}}
\newcommand{\hmpc}{\ensuremath{\,h^{-1}\,{\rm Mpc}\,}}
\newcommand{\ihmpc}{\ensuremath{\,h\,{\rm Mpc^{-1}}}}
\newcommand{\K}{\mbox{ K}}
\newcommand{\bma}{\begin{math}}
\newcommand{\ema}{\end{math}}
\newcommand{\beq}{\begin{equation}}
\newcommand{\eeq}{\end{equation}}
\newcommand{\beqa}{\begin{eqnarray}}
\newcommand{\eeqa}{\end{eqnarray}}
\newcommand{\bc}{\begin{center}}
\newcommand{\ec}{\end{center}} 
\newcommand{\bit}{\begin{itemize}}
\newcommand{\eit}{\end{itemize}}
\font\BFd=cmmib10
\font\BFt=cmmib10
\font\BFs=cmmib10 scaled 700
\font\BFss=cmmib10 scaled 500
\def\bbox#1{%
\relax\ifmmode
\mathchoice
{{\hbox{\BFd #1}}}
{{\hbox{\BFt #1}}}
{{\hbox{\BFs #1}}}
{{\hbox{\BFss #1}}}
\else \mbox{#1} \fi }
\def\k{{\bbox{k}}}
\def\q{{\bbox{q}}}
\def\r{{\bbox{r}}}
\def\x{{\bbox{x}}}
\def\thetab{\pmb{\theta}}
\def\dk{\frac{d^3k}{2 \pi^3}}
\def\dq{\frac{d^3q}{2 \pi^3}}
\def\dkc{\frac{d^3k_3}{(2 \pi)^3}}
\newcommand{\MHz}{\mbox{MHz}}
%\newcommand{\vdag}{(v)^\dagger}
%\newcommand{\myemail}{tfj@sas.upenn.edu}
%\newcommand{\musicheader}[2]{\colorbox{black}{\textcolor{white}{\emph{\# #1 #2}}}}
%\newcommand{\tcode}[1]{{\tt #1}}
%\newcommand{\E}{\times 10^}
%\newcommand{\magthresh}{20.0 }
%\newcommand{\startdate}{September 1 }
%\newcommand{\findate}{February 22 }
%\newcommand{\bands}{\emph{griz}}
%\newcommand{\seeingtrigger}{1.1" }
%\newcommand{\deadtrigger}{7 days }
%\newcommand{\searchradius}{1.08 arcseconds }
%\newcommand{\assocradius}{1.80 arcseconds }
%\newcommand{\numofbfakes}{8 } 
%\newcommand{\bfakemag}{20 }
%\newcommand{\deepSNRbfakes}{80 }
%\newcommand{\shallowSNRbfakes}{20 } 
%\newcommand{\SVstart}{November 18, 2012 }
%\newcommand{\SVend}{February 22, 2013 }  
%\newcommand{\SNnum}{1200 }
%\newcommand{\hostspectra}{281 }
%\newcommand{\SNtenfivefive}{451 }
%\newcommand{\psnid}{\tcode{psnid} }
%\newcommand{\rv}{$R_{Vol}$ } 
%%%%%%%%%%%%%%%%%%%%%%%%%%%%%%%%%%%%%%%%%%%%%%%%%%%%%%%%%%%%

\newcommand{\tita}{UNDERSTANDING POLARIZATION AS A FOREGROUND FOR HI EPOCH OF REIONIZATION MEASUREMENTS}
\newcommand{\titlow}{Understanding polarization as a foreground for HI Epoch of Reionization measurements}

\begin{document}
\addcontentsline{toc}{chapter}{Title}
\frontmatter
\doublespaced
\thispagestyle{empty}
\parskip=0.3in
\begin{center}
{\tita }\\

Saul Aryeh Kohn\\

A DISSERTATION\\

in\\ 

Physics and Astronomy\\

Presented to the Faculties of the University of Pennsylvania \\
in Partial Fulfillment of the Requirements for the Degree of Doctor of Philosophy\\

2018
\end{center}
\parskip=0in

\begin{multicols}{2}
\noindent Supervisor of Dissertation \\

\begin{flushright}
\noindent Graduate Group Chairperson\\
\end{flushright}

\end{multicols}

\noindent\makebox[0in][l]{\rule[2ex]{2.8in}{.3mm}} \hspace{3.05in} \makebox[0in][l]{\rule[2ex]{2.8in}{.3mm}} 
\vspace{-.5in}
\begin{multicols}{2}
\singlespaced
\noindent James E. Aguirre\\ \small Associate Professor of Physics and Astronomy\\

\normalsize

%this is the grad chair, not the chair of the committee, right?
\begin{flushright}
Whoever the graduate chair is\\ \small Professor of Physics and Astronomy
\end{flushright}
\end{multicols}

\halfspaced
\noindent Dissertation Committee:

\noindent Adam Lidz, Associate Professor of Physics and Astronomy

\noindent Masao Sako, Associate Professor of Physics and Astronomy

\noindent another professor, Assistant Professor of Physics and Astronomy

\noindent another professor, Professor of Physics and Astronomy

\newpage

\pagestyle{plain}
\unchapter{Dedication}
\doublespaced
\vspace*{2in}
\begin{center}
{\large\emph{for my grandparents, endless sources of inspiration}}
\end{center}

\newpage

%\doublespacing

\thispagestyle{empty} % No page number as per Manual, p. 11

\vspace*{\fill}

\begin{flushleft}
{\tita }

\copyright \space COPYRIGHT
 
2018

Saul Aryeh Kohn\\[24 pt] % If traditional copyright then delete everything below here, but keep \end{flushleft}

This work is licensed under the \\
Creative Commons Attribution \\
NonCommercial-ShareAlike 3.0 \\
License

To view a copy of this license, visit

\url{http://creativecommons.org/licenses/by-nc-sa/3.0/}
\end{flushleft}


%%%%%%%%%%%%%%%%%%%%%%%%%%%%%%%%%%%%%%%%%%%%%%%%%%%%%%%%%%%%

\chapter{Acknowledgments}
\halfspaced
%blah di blah
% PAPER & HERA teams
% Danny, Carina, Matt, Josh K., Paul LP
% Friends: Ashley, Christian, Elodie, Julie, Steve K., Irteza
% James (who counts as a friend too)
% Kohns
% Goodmans
% Gabby
Acknowledgements require a certain mindset to be written well.


%%%%%%%%%%%%%%%%%%%%%%%%%%%%%%%%%%%%%%%%%%%%%%%%%%%%%%%%%%%%


\newpage
%\vspace*{.75 in}
\vspace*{.15 in}
\begin{center}
\addcontentsline{toc}{chapter}{Abstract}
{\bf ABSTRACT}\\
\tita \\
\parskip=0.2in


Saul A. Kohn\\
James E. Aguirre
\end{center}
\noindent
%350 Word Limit
%Write One
Abstracts are written last.

\vspace*{\fill}

\newpage

\singlespaced
\tableofcontents

\newpage
\phantomsection
\addcontentsline{toc}{chapter}{List of Tables}
\listoftables

\newpage
\addcontentsline{toc}{chapter}{List of Figures}
\listoffigures


\halfspaced
\setlength{\parindent}{0.25in}

%%%%%%%%%%%%%%%%%%%%%%%%%%%%%%%%%%%%%%%%%%%%%%%%%%%%%%%%%%%%


\mainmatter
\part{Introduction \& Mathematical Formalisms}
\chapter{The Epoch of Reionization} % The Epoch of Reionization
%
% what is the EoR
% why is it interesting
% current (CMB, high-z galaxies, HI limits) and future (HI, CO, C+ intensity mapping, extreme deep fields from JWST) probes
%
\chapter{Astrophysical Radiation}
\label{chapter:astro_rad} % Astrophysical Polarization
%
% what is polarization (emag)
% sources of astrophysical polarization
% [magnetic fields in SPAAAACE]
%
\chapter{Interferometry, Calibration \& Polarimetry}
\label{chapter:interferometry}

In this Chapter I wished to build a formalism around wide-field, polarized interferometric measurements that could be used throughout this work. Many traditional assumptions used in radio interferometry are broken in the case of the wide-field, fully-polarized, drift-scanning measurements native to interferometric EoR observations. In Section~\ref{sec:interferometry_vis}, I derive the equation describing the fundamental observable for an interferometer, called a ``visibility". Section~\ref{sec:interferometry_cal}, I describe calibration techniques relevant to this work and in Section~\ref{sec:interferometry_pol} I review some of the implications of the previous two sections for polarized measurements.

For a comprehensive review of interferometry from a more traditional perspective, see \cite{TMS}.

\section{The Visibility Equation}
\label{sec:interferometry_vis}

A radio interferometer (a term used interchangeably with ``interferometric array" for radio observations) is an ensemble of receiving elements, where each element's measurement is correlated with every other element's. The simplest case is a two-element interferometer, which we will focus on below. We assume (for now) that the elements are coplanar and identical.

\subsection{The Classical Visibility Equation}

Consider two receiving elements $i$ and $j$, separated by baseline vector $\vec{b}$. Suppose a plane wave of wavelength $\lambda$ is incident upon these elements, with direction of propagation $-\hat{s}$. The geometry of this interferometer is illustrated in Figure~\ref{fig:interferometry_2element}.

\begin{figure}
\centering
\includegraphics[width=0.9\textwidth]{chapters/interferometry/figures/visibility_explanation.pdf}
\caption{The geometry of a two-element interferometer, with a plane wave incident from direction $\hat{s}$.}
\label{fig:interferometry_2element}
\end{figure}

We can define the electromagnetic wave to have a frequency dependent phase, such that the electric field measured by element $i$ at time $t$ is

\begin{equation}
E_i = E_0 e^{-2\pi i \nu t}.
\end{equation}

The time difference between the arrival at $i$ and $j$ is called the ``geometrical delay", $\tau_g$:

\begin{equation}
\tau_g = \frac{\vec{b}\cdot\hat{s}}{c}
\end{equation}

and the electric field measured by element $j$ is

\begin{equation}
E_j = E_0 e^{-2\pi i \nu (t+\tau_g)}
\end{equation}

An interferometer is an instrument which correlates these electric fields together, integrating their product over some coherent time-scale. This correlation grants:

\begin{equation}
\langle E_i E_j^* \rangle 
= \lim_{T\rightarrow\inf}\frac{1}{2T}\int^T_{-T} E_i(t) E_j(t) {\rm d}t
= | E_0 |^2 e^{-2\pi i \nu \tau_g}
\end{equation}

where $e^{-2\pi i \nu \tau_g} = e^{-2\pi i \nu \vec{b}\cdot\hat{s}/c}$ is known as the ``fringe" term, due to its sinusoidal nature. We can generalize this relationship to include more than a single plane wave from direction $\hat{s}$. Many plane waves, from all directions, can be incident upon the interferometer at a given time and frequency. We can represent the power distribution on the sky as $S(\Omega)$, where $S(\Omega)$. However, no instrument is equally sensitive to radiation from every direction $\hat{s} \in \Omega$. Instead, an instrument has some sensitivity pattern -- a \textit{beam pattern} -- that tapers the power distribution on the sky into an ``observed sky",  $S'(\Omega) = A(\Omega)S(\Omega)$. 

These generalizations lead to the classical visibility equation:

\begin{equation}
V_{ij}(\nu) = \int A(\Omega, \nu) S(\Omega, \nu) e^{-2\pi i \nu \vec{b}\cdot\hat{s}/c} {\rm d}\Omega
\label{eq:classical_visibility}
\end{equation}

If we choose to represent the source direction in terms of directional cosines $\ell$ and $m$, and represent the baseline vector in units of wavelengths, $\vec{b}/\lambda=(u,v,w)$, we can perform a change of variables in Equation~\ref{eq:classical_visibility} to give

\begin{equation}
V_{ij}(u,v) = \int\int A(\ell, m)S(\ell, m) e^{-2\pi i (u\ell + vm + w\sqrt{1 - \ell^2 - m^2})} {\rm d}\ell {\rm d}m
\end{equation}

This relationship is often simplified by assuming only a small area of the sky is under observation -- that is, that $A(\ell,m)$ falls-off steeply from zenith -- and therefore $\ell^2$ and $m^2$ are small. This grants

\begin{equation}
V_{ij}(u,v) \approx e^{-2\pi i w} \int\int A(\ell, m)S(\ell, m) e^{-2\pi i (u\ell + vm)} {\rm d}\ell {\rm d}m
\end{equation}

which plainly casts $V(u,v)$ as the Fourier transform of the observed sky if $w$ is small. Modern low frequency interferometers used in this work greatly violate this approximation, the consequences of which I will discuss in the proceeding sections.

% fourier relationship, uvws, image plane, touch on deconvolution
% xx,xy,yx,yy <-> add in feed polarization indicies

\subsection{The Measurement Equation}

% building the [classical] visibility equation in the unpolarized case
% pointing vs drift-scanning
% wide field effects
% rebuilding the visibility equation for widefield, polarized instruments
% Stokes visibilities -- previous chapter defines Stokes Parameters -- discuss pseudo-ness

\section{Calibration Techniques}
\label{sec:interferometry_cal}

% basics of calibration for drift-scanning interferometers
% redundant calibration theory (more in polcal chapter)
% redundant vs imaging configurations
% CLEAN: Hogbom, 1D-CLEAN [NEEDS DELAY] (Parsons & Backer), linCLEAN

\section{Instrumental Polarization}
\label{sec:interferometry_pol}

%%% Instrumental polarization:
% explore the matrix-formalized visibility equation
% DI leakage (calibration errors)
% DD leakage (cannot calibrate away -- must model) % Interferometry
%
% building the [classical] visibility equation in the unpolarized case
% pointing vs drift-scanning
% wide field effects
% rebuilding the visibility equation for widefield, polarized instruments
% basics of calibration for drift-scanning interferometers
%
\include{chapters/instrumentpol} % Instrumental Polarization
%
% explore the matrix-formalized visibility equation
% DI leakage
% DD leakage
%
\chapter{Instruments}
\section{Instruments used in this work}
\subsection{The Donald C. Backer Precision Array for Probing the Epoch of Reionization (PAPER)}
\subsubsection{PAPER-32 redundant array}
\subsubsection{PAPER-32 polarized imaging array}
\subsubsection{PAPER-64}
\subsubsection{PAPER-128}

\subsection{The Hydrogen Epoch of Reionization Array (HERA)}
\subsubsection{HERA-19 Commissioning Array}
\subsubsection{HERA-47}
\subsubsection{Future HERA Build-Outs}

\section{Other current and future low-frequency interferometers}
\subsection{The Low Frequency Array (LOFAR)}
\subsection{The Murchinson Widefield Array (MWA)}
\subsection{Square Kilometer Array -- Low band (SKA-Low)}

%
% brief overview of instruments used in this work:
%   PAPER: 32, 32-pol-img, 64 (xtlak), 128
%   HERA: 19, H1C?
% instruments of note, not used in this work, but exploring similar things and facing similar challenges:
%   LOFAR, MWA
%   SKA-Low
%
\part{Outer space in Fourier space}
\chapter{Peering through the EoR Window}

%
% -- sort of a lit review chapter --
% The issue of low frequency foregrounds for EoR measurements
% Define power spectra...
% foreground wedge and EoR window
% Foreground avoidence: delay spectrum
% Foreground subtraction: theory and LOFAR results
% Hybrid approach: MWA
%
%
\chapter{Data Preparation and Processing}

%
% compression -- DDR algorithm & software overview
% PAPER-128 RFI memo
% HERA RFI memo
% Crosstalk subtraction -- theory (PAPER-64 memo)
% PAPER QA
%
\chapter{Polarimetric Calibration}

\section{Redundant Calibration}

\section{Imaging Calibration}

%
% Omnical (2pol, 4pol, 4polminV) [omnipolcal.pdf, Dillon paper]
% Imaging calibration [briefly]
%
\chapter{The Ionosphere}
\label{chapter:ionosphere}

The ionosphere is a section of Earth's atmosphere composed of several layers, between 60 and 1000\,km in altitude. It overlaps the Troposphere, Stratosphere, Mesosphere, Thermosphere and Exosphere. The ionosphere is an ionized plasma, composed of ions from molecules in the atmospheric layers it overlaps that are ionized by solar radiation. The ionization state of the ionosphere can be quantified by the Total Electron Content (TEC) -- an integral of electron count in a given direction -- among other metrics. 

Spatiotemporal variations of the TEC are tied to solar activity, and therefore largely both diurnal and seasonal. More ionization, and therefore a larger TEC, is to be expected in the day time and closer to the summer solstice. The Solar Cycle also influences TEC, with more sunspots proportional with a higher TEC; at solar maximum, this effect dominates the seasonal variation \citep{Sotomayor-Beltran.13}. Ionospheric variations are typically described as Kolmogorov turbulence (i.e. small scale motions are isotropic in their direction and scale with wavenumber; \citealt{Zolesi.14}), however, LOFAR observations report deviations from isotropy in their observations \citep{Intema.09, Mevius.16}. Regions of the ionosphere that can be assumed to be constant in density and shape at a given time are referred to as ``isoplanatic patches''. At 74\,MHz, these patches are observed to be $1^{\circ}-2^{\circ}$ in radius \citep{Cotton.02}.

The ionosphere is composed of three main layers: D, E and F, which vary according to the day-night cycle. These are summarized in Section~\ref{tab:ionosphere_layers} (which summarizes Chapter 2 of \citealt{Zolesi.14}). At night, there are not enough high-energy electrons to penetrate to lower altitudes, causing the D layer to recombine. The E layer increases in altitude at night due to a similar effect. The E and F layers persist at all times, but during daylight the F layer is divided into two sub-layers, F$_1$ and F$_2$. 

\begin{deluxetable}{lllll}
\centering
\label{tab:ionosphere_layers}
\tablewidth{0pt}
\tablecaption{Ionospheric Layers}
\tabletypesize{\footnotesize}
\tablehead{
\colhead{Layer} & \colhead{Time} & \colhead{Altitude} &\colhead{Components} & \colhead{Electron Density}\\
\colhead{} & \colhead{} & \colhead{km} & \colhead{} & \colhead{$e^-\,m^{-3}$}
}
\startdata
D & Day & 60--90 & NO$^+$, N$_2$, Ar, O$_2^-$ & $10^8 - 10^9$  \\
E & Day/Night & 90--150 & NO$^+$, O$_2^+$, O$^+$, N$_2^+$ & $10^{11}$ \\
F$_1$ & Day & 140--600 & NO$^+$, O$_2^+$, O$^+$, N$^+$ & $10^{11}$\\
F$_2$ & Day/Night & 220--800 & O$^+$, H$^+$, He$^+$ & $10^{10} - 10^{13}$\\
\enddata
\end{deluxetable}

The diurnal nature of the ionosphere is important to radio propagation. During the day, the D layer reflects radio transmissions much closer to the Earth than during the night, when the E and F layers reflect. This leads to longer-range transmissions being possible after sunset\footnote{This effect was first observed by E. V. Appleton \citep{Appleton.46}, confirming the ionosphere's existence, for which he was awarded the 1947 Nobel Prize in Physics.}.

The relevance of the ionosphere to this work is its coupling with Earth's magnetic field. Recall that, as mentioned in previous chapters, a linearly polarized electromagnetic wave, propagating through an ionized plasma which has an incident magnetic field, will experience Faraday Rotation of its original polarization angle $\chi$:

\begin{equation}
\chi_{\rm obs} = \chi + \phi\lambda^2
\end{equation}

where $\lambda$ is the wavelength, and

\begin{equation}
\phi(\hat{s}) \approx 0.81 \int^{\rm obs}_{\rm source} n_e(\hat{s}) \vec{B}(\hat{s}) \cdot d\vec{s}
\label{eq:ionopshere_rm}
\end{equation}

where the source of the electromagnetic wave is in direction $\hat{s}$ on the sphere, $n_e$ is the electron density scalar field and $\vec{B}$ is the magnetic vector field. The Rotation Measure (RM) $\phi$ is the integral of the product along the line of sight, and has units of rad\,m$^-2$. Since the ionosphere is capable of imparting an additional RM to polarized radio waves, inducing spectral structure to interferometric visibilities, understanding it is crucial to quantifying the effect of polarization on EoR measurements.

In this chapter, I review historical measurements of the ionospheric TEC and RM distributions in Section~\ref{sec:historicalTEC} and modern observations in Section~\ref{sec:lowfreqionosphere}. In Section~\ref{sec:widefieldRMionosphere} I present our work on the role of the ionosphere in PAPER and HERA measurements, and software we developed to quantify those effects.

\section{Historical measurements of TEC and RMs}
\label{sec:historicalTEC}

The existence and layered nature of the ionosphere was confirmed between the 1920s and the 1940s. Measurements of the TEC and RM distributions came later, once radio-communications satellites were put in orbit, and are closely tied to the Global Positioning System (GPS) launched in the late 1970s (called the NAVSTAR system). NAVSTAR GPS satellites transmit at two narrow frequency bands, centered about 1.2276\,GHz (`L$_2$') and 1.57542\,GHz (`L$_1$'). Encoded in these transmissions are the local clock times per satellite (precisely calibrated with one another and with ground clocks) and their positions. With four satellites in view of a receiver, one is capable of computing their three-dimensional position and their local clock relative deviation from the satellite clock time. 

\cite{Macdoran.89} showed that one could use a frequency-dependent time delay induced by the ionospheric plasma \citep{Klobuchar.83, Brunner.93}:

\begin{equation}
\Delta t_{\rm iono} = \frac{40.3}{c\nu^2}{\rm TEC}
\label{eq:delta_iono}
\end{equation}

to calculate an estimate of the TEC in the direction of a GPS satellite. Their approach has been continously refined. 
Using an estimate of the polarization angle of the emitted L$_{1,2}$ transmissions, \cite{Titheridge.72} and \cite{Royden.84} presented measurements of TEC by measuring the Faraday Rotation induced and worked towards an estimate of the TEC based on the RM.
\cite{Lanyi.88} showed that the more accurate method was calculation of the TEC using $\Delta t_{\rm iono}$ from Equation~\ref{eq:delta_iono}.
\cite{Mannucci.98} introduced the Ionosphere Map Exchange Format (IONEX): a method and file format for storing TEC measurements using GPS beacons across the globe, allowing the first global TEC maps to be calculated. IONEX files contain global TEC measurements with a 2 hour cadence and generally 5$^\circ$ by 2.5$^\circ$ resolution in longitude and latitude respectively. They neglect the layered nature the ionosphere, modelling it as a thin sheet.
\cite{Iijima.99} provided a server that automatically pushed IONEX files to the World Wide Web as soon as they could be constructed.
\cite{Komjathy.05} presented the first measurements with over 1000 GPS stations.
Recently, \cite{Erdogan.16} presented a method for time-series forward modelling of the TEC distribution using IONEX files.

Meanwhile, many generations of the International Geomagnetic Reference Field \citep[IGRF][]{Finlay.10} have continually improved the model of the Earth's magnetic field. This model is composed by spatial interpolation of magnetic field measurements (in up to 13th-order spherical harmonic coefficients) reported by institutions around the world.

Combining these two measurements -- IONEX and IGRF data -- can provide a map of RM distribution above any given position on Earth to moderate precision (better in the Northern Hemisphere than the Southern one, based on the number of GPS beacons in each). \cite{Afaraimovich.08} offered the first such software implementation, with the objective of using it to track Solar Activity\footnote{\cite{Erickson.01} were the first to present software capable of calculating ionospheric RMs using the IGRF, but they used local GPS beacons instead of IONEX files}. \cite{Sotomayor-Beltran.13} introduced the {\tt ionFR} package, which calculated ionospheric RMs towards a given position on the sky. We generalized their approach for the wide-field measurements in our {\tt radionopy} software package, which we present in Section~\ref{sec:widefieldRMionosphere}.

\section{Low frequency observations: discoveries and challenges}
\label{sec:lowfreqionosphere}

Low frequency interferometric observations are effected in two main ways by ionospheric turbulence: scintillation in Stokes I observations, and Faraday Rotation in Stokes Q and U observations. 

TEC variations introduce a variable index of refraction across a field of view. Stokes I signal from a point source will scintillate, change position, by an amount \citep[e.g.][]{TMS}:

\begin{equation}
\Delta\theta = - \frac{1}{8\pi^2} \frac{e^2}{\epsilon_0 m_e}\frac{1}{\nu^2} \nabla_{\perp}({\rm TEC})
\label{eq:ionosphere_scintillation}
\end{equation}

at the observed frequency $\nu$, where $\nabla_{\perp}$ is the transverse gradient in TEC towards the direction of the source. The time, space and frequency dependence of this effect causes difficulty for long integrations, since the scintillation will cause averaging of point sources with empty space, spreading-out their signal over a $\sim\Delta^2\theta$ area. This can be interpreted as an additional source of noise in a Stokes I map. \cite{Vedantham.15} showed that this scintillation noise can be much larger than image noise for baselines longer than $\sim 200$\,m. \cite{Vedantham.16}, extending the previous analysis to the Fourier domain, showed that this noise does not pose large issues to HERA or SKA-Low EoR efforts, since realistic amounts scintillation were not sufficient to wash-out EoR signals on large scales (their dense cores of relatively short baselines also help). However, it could pose large issues for point-source calibration and subtraction methods -- as emphasized in a public SKA memo by \cite{Cornwell.16}.

\cite{Loi.15} used MWA observation snapshots to map the scintillation as a function of space and time, resulting in the discovery of ``tubes'' of plasma density waves across the Southern Hemisphere in lines of roughly constant latitude. Comparing the sources in their snapshot images to source positions in the NRAO VLA Sky Survey \citep[NVSS][]{Condon.98} they were able to calculate displacement vectors, and showed that they were strongly aligned to Earth's magnetic field.

The literature surrounding ionospheric Faraday Rotation is less extensive than work focussing on the unpolarized component. \cite{Lenc.16} showed that MWA measurements of diffuse foregrounds could provide a map of ionospheric spatiotemporal variance as their RM changed throughout a series of observations. \cite{Lenc.17} showed that point source power could be seen ``twinkling'' in and out of polarized intensity maps due to ionospheric activity.

\section{Relevance for PAPER and HERA EoR measurements}
\label{sec:widefieldRMionosphere}

Within the PAPER and HERA power spectrum pipelines, many tens to hundreds of days of visibilities are averaged over during binning in LST. The the ionosphere-induced spatial and temporal fluctuations in RM could produce sufficient phase scrambling of the celestial Faraday-rotated, polarized signal to suppress a fraction of any polarized signal leaked by some mechanism into Stokes I measurements. The fringe size of the 30\,m baselines used in power spectrum analyses is large enough that scintillation effects are negligible.

This effect was first investigated in \cite{Moore.17}. Using the {\tt ionFR} package \citep{Sotomayor-Beltran.13} we calculated the RM distribution at a single zenithal pointing throughout the PAPER-32 observation season. This was a vast simplification given the PAPER primary beam was much larger than a typical isoplanatic patch. Shown in Figure~\ref{fig:ionosphere_psa32hist}, there was a large spread of ionospheric RMs for each LST. There was a decrease in the average magnitude of the RM as LST increased. This was expected, given the strong correlation between the day/night cycle and TEC values \citep[e.g.][]{Tariku.15}, and given that for this observing season, LST=4 hr corresponded to observations taken shortly after sunset, while LST=8 hr was always well into the night.

\begin{figure}
\centering
\includegraphics[width=0.9\textwidth]{chapters/ionosphere/figures/MooreHist.png}
\caption[Distribution of zenithal ionospheric RMs for 3 LSTs in the PAPER-32 observing season]{Distribution of zenithal ionospheric RMs for 3 LSTs in the PAPER-32 observing season. From top to bottom: a histogram of the zenith ionospheric RMs over the season, for the transit of LSTs 4, 6, and 8 hr. Taken from \cite{Moore.17}.}
\label{fig:ionosphere_psa32hist}
\end{figure}

Treating the single pointing as constant over the sky, we calculated the expected attenuation of polarized signal, leaked into pseudo-Stokes I visibilities, that would be averaged over varying ionospheric conditions during LST binning. These attenuation factors were 43$\pm$6\% at 165\,MHz and 7$\pm$5\% at 126\,MHz.

To build on this result, we required more sophisticated simulations of the interaction of the polarized sky with the instrument and whole-sky maps of the ionosphere. To accomplish the latter, we developed the open-source Python package {\tt radionopy}\footnote{\url{https://github.com/UPennEoR/radionopy}}. Like {\tt ionFR}, {\tt radionopy} uses GPS-derived TEC maps from IONEX files and the IGRF to estimate the value of ionospheric RM at a given latitude, longitude and date. Unlike its predecessor, {\tt radionopy} does not necessarily calculate an RM at a given pointing, but instead is capable of calculating the ionospheric RM over a {\tt HEALPix} grid of the sky \citep{healpix}. Such an expression of ionospheric variation is natural to wide-field, drift-scanning EoR arrays, and reflects the format of the IONEX input measurements, which are given in their spherical harmonic decompositions. {\tt radionopy} is vectorized, leading to efficient generation of full-sky ionospheric maps, and object-oriented, allowing for easier collaborative development. Additionally we implemented the interpolation scheme recommended in the IONEX documentation to obtain `best-guess' full-sky maps for arbitrary times between the 2-hour time resolution of IONEX data. 

An example output from {\tt radionopy} is shown in Figure~\ref{fig:ionosphere_radionopy_example} as a {\sc HEALPix} grid of the hemisphere observable from the PAPER site in the Karoo. In Figure~\ref{fig:ionfr_compare} we show {\tt radionopy} and {\tt ionFR} output for a single pointing towards Cassiopeia A (Cas A; RA=23$^{\rm h}$23$^{\rm m}$27.9$^{\rm s}$, Dec=$+58^{\circ}48'42.4''$) from the LOFAR Core site in the Netherlands. The two codes gave qualitative agreement. Slight offsets at the highest RM values that day could be attributed to differences in our interpolation schemes.

\begin{figure}
\centering
\includegraphics[width=0.9\textwidth]{chapters/ionosphere/figures/widefield_RM_snap.pdf}
\caption{An example of widefield ionospheric RMs calculated by {\tt radionopy}.}
\label{fig:ionosphere_radionopy_example}
\end{figure}

\begin{figure}
\centering
\includegraphics[width=0.9\textwidth]{chapters/ionosphere/figures/ionFRcompare.png}
\caption[The RM of Cas A as viewed from the LOFAR Core site in the Netherlands on April 11th, 2011, according to {\tt ionFR} and {\tt radionopy}.]{The RM of Cas A as viewed from the LOFAR Core site in the Netherlands on April 11th, 2011, according to {\tt ionFR} and {\tt radionopy}. The two codes show quantitative agreement, demonstrating that radionopy can be used for single-pointing as well as full-sky RM measurements.}
\label{fig:ionfr_compare}
\end{figure}

{\color{red}Martinot et al. (in prep.)} investigated the full interaction of the polarized sky with the ionosphere, using realistic polarized sky models and fully-polarized HERA beam models (see Chapter~\ref{chapter:eor_window_HERA} for an example). Their work revealed that the \cite{Moore.17} analysis overestimated the ionospheric attenuation due to their single-pointing and simple beam models. Realistic levels of attenuation for a 100 day HERA integration can be expected to reach a factor of $\leqslant 0.1$. If polarization leakage occurs close to the EoR level, this is sufficient to recover the EoR power spectrum. However, if it is above the EoR level (as expected by \citealt{Nunhokee.17}), the ionosphere alone will not be sufficient to rule out polarization leakage being detected before the EoR can be recovered.

% 
% initial pass: Moore et al. 2017
% Martinot (?) et al. 2018 (?)
% radionopy
%
\chapter{A view of the EoR window from the PAPER-32 imaging array}
\label{chapter:eor_window_paper32img}

In this Section, we present 2D power spectra created from data taken by the PAPER-32 imaging array in Stokes I, Q, U and V. 

The PAPER 32-antenna array relied on its highly redundant configuration in order to take the measurements resulting in the strong upper limits on the 21 cm power spectrum \citep{Parsons.14, Jacobs.15, Moore.17}. However, for three nights in 2011 September, the 32 elements were reconfigured into an polarized imaging configuration. 

Power spectra allowed us to observe and diagnose systematic effects in our calibration at high signal-to-noise within the Fourier space most relevant to EoR experiments. We observed well-defined windows in the Stokes visibilities, with Stokes Q, U and V power spectra sharing a similar wedge shape to that seen in Stokes I.  With modest polarization calibration, we saw no evidence that polarization calibration errors moved power outside the wedge in any Stokes visibility, to the noise levels attained.  Deeper integrations will be required to confirm that this behavior persists to the depth required for EoR detection.

The layout of this Chapter is as follows.  In Section~\ref{sec:psa32_obs} we provide a brief description of the PAPER array in its imaging configuration, the data from which this paper is based, and describe its calibration and reduction. We also describe the method used to create 2D power spectra in this section. We analyze the power spectra in Section~\ref{sec:psa32_res}, and discuss the implications of our findings and conclude in Section~\ref{sec:psa32_disc}.

\section{Observations \& Reduction}
\label{sec:psa32_obs}
We present measurements taken overnight on 2011 September 14--15 over local sidereal times (LSTs) 0--5 hr.

\begin{figure}[h!]
\centering
\includegraphics[width=0.9\columnwidth]{chapters/eor_window_PAPER/figures/new_antenna_config.pdf}
\includegraphics[width=0.9\columnwidth]{chapters/eor_window_PAPER/figures/uv_coverage_exclbad_overlaid-compressed.png}
\caption[The PAPER-32, dual-pol antenna imaging configuration and \textit{uv} distribution.]{The PAPER-32, dual-pol antenna imaging configuration (top). They were arranged in a pseudo-random scatter within in a $\sim$300\,m diameter circle to maximize instantaneous $uv$ coverage (bottom). $uv$ coverage is shown for 100--200\,MHz over 203 channels in blue, and 146--166\,MHz over 20 channels in red (the latter being the frequencies used in our power spectrum analysis). Malfunctioning antennae identified during calibration are overlaid with red crosses (and are excluded from the $uv$ coverage map).\\}
\label{fig:psa32img_config}
\end{figure}

Antennae were arranged in a pseudo-random scatter within a 300\,m-diameter circle, the layout of which is shown in Figure~\ref{fig:psa32img_config}. This allowed us to obtain resolutions between 15' and 25' across the bandwidth (100--200 MHz nominally, although in reality this extends 110--185 MHz due to band edge effects and VHF TV). Drift-scan visibilities were measured every 10.7 s, and divided into datasets about 10 minutes in length. We express an interferometric visibility $V^{pq}_{ij}$ between antennae $i$ (with dipole arm $p$, which can be $x$ (East-West) or $y$ (North-South) for PAPER dipoles), and $j$ (with dipole arm $q$), in directional cosines $l$ and $m$ for frequency $\nu$ at time $t$, as:

\begin{equation}
\begin{aligned}
V_{ij}^{pq}(\nu,t) = g^p_i g^{q*}_j \exp(-2\pi i \nu \tau_{pq}) \times \\ \int d\Omega \, A^{pq}(\Omega, \nu)
S(\Omega, \nu) \exp\left(\frac{ -i\nu}{c} \vec{b}(t) \cdot \hat{s}(\Omega)\right)
\end{aligned}
\label{eq:psa32_visibility}
\end{equation}

where the $g$ terms represent the complex gains for each antenna and dipole arm, $A^{pq}$ is the polarized beam and $S$ is the sky. The product $\vec{b}(t) \cdot \hat{s}(\Omega)$ represents the projection of the baseline between $i$ and $j$ with respect to an arbitrary location on the sky. 
The motivation for including the term for the delay between dipole arms $p$ and $q$, $\tau_{pq}$, is given in Section \ref{subsubsec:psa32_polcal}. This delay is clearly zero if $p=q$.

Visibilities were obtained from correlating both $x$ and $y$ dipoles, forming V$^{xx}$, V$^{xy}$, V$^{yx}$ and V$^{yy}$. Frequencies from 100 to 200\,MHz were sampled into 2048 channels.
Data were delay-filtered to 203 frequency channels \citep[see the Appendix of][]{Parsons.14} and Chapter~\ref{chapter:data_prep_and_proc}.  Cross-talk was modelled and removed by subtracting the average power over the 5 hours of observation, which extended across LST=0h--5h. An initial RFI-flagging removed any outliers more than $6\sigma$ from a spectrally smooth profile.

\subsection{Calibration}

Calibration took place in three stages, detailed below: a first-order delay-space calibration for the initial gains and phases with respect to Pictor A, an absolute calibration using imaging with respect to Pictor A and Fornax A, and a polarimetric correction for the $\tau_{xy}$ phase term in the V$^{xy}$ and V$^{yx}$ visibilities. 
Traditional polarimetric calibration proceeds by observing a source with a known polarization angle, and solving for up to seven direction-independent terms in the Jones matrix \citep[e.g.][]{TMS, HBS.1.96}, as well correcting for the effects of the primary beam. 
Given the dearth of suitable calibrators at our observing frequencies, especially at the relatively low resolution and sensitivity of the array, we proceeded with polarized calibration using different techniques, as described in Section~\ref{subsubsec:psa32_polcal} below.

\subsubsection{Initial calibration}
A first-order gain and phase calibration was performed by a similar approach to \citet{Jacobs.13}. Each 10~minute drift-scan dataset was phased to the known position of Pictor A using {\tt aipy} routines.

The gain term in Equation \ref{eq:visibility} was approximated as 
\begin{equation}
 g^p_i = G^p_i \exp(-2\pi i \nu \tau_{ip})
\end{equation}
and the required delay $\tau_{ip}$ offset of the uncalibrated delay tracks to the real position on the sky solved for to obtain a phase calibration; the absolute flux calibration $G^p_i$ was found by isolating the tracks of Pictor A in delay space, and applying the required flux scale across the band \citep[for a discussion of delay-space calibration, see][and Figure~\ref{fig:delay_spectra}]{Parsons.12a}.

\subsubsection{Absolute Calibration}
\label{subsubsec:psa32_abscal}

Visibilities were converted to {\tt CASA} Measurement Sets to be further calibrated using a custom pipeline developed around {\tt CASA} libraries. Snapshot images were generated for each 10~minute observation by Fourier transforming the visibilities. 
We used uniform weights and the multi-frequency synthesis algorithm to further improve the $uv$~coverage. 
Dirty images were deconvolved down to a 5~Jy threshold using the Cotton-Schwab algorithm. 
The sky model generated by the CLEAN components was used to self-calibrate each snapshot over the full bandwidth, using a frequency-independent sky-model and averaging over the 10~minute observation.
We corrected for residual cable length errors by computing antenna-based phase solutions for each frequency channel for each snapshot observation. 
After self-calibration, snapshot visibilities were again Fourier transformed into images and deconvolved down to a 2~Jy threshold to form the final sky models. 
These final sky models were used to solve for a frequency independent, diagonal, complex Jones matrix \citep{HBS.1.96,Smirnov.11} for each antenna in order to calibrate gain variations from snapshot to snapshot. 
We make no attempt to correct sky models for polarized primary beams and, therefore, our gain solutions incorporate both the direction independent and the direction dependent responses of the two gain polarizations. 
This is a reasonable approximation for the scope of the paper, as, eventually, wide-field polarization corrections cannot be implemented directly in the per-baseline 
power spectrum estimation (see Section~\ref{sec:psa32_res}).

The average correction in magnitude through this second-order calibration was a $\pm$6\% change for $x$ gains and $\pm$7\% for $y$ gains from those derived in the initial delay-space calibration.
If the gain on an antenna deviated by more than 30\% from image-to-image during this analysis, it was discarded from future processing stages, since it was likely malfunctioning. This was true for 3 antennae (see the top panel of Figure~\ref{fig:psa32img_config}). 

\begin{figure*}[h!]
\centering
\includegraphics[width=0.4\textwidth]{chapters/eor_window_PAPER/figures/zen_2455819_50285_orig-I-image.pdf}
\includegraphics[width=0.4\textwidth]{chapters/eor_window_PAPER/figures/zen_2455819_50285_orig-Q-image.pdf}
\includegraphics[width=0.4\textwidth]{chapters/eor_window_PAPER/figures/zen_2455819_50285_orig-U-image.pdf}
\includegraphics[width=0.4\textwidth]{chapters/eor_window_PAPER/figures/zen_2455819_50285_orig-V-image.pdf}
\noindent\rule{14cm}{0.6pt}
\includegraphics[width=0.4\textwidth]{chapters/eor_window_PAPER/figures/zen_2455819_50285-I-image.pdf}
\includegraphics[width=0.4\textwidth]{chapters/eor_window_PAPER/figures/zen_2455819_50285-Q-image.pdf}
\includegraphics[width=0.4\textwidth]{chapters/eor_window_PAPER/figures/zen_2455819_50285-U-image.pdf}
\includegraphics[width=0.4\textwidth]{chapters/eor_window_PAPER/figures/zen_2455819_50285-V-image.pdf}
\caption[Snapshot images of Stokes parameters before and absolute calibration.]{\textit{Above:} Example of a Stokes I snapshot image (top left) with corresponding Stokes Q (top right), Stokes U (bottom left) and Stokes V (bottom right) images before absolute calibration. \textit{Below:} The same organization as above, after absolute calibration. No primary beam correction was applied. The Stokes I image was deconvolved down to 5~Jy~beam$^{-1}$ whereas the other images were not deconvolved. We note that the Stokes Q image is relatively featureless apart from a few faint sources that appear instrumentally polarized. Stokes U and Stokes V images are, instead, dominated by Fornax A that shows instrumental polarization leaked from total intensity. Units are Jy~beam$^{-1}$; note the change in scale between polarizations and calibration stages.}
\label{fig:psa32_sky_image}
\end{figure*}

The final gain amplitude calibration was carried out similarly to \citet{Ali.15}. We generated single channel images between 120 and 174 MHz for each snapshot and deconvolved each of them down to 10 Jy. For each snapshot, a source spectrum is derived for Pictor A by fitting a two dimensional Gaussian the source using the {\tt PyBDSM}\footnote{\url{http://www.lofar.org/wiki/doku.php?id=public:user software:pybdsm}} source extractor \citep{pybdsm}. Spectra were optimally averaged together by weighting them with the primary beam model evaluated in the direction of Pictor A. To fit the absolute calibration, we divided the model spectrum \citep{Jacobs.13} by the measured one and fit a 6th order polynomial over the 120-174 MHz frequency range. This procedure was repeated using Fornax A with the only difference that a taper was applied to the visibilities (120\,m) in order to reduce Fornax A to a point-like source and use the model spectrum from \citet{Bernardi.13}. The best fit coefficients for Pictor A and Fornax A were averaged together to obtain the final absolute flux density calibration. Snapshots of fully CASA-calibrated data are shown in Figure~\ref{fig:psa32_sky_image}.

\subsubsection{Polarimetric factors}
\label{subsubsec:psa32_polcal}
Standard full polarization calibration involves correcting for leakage of Stokes~$I$ into the $V_{ij}^{xy}$ and $V_{ij}^{yx}$ visibilities and leakage of polarized signal into total intensity (the so called Jones $D$ matrices or $D$-terms; e.g. \citet[][]{TMS, HBS.1.96}), and an unknown phase difference between the $x$ and $y$ feeds \citep[e.g.][]{Sault.96}. 

We attempt no $D$ matrix calibration in this paper, as there is not a dominant source to be used for such calibration: the limited sensitivity of our observations does not offer good signal-to-noise ratio on PMN~J0351-2744, the only  polarized source at low frequencies known so far in our survey area.  In addition, $D$-term calibration would require determination of the primary beam Mueller matrices beyond our current accuracy. The consequences of this limitation are discussed in the analysis of our power spectra in Section~\ref{sec:psa32_res}.  

As a intermediate measure compatible with these limitations, we therefore adopted a minimization of the phase difference between the $V_{ij}^{xy}$ and $V_{ij}^{yx}$ visibilities, minimizing a sum of squared weighted residuals $w$:

\begin{equation}
w(\nu,\,t,\,\tau_{xy}) = \sum_{ij} | V_{ij}^{xy} - V_{ij}^{yx}\exp(-2\pi i \nu \tau_{xy}) |^2
\end{equation}
to find an estimated value of $\tau_{xy}$ for the array at each $(\nu,\,t)$ sample. This is equivalent to assuming that the sky is intrinsically not circularly polarized at the frequencies observed by PAPER.

We choose not to correct for ionospheric Faraday rotation in our calibration. Not only is this difficult to do for widefield instruments, but also the ionosphere was relatively stable during the observations, so we expect little incoherent averaging during the power spectrum stage below. We calculated the stability of ionospheric RM ($\phi_{\rm iono}$) using the {\sc ionFR} software \citep{Sotomayor-Beltran.13}, which calculates the $\phi_{\rm iono}$ for a given longitude, latitude and time by interpolating values of GPS-derived total electron content maps and the International Geomagnetic Reference Field \citep{Finlay.10}. The values of $\phi_{\rm iono}$ for different lines of sight are shown in Figure~\ref{fig:psa32img_ionosphere}. 
Fluctuations of $\phi_{\rm iono}$ will cause incoherent time-averaging and subsequent loss of polarized signal. Using the formalism of \citet{Moore.15} to calculate the attenuation factor, we found that none of the lines of sight (except for the 21h,0$^{\circ}$ one which goes beneath the horizon) shown are responsible for attenuating signal by $>\,20\%$ in power-spectrum space (see Section~\ref{subsec:psa32_create_pspec}).

\begin{figure}
\centering
\includegraphics[scale=0.45]{chapters/eor_window_PAPER/figures/ionosphere4casacal_morelines.pdf}
\caption[The values of ionospheric RM for different lines of sight a range of LSTs.]{The values of ionospheric RM for different lines of sight the range of LSTs in this analysis, as calculated by {\sc ionFR} \protect\citep{Sotomayor-Beltran.13}. The 21h,0$^{\circ}$ line of sight goes beneath the horizon after LST=3h, and therefore has fewer data points.\\ }%As an LoS gets closer to the horizon, the value of $\phi_{\rm iono}$ becomes more uncertain as projection effects become larger.}
\label{fig:psa32img_ionosphere}
\end{figure}


We form linear combinations of the instrument visibilities, the so-called pseudo-Stokes visibilities \citep[see e.g.][]{Moore.13} $V^I,\,V^Q,\,V^U$ and $V^V$ as:

\begin{equation}
\left(\begin{array}{c}
V^{I}\\
V^{Q}\\
V^{U}\\
V^{V}\end{array} \right)
=
\left( \begin{array}{cccc}
1 & 0 & 0 & 1 \\
1 & 0 & 0 & -1 \\
0 & 1 & 1 & 0 \\
0 & -i & i & 0 \end{array} \right) 
\left(\begin{array}{c}
V^{xx}\\
V^{xy}\\
V^{yx}\\
V^{yy}\end{array} \right) 
\label{eq:psa32_stokes}
\end{equation}

Data that were reduced, calibrated, and formed into Stokes visibilities were separated into delay spectra inside and outside of the horizon for each baseline. We used a 50\,ns margin for what was considered `inside' the horizon, in order to confine all supra-horizon emission \citep[e.g.][]{Parsons.12a, Pober.13} to the foreground component of the data. We implemented a one-dimensional CLEAN \citep{ParsonsBacker.09, Parsons.12b} with a Blackman-Harris window to a tolerance of $10^{-9}$. RFI is more easily identified in foreground-removed data, so we RFI-flagged again on the background data deviations greater than $3\sigma$. We then added the inside- and outside-horizon visibilities back together; RFI flags were preserved in the process. 

The effect of our calibration is shown in the delay-transformed visibilities in Figure~\ref{fig:psa32_delay_spectra}. As is apparent in Figure~\ref{fig:psa32_sky_image}, after improved calibration there are fewer delay tracks (i.e. sources) in the Stokes Q visibilities, while there is little overall change in Stokes U. The minimization of Stokes V, performed after the imaging calibration stage, moves power from Stokes V into Stokes U, effectively accounting for part of a $D$-term correction. But without an accurate $D$-term calibrator, Stokes U exhibits additional (and dominant) $D$-term leakage from Stokes I, 
in this case due to Pictor A. Pictor A is the brightest source in Stokes I in our observed field, and thus dominates the visibility shown.  There is no reason to suppose that Pictor A is pure Stokes U (compare also Figure \ref{fig:psa32_sky_image}), and thus the bulk of this emission must be leakage.

\begin{figure}[h!]
\centering
\includegraphics[width=\columnwidth]{chapters/eor_window_PAPER/figures/delayfalls_wedgeres.pdf}
\caption[The absolute value of delay-transformed visibilities over the bandwidth (146--166\,MHz) used to create the power spectra shown in this Chapter.]{The absolute value of delay-transformed visibilities over the bandwidth (146--166\,MHz) used to create the power spectra shown in this Chapter. The left and right columns show the visibilities before and after absolute calibration (and for Stokes U and V, the application of the $\tau_{xy}$ parameter), respectively, for baseline formed by antennae 6 and 14 ($\sim 250$\,m in length, approximately East-West). The flux scale in the left column as been boosted for a more fair comparison to the absolute-calibrated data. From top to bottom, the rows correspond to Stokes I, Q, U and V. 
The horizon limit is marked by white dashed lines. 
}
\label{fig:psa32_delay_spectra}
\end{figure}

\subsection{Creating power spectra}
\label{subsec:psa32_create_pspec}
 

Expressing the visibility $V_{ij}^{pq}(\nu,t)$ observed at time $t$ (see Equation~\ref{eq:psa32_visibility}) in terms of the geometrical delay $\tau_g=\vec{b}(t) \cdot \hat{s}(l,m)/c$ for the baseline $ij$, \citet{Parsons.12a} define the delay transform as the Fourier transform of the visibility along the frequency axis:
\begin{equation}
\tilde{V}_{ij}^{pq}(\tau,t) = \int d\nu \, V_{ij}^{pq}(\nu,t) e^{2\pi i \nu \tau}
%\int dl\, dm\, d\nu\, A^{pq}_{ij}(l, m, \nu) S^{pq}_{ij}(l, m, \nu) e^{-2\pi i \nu (\tau_g - \tau)}
\end{equation}

We can represent the power at each frequency and baseline in an array as a power spectrum in terms of their respective Fourier components $k_{\parallel}$ and $k_{\perp}$ as:
\begin{equation}
P(k_{\parallel},k_{\perp}) \approx |\tilde{V}_{ij}^{pq}(\tau,t)|^2 \frac{X^2Y}{\Omega B}\left(\frac{c^2}{2k_B\nu^2}\right)^2
\label{eq:psa32_pspec}
\end{equation}
where $B$ is the bandwidth, $\Omega$ is the angular area (i.e. proportional to the beam area), and X and Y are redshift-dependent scalars calculated in \citet{Parsons.12b}. 

To form $|\tilde{V}_{ij}^{pq}(\tau,t)|^2$, consecutive integrations were cross-multiplied, phasing the zenith of latter to the former i.e.:
\begin{equation}
|\tilde{V}_{ij}^{pq}(\tau,t)|^2 \approx |{V}_{ij}^{pq}(\tau,t)\times {V}_{ij}^{pq}(\tau,t+\Delta t)e^{i\theta_{ij,{\rm zen}}(\Delta t)}|^2
\end{equation}
where $\Delta t$=10.7 seconds and $\theta_{ij,{\rm zen}}(\Delta t)$ is the appropriate zenith rephasing factor. This method should avoid noise-biased power spectra except on very long baselines, which the PAPER configuration does not contain, while sampling essentially identical $k$-modes.  Note that this is the same method used by \citet{Pober.13} in their investigation of the unpolarized wedge.


%
% Kohn et al. 2016
%
\chapter{A view of the EoR window from the HERA-19 commissioning array}

%
% Kohn et al. 2018 (HERA wedges)
%
\chapter{Deep integrations with PAPER-128}

%
% PAPER-128 (wherever we get to)
%
\part{Expanding the potential of EoR measurements}
% extra stuff
\chapter{Time-Averaged Visibilities}
\label{chapter:TAV}
% theory and observation of PAPER and HERA time-averaged visibilities
\chapter{Higher-order correlation functions between the kSZ and 21cm fields during the EoR}

% bispectrum & trispectrum squeezed formalisms
% noise in image space
\chapter{Deep Learning for 21cm Observations}

% CNNs on 21cmFAST - regression
% CNNs on 21cmFAST - Sultan's classification
\chapter{Conclusions}
\label{chapter:conc}
%
% WTF how do I write this?! answer: not yet.
%


%\appendix
\begin{appendices}
\addcontentsline{toc}{chapter}{Appendices}
\chapter{Software}

Software engineering and maintenance of existing codebases has been, generally speaking, historically undervalued and unappreciated \citep{AstropyProblem}. In this Appendix I would like to provide a brief description of the major software packages used in this work -- without which, the work would not exist.

\section{Astronomical Interferometry in Python ({\tt aipy})}
\label{sec:aipy}

The {\tt aipy} software package \citep{aipy} was developed by a team based largely at the University of California, Berkeley and led by Aaron Parsons. Developed under NSF funding for the PAPER experiment, it provides a Python API to interact with interferometric visibilities stored in the {\sc miriad} file format \citep{miriad}. It is able to efficiently query large {\sc miriad} files due the APIs closeness to the underlying C code. It also contains calibration, deconvolution, imaging and phasing code in Python, and interfaces with {\tt HEALPix} (see Section~\ref{sec:healpix}, below) as well as other astronomical Python packages.

{\tt aipy} is maintained by the HERA software team, and can be found at: \url{https://github.com/HERA-Team/aipy}.

\section{Astronomy in Python ({\tt astropy})}

{\tt astropy} is an open-source and community-developed core Python package for Astronomy, containing a host of extremely useful utility functions and objects \citep{astropy}.

\section{Common Astronomy Software Applications ({\tt CASA})}
\label{sec:casa}

{\tt CASA} is under active development, with the primary goal of supporting the data post-processing needs of the next generation of radio telescopes. It is developed by an international consortium of scientists based at the National Radio Astronomical Observatory (NRAO), the European Southern Observatory (ESO), the National Astronomical Observatory of Japan (NAOJ), the CSIRO Australia Telescope National Facility (CSIRO/ATNF), and the Netherlands Institute for Radio Astronomy (ASTRON), under the guidance of NRAO \citep{casa}.

\section{Deep Learning packages}
\label{sec:keras_pytorch_tf}

Experimentation with deep learning analyses of 21\,cm simulated observations took place in Keras \citep{keras}, PyTorch \citep{pytorch} and Tensorflow \citep{tensorflow}.

\section{Hierarchical Equal Area isoLatitude Pixelization of the sphere ({\tt HEALPix})}
\label{sec:healpix}

The {\tt HEALPix} software, and its Python wrapper {\tt healpy}, provide a pixelization which subdivides a spherical surface into pixels which each cover the same surface area as every other pixel. Pixel centers occur on a discrete number of rings of constant latitude. This scheme makes natively spherical measurements, such as angular power spectra and wide-field images, simple and efficient to interact with \citep{healpix}.

\section{{\tt pyuvdata}}
\label{sec:pyuvdata}

{\tt pyuvdata} provides a Python interface to interferometric data. It can read and write {\sc miriad} and {\sc uvfits} file formats, as well as read {\tt CASA} measurement sets and {\tt FHD} \citep{FHD} visibility save files \citep{pyuvdata}.

{\tt pyuvdata} is maintained by the HERA software team, and can be found at: \url{https://github.com/HERA-Team/pyuvdata}.

\section{The Scientific Python Ecosystem ({\tt scipy})}
\label{sec:scipy}

Many of the above tools require at least one of the many packages under the {\tt scipy} ecosystem. It is truly foundational to almost any scientific analysis that takes place in Python \citep{ScipyEcosystem}.


\end{appendices}

\newpage
%\include{bibliography}
%\chapter{Bibliography}

\bibliographystyle{astron}
\addcontentsline{toc}{chapter}{Bibliography}
\bibliography{thesisbib}

\end{document}
